% !TEX ROOT=./main.tex

% Broader impact: Authors are asked to include a section in their submissions discussing the broader impact of their work, including possible societal consequences — -both positive and negative.
\section{Broader impact}
With the growing availability of mobile devices and their increasing ability to capture data, enormous amounts of information can potentially be integrated to help improve and transform people's lives in many aspects such as personalized and enhanced services~\cite{fallah2020personalized,47586} and precision medicine~\cite{uddin2019artificial,frohlich2018hype}, to name a few.

Meanwhile, data breaches from central servers of big companies have become increasingly common, raising concerns about privacy and information security when individual clients' information is aggregated and stored centrally. 

The challenge we face is therefore to properly integrate and leverage data from different sources to arrive at better solutions, while at the same time minimizing privacy risks. This trade-off between efficiency and privacy/security
lies at the heart of many applications, including 
but not limited to recommendation systems~\cite{chen2018federated}, virtual assistants~\cite{lamautonomy}, and mobile keyboard prediction~\cite{47586}.

Federated learning has emerged as a promising solution to the challenge. Built on insights from decentralized and distributed optimization, it aggregates information by integrating model \emph{parameters} learned locally at each source, eliminating the need to transmit and store data centrally. 

This desirable feature makes federated learning an effective tool to balance efficiency and privacy not just in the aforementioned applications with mobile devices, but also in other settings such as healthcare where it has the potential to save millions of lives. Under the unprecedented outbreak
of Covid'19, coronavirus patients have filled emergency rooms nationwide. It is imperative to have reliable, robust machine learning
solutions that could help doctors diagnose and treat patients. However,  Health Insurance Portability and Accountability (HIPAA) regulations~\cite{assistance2003summary}
impose strict restrictions on sharing patients' clinical data: privacy is a
top priority when developing solutions by leveraging information from different medical institutions. 
Therefore, propelling the frontiers of federated learning to develop more efficient 
algorithms under the privacy-preserving constraint has never been more
important.

Two important aspects where federated learning departs from traditional distributed optimization is the heterogeneity of data at different sources and the fact that not all local devices/sources are able to participate in the model averaging at every iteration. In this general setting, the question remains as to whether joint learning still enjoys the desirable feature of \emph{linear speedup} with respect to the number of participating devices/institutions.

In this work, we establish solid theoretical foundations for federated
learning and demonstrate that linear speedup is possible in a variety of settings with the FedAvg algorithm and its variants, under both data and system heterogeneity, significantly improving previous understanding of federated learning.

We therefore show that federated learning is able to achieve efficiency and privacy at the same time. On one hand, data is isolated in each local device/institution, preventing
the potential leakage of private information. On the other hand, joint learning can speed up substantially if the number of
participating devices/institutions increases. In conclusion, our work shows a promising machine learning pathway to simultaneously securing people's privacy and improving
their lives.



