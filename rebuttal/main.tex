\documentclass{article}

\usepackage{neurips_2020_author_response}

\usepackage[utf8]{inputenc} % allow utf-8 input
\usepackage[T1]{fontenc}    % use 8-bit T1 fonts
\usepackage{hyperref}       % hyperlinks
\usepackage{url}            % simple URL typesetting
\usepackage{booktabs}       % professional-quality tables
\usepackage{amsfonts}       % blackboard math symbols
\usepackage{nicefrac}       % compact symbols for 1/2, etc.
\usepackage{microtype}      % microtypography
\usepackage{color}
\usepackage{lipsum}
\usepackage{wrapfig}
\usepackage{graphicx}
\usepackage{caption}
\usepackage{enumitem}
\setlist{leftmargin=2.5mm}
%% common packages
\usepackage{amsbsy}
\usepackage{amsmath}
\usepackage{graphicx}
\usepackage{subfigure}
\usepackage{color}
\usepackage{booktabs}

%% to allow citation as footnote
\usepackage{natbib}
% and to reduce footnote font size
\usepackage{etoolbox}
\makeatletter
\patchcmd{\@makefntext}{\insertfootnotetext{#1}}{\insertfootnotetext{\scriptsize#1}}{}{}
\makeatother


%% macros for commenting
\usepackage[normalem]{ulem} % to use \sout
\newcommand{\remove}[1]{{\color{Gray}\sout{#1}}}
\newcommand{\revise}[1]{{\color{blue}#1}}
\newcommand{\commwy}[1]{{\color{red}(wy: #1)}} % Wotao Yin

%% template for beamer

%\mode<presentation>
%{
%  % page number
%  %\setbeamertemplate{footline}{\insertframenumber/\inserttotalframenumber}
%  \setbeamertemplate{footline}[frame number]
%
%  % background and theme
%  \setbeamertemplate{background canvas}[vertical shading][bottom=white!10,top=white!10]
%  \usetheme{default}
%
%  % section in table of contents has numbers
%  \setbeamertemplate{sections/subsections in toc}[sections numbered]
%
%  % no navigation bottoms
%  \setbeamertemplate{navigation symbols}{}
%
%  % itemize, black bullet, %150 spacing between items using "witemize"
%  \setbeamertemplate{itemize items}[circle]
%  \setbeamercolor{itemize item}{fg=black}
%  \setbeamercolor{enumerate item}{fg=black}
%  \setbeamercolor{itemize subitem}{fg=black}
%  \setbeamercolor{enumerate subitem}{fg=black}
%  \newenvironment{witemize}{\itemize\addtolength{\itemsep}{0.0\baselineskip}}{\enditemize}
%
%  % block, black over gray with no shadow
%  \setbeamertemplate{blocks}[rounded][shadow=false]
%  \setbeamercolor{block title}{fg=black,bg=gray!40}
%  \setbeamercolor{block body}{fg=black,bg=gray!10}
%
%  % frametitle, bold, black, centered
%  \setbeamertemplate{frametitle}[default][center]
%  \setbeamercolor{frametitle}{fg=black}
%  \setbeamerfont{frametitle}{shape=\bfseries}
%
%  % line spacing
%  \linespread{1.2}
%
%  \setlength{\parskip}{\smallskipamount}
%}

%% macros for letters

\newcommand{\va}{{\mathbf{a}}}
\newcommand{\vb}{{\mathbf{b}}}
\newcommand{\vc}{{\mathbf{c}}}
\newcommand{\vd}{{\mathbf{d}}}
\newcommand{\ve}{{\mathbf{e}}}
\newcommand{\vf}{{\mathbf{f}}}
\newcommand{\vg}{{\mathbf{g}}}
\newcommand{\vh}{{\mathbf{h}}}
\newcommand{\vi}{{\mathbf{i}}}
\newcommand{\vj}{{\mathbf{j}}}
\newcommand{\vk}{{\mathbf{k}}}
\newcommand{\vl}{{\mathbf{l}}}
\newcommand{\vm}{{\mathbf{m}}}
\newcommand{\vn}{{\mathbf{n}}}
\newcommand{\vo}{{\mathbf{o}}}
\newcommand{\vp}{{\mathbf{p}}}
\newcommand{\vq}{{\mathbf{q}}}
\newcommand{\vr}{{\mathbf{r}}}
\newcommand{\vs}{{\mathbf{s}}}
\newcommand{\vt}{{\mathbf{t}}}
\newcommand{\vu}{{\mathbf{u}}}
\newcommand{\vv}{{\mathbf{v}}}
\newcommand{\vw}{{\mathbf{w}}}
\newcommand{\vx}{{\mathbf{x}}}
\newcommand{\vy}{{\mathbf{y}}}
\newcommand{\vz}{{\mathbf{z}}}

\newcommand{\vA}{{\mathbf{A}}}
\newcommand{\vB}{{\mathbf{B}}}
\newcommand{\vC}{{\mathbf{C}}}
\newcommand{\vD}{{\mathbf{D}}}
\newcommand{\vE}{{\mathbf{E}}}
\newcommand{\vF}{{\mathbf{F}}}
\newcommand{\vG}{{\mathbf{G}}}
\newcommand{\vH}{{\mathbf{H}}}
\newcommand{\vI}{{\mathbf{I}}}
\newcommand{\vJ}{{\mathbf{J}}}
\newcommand{\vK}{{\mathbf{K}}}
\newcommand{\vL}{{\mathbf{L}}}
\newcommand{\vM}{{\mathbf{M}}}
\newcommand{\vN}{{\mathbf{N}}}
\newcommand{\vO}{{\mathbf{O}}}
\newcommand{\vP}{{\mathbf{P}}}
\newcommand{\vQ}{{\mathbf{Q}}}
\newcommand{\vR}{{\mathbf{R}}}
\newcommand{\vS}{{\mathbf{S}}}
\newcommand{\vT}{{\mathbf{T}}}
\newcommand{\vU}{{\mathbf{U}}}
\newcommand{\vV}{{\mathbf{V}}}
\newcommand{\vW}{{\mathbf{W}}}
\newcommand{\vX}{{\mathbf{X}}}
\newcommand{\vY}{{\mathbf{Y}}}
\newcommand{\vZ}{{\mathbf{Z}}}

\newcommand{\cA}{{\mathcal{A}}}
\newcommand{\cB}{{\mathcal{B}}}
\newcommand{\cC}{{\mathcal{C}}}
\newcommand{\cD}{{\mathcal{D}}}
\newcommand{\cE}{{\mathcal{E}}}
\newcommand{\cF}{{\mathcal{F}}}
\newcommand{\cG}{{\mathcal{G}}}
\newcommand{\cH}{{\mathcal{H}}}
\newcommand{\cI}{{\mathcal{I}}}
\newcommand{\cJ}{{\mathcal{J}}}
\newcommand{\cK}{{\mathcal{K}}}
\newcommand{\cL}{{\mathcal{L}}}
\newcommand{\cM}{{\mathcal{M}}}
\newcommand{\cN}{{\mathcal{N}}}
\newcommand{\cO}{{\mathcal{O}}}
\newcommand{\cP}{{\mathcal{P}}}
\newcommand{\cQ}{{\mathcal{Q}}}
\newcommand{\cR}{{\mathcal{R}}}
\newcommand{\cS}{{\mathcal{S}}}
\newcommand{\cT}{{\mathcal{T}}}
\newcommand{\cU}{{\mathcal{U}}}
\newcommand{\cV}{{\mathcal{V}}}
\newcommand{\cW}{{\mathcal{W}}}
\newcommand{\cX}{{\mathcal{X}}}
\newcommand{\cY}{{\mathcal{Y}}}
\newcommand{\cZ}{{\mathcal{Z}}}

\newcommand{\ri}{{\mathrm{i}}}
\newcommand{\rr}{{\mathrm{r}}}


%% macros for math notions and operators

\newcommand{\RR}{\mathbb{R}}
\newcommand{\EE}{\mathbb{E}}
\newcommand{\CC}{\mathbb{C}}
\newcommand{\ZZ}{\mathbb{Z}}
\renewcommand{\SS}{{\mathbb{S}}}
\newcommand{\SSp}{\mathbb{S}_{+}}
\newcommand{\SSpp}{\mathbb{S}_{++}}
\newcommand{\sign}{\mathrm{sign}}
\newcommand{\Sign}{\mathrm{Sign}}
\newcommand{\vzero}{\mathbf{0}}
\newcommand{\vone}{{\mathbf{1}}}
\newcommand{\Null}{{\mathrm{Null}}}
\newcommand{\dist}{{\mathrm{dist}}}
\newcommand{\Co}{{\mathbf{Co}}}

\newcommand{\op}{{\mathrm{op}}} % subscript for operator norm
\newcommand{\opt}{{\mathrm{opt}}} % subscript for optimal solution
\newcommand{\supp}{{\mathrm{supp}}} % support
\newcommand{\Prob}{{\mathrm{Prob}}} % probability
\newcommand{\Diag}{{\mathrm{Diag}}} % vector -> diagonal matrix
\newcommand{\diag}{{\mathrm{diag}}} % matrix diagonal -> vector
\newcommand{\dom}{{\mathrm{dom}}} % domain
\newcommand{\grad}{{\nabla}}    % gradient
\newcommand{\tr}{{\mathrm{tr}}} % trace
\newcommand{\TV}{{\mathrm{TV}}} % total variation
\newcommand{\Proj}{{\mathrm{Proj}}}
\DeclareMathOperator{\shrink}{shrink} % shrinkage
\DeclareMathOperator*{\argmin}{arg\,min}
\DeclareMathOperator*{\argmax}{arg\,max}
\DeclareMathOperator*{\mini}{minimize}
\DeclareMathOperator*{\maxi}{maximize}
\DeclareMathOperator*{\Min}{minimize}
\DeclareMathOperator*{\Max}{maximize}
\newcommand{\prox}{{\mathbf{prox}}}
\newcommand{\st}{{\quad\text{s.t.}~}}

%% macros for environments math equations

\newcommand{\MyFigure}[1]{../fig/#1}

\newcommand{\bc}{\begin{center}}
\newcommand{\ec}{\end{center}}

\newcommand{\bdm}{\begin{displaymath}}
\newcommand{\edm}{\end{displaymath}}

\newcommand{\beq}{\begin{equation}}
\newcommand{\eeq}{\end{equation}}

\newcommand{\bfl}{\begin{flushleft}}
\newcommand{\efl}{\end{flushleft}}

\newcommand{\bt}{\begin{tabbing}}
\newcommand{\et}{\end{tabbing}}

\newcommand{\beqn}{\begin{eqnarray}}
\newcommand{\eeqn}{\end{eqnarray}}

\newcommand{\beqs}{\begin{align*}} % no equation numbers
\newcommand{\eeqs}{\end{align*}}  % no equation numbers

%% macros for theorem-like environments

% \newtheorem{theorem}{Theorem}
% \newtheorem{condition}{Condition}
\newtheorem{assumption}{Assumption}
\newtheorem{definition}{Definition}
% \newtheorem{corollary}{Corollary}
% \newtheorem{remark}{Remark}
% \newtheorem{lemma}{Lemma}
% \newtheorem{proof}{Proof}
% \newtheorem{proof*}{Proof}
% \newtheorem{proposition}{Proposition}
%\newtheorem{example}{Example}

% \newtheorem{example}[remark]{Example}



\newcommand{\ov}[1]{{\overline{\mathbf{#1}}}}
\begin{document}
The authors would like to thank three reviewers for their valuable feedback. 
% Based on their comments, upon acceptance of this paper, we shall improve in 
% the final version (a) Refine the presentation, fix typo, figures, and properly 
% located assumptions etc. (b) More experiments in terms of 
Our point-to-point response are provided below.

% {\color{blue}\textbf{Reviewer 1:}} We thank the reviewer for the valuable feedback. Your concerns are addressed as follows: 

\textbf{\textit{1. Improvement compared to [20] and [21].}} Compared to [21], the improvement 
of convergence comes from the improved bound of $2\alpha_{t}\sum_{k=1}^{N}p_{k}\left[F_{k}(\vw^{\ast})-F_{k}(\ov{w}_{t})\right]$ (as we mentioned in L489, Appendix C.1). Because of this improvement, we are able to cancel square of gradient norm ($\alpha_{t}^{2}\|\ov{g}_{t}\|^{2}$) in the upper bound of A1 (defined in L483, also see explanations in L492) while in [21] this term remains as leading term ($6\eta_t^2L \Gamma$). 

Our work is not an extension of [20], we have discussed that our assumption is
a more relaxed assumption compared to the bounded gradient diversity in [20] (see L44 and Appendix B). 

\textbf{\textit{2. The overparameterized case seem to be out of context here with no experimental results to back them up.}} The overparameterized 
setting is an important setting [29, 30], especially in the era of deep learning
where the neural network can achieve zero training loss. We provide the 
geometric rate with linear speed up in this setting including a general overparameterized setting and overparameterized linear regression. 
Experimentally, we verify the linear speedup results of overparameterized setting in 3rd column, Figure 1 and Figure 2. 

\textbf{\textit{3. The novelty and main results.}}
We would like to emphasize that the techniques used for different settings
to achieve linear speedup are quite different (strongly convex see L489, L492, convex smooth see L553). The main contributions include
the linear speedup results in strongly convex, convex smooth, and overparameterized
setting, for both FedAvg and its accelerated variants (this is the first linear speedup results of accelerated FedAvg). We further propose FedMass,
which is the first algorithm that enjoys an improved convergence rate comparing to
FedAvg. We respectfully disagree that our main contribution is the improvement over [21] as the linear speedup results under other settings are not a trivial extension of the strongly convex case.



{\color{blue}\textbf{Reviewer 1:}} We thank the reviewer for the valuable feedback. Your concerns are addressed as follows: 

\textbf{\textit{1. Improvement compared to [20] and [21].}} Compared to [21], the improvement 
of convergence comes from the improved bound of $2\alpha_{t}\sum_{k=1}^{N}p_{k}\left[F_{k}(\vw^{\ast})-F_{k}(\ov{w}_{t})\right]$ (as we mentioned in L489, Appendix C.1). Because of this improvement, we are able to cancel square of gradient norm ($\alpha_{t}^{2}\|\ov{g}_{t}\|^{2}$) in the upper bound of A1 (defined in L483, also see explanations in L492) while in [21] this term remains as leading term ($6\eta_t^2L \Gamma$). 

Our work is not an extension of [20], we have discussed that our assumption is
a more relaxed assumption compared to the bounded gradient diversity in [20] (see L44 and Appendix B). 

\textbf{\textit{2. The overparameterized case seem to be out of context here with no experimental results to back them up.}} The overparameterized 
setting is an important setting [29, 30], especially in the era of deep learning
where the neural network can achieve zero training loss. We provide the 
geometric rate with linear speed up in this setting including a general overparameterized setting and overparameterized linear regression. 
Experimentally, we verify the linear speedup results of overparameterized setting in 3rd column, Figure 1 and Figure 2. 

\textbf{\textit{3. The novelty and main results.}}
We would like to emphasize that the techniques used for different settings
to achieve linear speedup are quite different (strongly convex see L489, L492, convex smooth see L553). The main contributions include
the linear speedup results in strongly convex, convex smooth, and overparameterized
setting, for both FedAvg and its accelerated variants (this is the first linear speedup results of accelerated FedAvg). We further propose FedMass,
which is the first algorithm that enjoys an improved convergence rate comparing to
FedAvg. We respectfully disagree that our main contribution is the improvement over [21] as the linear speedup results under other settings are not a trivial extension of the strongly convex case.


% {\color{blue}\textbf{Reviewer 2:}} We thank the reviewer for the valuable feedback. Your concerns are addressed as follows: \\
\textbf{\textit{1.The range of N/K.}}


{\color{blue}\textbf{Reviewer 2:}} We thank the reviewer for the valuable feedback. Your concerns are addressed as follows: \\
\textbf{\textit{1.The range of N/K.}}

% {\color{blue}\textbf{Reviewer 3:}} We thank the reviewer for the valuable feedback. Your concerns are addressed as follows: \\
\textbf{\textit{1.Comparison with SCAFFOLD.}}: 
1. The analyses in SCAFFOLD do not imply $E=\cO(\sqrt{T})$ but in fact also require $E=O(1)$ under partial participation (sampling). To see this, it is easier to examine Theorem V in their paper, which is stated in terms of optimality gaps, but otherwise equivalent to their Theorem I. Notice that when $S<N$, the second term in $M^2$ gives rise to a term $O(1/R)$ in the bound for the strongly convex case (ignoring constants), and since $R=T/E$ in our notation, this term is $O(E/T)$. Therefore, in order to achieve a $O(1/T)$ convergence rate for the optimality gap, it must be the case that $E=O(1)$ as well. Similarly for the general convex case. Thus in terms of communication complexity, our results imply the same requirements for both full and partial participation as that in SCAFFOLD. 

2. Furthermore, the sampling procedure analyzed in our paper is different from that in SCAFFOLD, as we allow the sampling probability to scale with device specific weights and sample with replacement, whereas in SCAFFOLD the sampling is uniform without replacement. 

3. Under our sampling schemes, we explicitly analyze the contribution of sampling variance to the optimality gap, and in Lemma 10 of our paper in the Appendix, we provided a problem instance that lower bounds the sampling variance, showing that $E=O(1)$ cannot be improved in general when there is partial participation with sampling.

4. As there has not been any linear speedup analysis of Nesterov accelerated FedAvg for convex problems, our result is a first in this regard and completes the picture. Our analyses of FedAvg and Nesterov FedAvg are also unified, highlighting the common elements and distinctions for the two algorithms, which has not been done by previous studies. 

5. Finally, our geometric rates in Theorem 5 are for general overparameterized strongly convex objectives rather than just linear regression, and as SCAFFOLD is a variance reduction based algorithm, our result, which is on the FedAvg algorithm, is not directly comparable to the geometric convergence result in SCAFFOLD.





\textbf{\textit{2. The convergence of Nestrov Accelerated FedAvg.}}



\textbf{\textit{3. The geometrical rates of overparametrized linear regression.}}


{\color{blue}\textbf{Reviewer 3:}} We thank the reviewer for the valuable feedback. Your concerns are addressed as follows: \\
\textbf{\textit{1.Comparison with SCAFFOLD.}}: 
\begin{itemize}
	\item The analyses in SCAFFOLD do not imply $E=\cO(\sqrt{T})$ but in fact also require $E=O(1)$ under partial participation (sampling). To see this, it is easier to examine Theorem V in their paper, which is stated in terms of optimality gaps, but otherwise equivalent to their Theorem I. Notice that when $S<N$, the second term in $M^2$ gives rise to a term $O(1/R)$ in the bound for the strongly convex case (ignoring constants), and since $R=T/E$ in our notation, this term is $O(E/T)$. Therefore, in order to achieve a $O(1/T)$ convergence rate for the optimality gap, it must be the case that $E=O(1)$ as well. Similarly for the general convex case. Thus in terms of communication complexity, our results imply the same requirements for both full and partial participation as that in SCAFFOLD. 
	\item Furthermore, the sampling procedure analyzed in our paper is different from that in SCAFFOLD, as we allow the sampling probability to scale with device specific weights and sample with replacement, whereas in SCAFFOLD the sampling is uniform without replacement. 
	\item  Under our sampling schemes, we explicitly analyze the contribution of sampling variance to the optimality gap, and in Lemma 10 of our paper in the Appendix, we provided a problem instance that lower bounds the sampling variance, showing that $E=O(1)$ cannot be improved in general when there is partial participation with sampling.
\end{itemize}

\textbf{\textit{2. The convergence of Nestrov Accelerated FedAvg.}}
As there has not been any linear speedup analysis of Nesterov accelerated FedAvg for convex problems, our result is a first in this regard and completes the picture. Our analyses of FedAvg and Nesterov FedAvg are also unified, highlighting the common elements and distinctions for the two algorithms, which has not been done by previous studies. 

\textbf{\textit{3. The geometrical rates of overparametrized linear regression.}}
The geometric rates in Theorem 5 are for general overparameterized strongly convex objectives rather than just linear regression, and as SCAFFOLD is a variance reduction based algorithm, our result, which is on the FedAvg algorithm, is not directly comparable to the geometric convergence result in SCAFFOLD.


    
\end{document}