% !TEX ROOT=./main.tex



\section{Exponential Convergence of FedAvg in the Overparameterized Setting}

So far we have shown $O(1/\sqrt{NT})$ and $O(1/NT)$ convergence
rates for FedAvg with local SGD and Nesterov udpates in the convex
and strongly convex settings, respectively. In this section, we turn
to the special setting of overparameterized problems, where a non-negative
loss function admits a zero-loss global minimizer. Such problems are
very common in machine learning systems where the number of parameters
far exceeds the number of data points, and perfect fitting is possible. 

There have been a line of recent works {[}CITE{\cite{ma2017power}]} showing that SGD
and accelerated methods achieve exponential convergence in this case,
thanks to the property of ``automatic variance reduction'' {[}CITE{]}.
The natural question is whether such a result also holds in the federated
learning setting. We first show that this is indeed the case and establish
the exponential (linear) convergence of FedAvg with local SGD updates
with constant step size for general strongly convex and smooth overparameterized
problems. In addition, we show that the convergence rate speeds up
linearly in the number of workers $N$ when $N$ is below some problem-dependent
threhold, while decreases with $E$ through $1/E$. 

To our knowledge, this is the first result on the exponential convergence
of FedAvg algorithms in the interpolation setting with linear speedup
in the number of workers and explicit dependence on the communication
interval $E$. We next sharpen this exponential convergence rate in
the special case of linear regression, and show that it is $O(\exp(-\frac{NT}{E\kappa_{1}}))$,
for $\kappa_{1}$ an appropriately defined condition number of the
system. 

Lastly, we turn to the question of whether FedAvg with momentum-based
local updates can outperform FedAvg with SGD updates. In contrast
to the gradient descent setting, in the single-agent stochastic gradient
setting, Nesterov and Heavy Ball updates are known to fail to accelerate
over SGD, both in the overparameterized setting and standard convex
setting{[}CITE{]}. Thus in general we cannot hope to obtain acceleration
results for the FedAvg algorithm with Nesterov and Heavy Ball updates.
On the other hand, in the overparameterized setting, {[}CITE Belkin{]}
introduced the MaSS algorithm, which is a modification of the Nesterov
update, where in the update a non-negative multiple of the gradient
is added to the Nesterov parameter update to correct for the ``over-descent''
of the Nesterov update. For quadratic objectives, the authors show
that the MaSS algorithm is able to achieve acceleration over the exponential
convergence of SGD both theoretically and empirically. In the last
part of this section, we introduce a new accelerated FedAvg algorithm
by adapting the MaSS algorithm to the federated learning setting.
For this new accelerated FedAvg algorithm with local MaSS updates,
we show that it achieves exponential convergence for overparameterized
quadratic problems with rate $O(\exp(-\frac{NT}{E\sqrt{\kappa_{1}\tilde{\kappa}}}))$,
where $\tilde{\kappa}$ is the ``statistical condition number''
introduced in {[}CITE{]} that satisfies $\tilde{\kappa}\leq\kappa_{1}$.
Thus the FedAvg algorithm with local MaSS updates preserves the linear
speedup property in the number of workers and the $1/E$ dependence
on $E$, while achieving a speedup of factor $\sqrt{\frac{\kappa_{1}}{\tilde{\kappa}}}$
over FedAvg with local SGD updates. 

\subsection{Exponential Convergence and Linear Speedup of FedAvg with local SGD }

In {[}CITE Belkin{]}, the authors show that in the overparameterized
setting, SGD achieves exponential convergence with linear speedup
in batch size that is below some critical threshold. In this section,
we extend their result to the federated learning setting and show
that a similar exponential convergence with linear speedup in the
number of workers using FedAvg with local SGD updates holds. 

Recall the federated learning problem $\min_{w}\sum_{k=1}^{N}p_{k}F_{k}(w)$
with $F_{k}(w)=\frac{1}{n_{k}}\sum_{j=1}^{n_{k}}\ell(w;\xi_{k,j})$.
In this section, we consider the standard ERM setting where each $\ell(w;\xi_{k,j})$
is non-negative, $l$-smooth, and convex, and as before, we assume
that each $F_{k}(w)=\frac{1}{n_{k}}\sum_{j=1}^{n_{k}}\ell(w;\xi_{k,j})$
is $L$-smooth and $\mu$-strongly convex. Note that $l\geq L$. This
setting includes many important problems in practice, such as linear
regression with a full rank sample covariance matrix. In the overparameterized
setting, we assume that there exists $w^{\ast}\in\arg\min_{w}\sum_{k=1}^{N}p_{k}F_{k}(w)$
such that $\ell(w^{\ast};\xi_{k,j})=0$ for all $\xi_{k,j}$. Given
these assumptions, we first show that FedAvg with local SGD updates
and communication every $E$ iterations achieves exponential convergence
with linear speedup in the number of workers.
\begin{theorem}
	For the overparameterized setting with general $\mu$-strongly convex
	and $L$-smooth objectives, FedAvg with local SGD updates and communication
	every $E$ iterations with constant step size $\overline{\alpha}=\frac{1}{4E}\frac{N}{l\nu_{\max}+L(N-\nu_{\min})}$
	gives the exponential convergence guarantee 
	\begin{align*}
	\mathbb{E}F(\overline{w}_{t}) & \leq\frac{L}{2}(1-\mu\overline{\alpha})^{t}\|w_{0}-w^{\ast}\|^{2}=O(L\exp(-\frac{\mu}{4E}\frac{N}{l\nu_{\max}+L(N-\nu_{\min})}t)\cdot\|w_{0}-w^{\ast}\|^{2})
	\end{align*}
\end{theorem}
%
\begin{remark}
	When $l>L$, the speedup factor is on the order of $\frac{N}{E(l\nu_{\max}+L(N-\nu_{\min}))}/\frac{1}{l\nu_{\max}+L(1-\nu_{\min})}\approx\frac{Nl}{E(l+L(N-1))}=O(N/E)$
	for $N\leq\frac{l}{L}+1$, i.e. FedAvg with $N$ workers and communication
	every $E$ iterations provides an exponential convergence speedup
	factor of $O(N/E)$, for $N\leq\frac{l}{L}+1$. When $N$ is above
	this threhold, however, the speedup is almost constant in the number
	of woerkers. Our bound also illustrates the tradeoff between increasing
	the number of workers and the communication latency. When the cost
	of communication is high, i.e. it is costly to aggregate local parameters
	and broadcast the average to all active devices, increasing the number
	of workers by some factor achieves the same order of speedup as decreasing
	the communication lag by the same factor. 
\end{remark}
In the next part, we specialize to the problem of linear regression,
and demonstrate how the above bound can be sharpened to explicitly
depend on the condition number of the system. This result also paves
the way for the analysis of a new FedAvg with local momentum-based
updates in the following part. 

\subsection{Overparameterized Quadratic Problems}

We now consider the federated learning problem with overparameterized
quadratic objectives. In other words, the local device objectives
are given by the sum of squares. 
\begin{align*}
F_{k}(w) & =\frac{1}{2n_{k}}\sum_{j=1}^{n_{k}}(w^{T}x_{k,j}-z_{k,j})^{2}
\end{align*}
and there exists $w^{\ast}$ such that $F(w^{\ast})=\sum_{k}p_{k}F_{k}(w^{\ast})\equiv0$.
For such quadratic problems, we will present a sharper bound for FedAvg
with explicit dependence on the condition number, and demonstrate
how the convergence rate can be further improved using a momentum-based
accelerated FedAvg algorithm based on the MaSS algorithm of {[}CITE
Belkin{]}. 

In this section, we first gather some necessary definitions and observations
useful in the analysis of exponential convergence of FedAvg algorithms.
Define the local Hessian matrix as $H^{k}:=\frac{1}{n_{k}}\sum_{j=1}^{n_{k}}x_{k,j}(x_{k,j})^{T}$
and the averaged Hessian matrix as $H:=\sum_{k=1}^{N}p_{k}H^{k}$.
In general $H$ has zero eigevalues. However, because the null space
of $H$ and range of $H$ are orthogonal, in our subsequence analysis
it suffices to project $\overline{w}_{t}-w^{\ast}$ onto the range
of $H$, thus we may restrict to the non-zero eigenvalue of $H$. 

We can use $w^{\ast T}x_{k,j}-z_{k,j}\equiv0$ to rewrite the local
objectives 
\begin{align*}
F_{k}(w) & =\frac{1}{2n_{k}}\sum_{j=1}^{n_{k}}(w^{T}x_{k,j}-z_{k,j}-(w^{\ast T}x_{k,j}-z_{k,j}))^{2}=\frac{1}{2n_{k}}\sum_{j=1}^{n_{k}}((w-w^{\ast})^{T}x_{k,j})^{2}\\
& =\frac{1}{2}\langle w-w^{\ast},H^{k}(w-w^{\ast})\rangle=\frac{1}{2}\|w-w^{\ast}\|_{H^{k}}^{2}
\end{align*}
so that $F(w)=\frac{1}{2}\|w-w^{\ast}\|_{H}^{2}$. 

Let $x_{t}^{k}$ be the stochastic sample on the $k$th device during
iteration $t$, and define $\tilde{H}_{t}^{k}:=x_{t}^{k}(x_{t}^{k})^{T}$
as the stochastic Hessian matrix corresponding to the sample $x_{t}^{k}$.
Note that $\mathbb{E}\tilde{H}_{t}^{k}=\frac{1}{n_{k}}\sum_{j=1}^{n_{k}}x_{k,j}(x_{k,j})^{T}=H^{k}$
so that $\tilde{H}_{t}^{k}$ is an unbiased estimate of $H^{k}$.
Moreover, we have
\begin{align*}
g_{t,k} & =\nabla F_{k}(w_{t}^{k},x_{t}^{k})=\tilde{H}_{t}^{k}(w_{t}^{k}-w^{\ast})\\
g_{t} & =\sum_{k=1}^{N}p_{k}\nabla F_{k}(w_{t}^{k},x_{t}^{k})=\sum_{k=1}^{N}p_{k}\tilde{H}_{t}^{k}(w_{t}^{k}-w^{\ast})
\end{align*}

Define $l$ to be the smallest positive number such that $\mathbb{E}\|x_{t}^{k}\|^{2}$$x_{t}^{k}$($x_{t}^{k})^{T}\preceq lH^{k}$
for all $k,t$. Note that $l\leq\max_{k=1,\dots,N}\max_{1\leq j\leq n_{k}}\|x_{k,j}\|^{2}$,
i.e. $l$ is upper bounded by the maximum norm squared among all feature
vectors on all devices. 

Let $L^{k}$ and $\mu^{k}$ to be the largest and smallest non-zero
eigenvalues of $H^{k}$, and $L:=\max_{k}L^{k}$, $\mu:=\min_{k}\mu^{k}$.
Define $\kappa_{1}:=l/\mu$ and $\kappa:=L/\mu$. 

Following {[}CITE: Liu\&Belkin and Jain et al{]}, we define the statistical
condition number $\tilde{\kappa}^{k}$ as the smallest positive real
number such that 
\begin{align*}
\mathbb{E}\left[\langle x_{t}^{k}(H^{k})^{-1},x_{t}^{k}\rangle x_{t}^{k}(x_{t}^{k})^{T}\right] & \preceq\tilde{\kappa}^{k}H^{k}
\end{align*}
and define $\tilde{\kappa}:=\max_{k}\tilde{\kappa}^{k}$. The condition
numbers $\kappa_{1}$ and $\tilde{\kappa}$ are important in the characterization
of convergence rates for FedAvg algorithms. Note that $\kappa_{1}>\kappa$ and $\kappa_{1}>\tilde{\kappa}$. 

\subsection{Acceleration of FedAvg with Momentum-based Updates}

In the single-agent stochastic optimization setting, Nesterov updates
are known to fail to accelerate over SGD updates, so that in the federated
learning setting, we cannot hope for acceleration results of FedAvg
with Nesterov updates in general. The natural question is then whether
there are accelerated versions of the FedAvg algorithm that have provable
improvements over the FedAvg algorithm based on local SGD updates.
We answer this question in the affirmative in the overparameterized
quadratic problem setting, extending recent results {[}CITE: Sidford,
Belkin{]} in the single agent setting. In particular, we show that
specialized to overparameterized quadratic objectives, the exponential
convergence bound can be further sharpened, with a similar linear
speedup in the number of workers. This rate will then be further improved
using a new accelerated FedAvg algorithm with momentum-based updates. 
\begin{theorem}
	For the overparamterized quadratic problem, FedAvg with local SGD
	updates and communication every $E$ iterations with constant step
	size $\alpha_{t}=\frac{1}{4E}\frac{N}{l\nu_{\max}+\mu(N-\nu_{\min})}$
	gives the exponential convergence guarantee 
	\begin{align*}
	\mathbb{E}F(\overline{w}_{t}) & \leq L(1-\frac{N}{4E(\nu_{\max}\kappa_{1}+(N-\nu_{\min}))})^{t}\|w_{0}-w^{\ast}\|^{2}
	\end{align*}
\end{theorem}
%

We see that when $N=O(\kappa_{1})$, the convergence rate is $O((1-\frac{N}{E\kappa_{1}})^{t})=O(\exp(-\frac{N}{E\kappa_{1}}t))$,
which exhibits linear speedup in the number of workers, as well as
a $1/\kappa_{1}$ dependence on the condition number $\kappa_{1}$.
In the last part of this section, we will demonstrate that a new FedAvg
algorithm based on a local momentum update {[}CITE Belkin{]} improves
upon this dependence to $1/\sqrt{\kappa_{1}\tilde{\kappa}}$. 

Following {[}CITE{]}, we propose the accelerated FedAvg algorithm
with local MaSS updates: 
\begin{align*}
w_{t+1}^{k} & =\begin{cases}
u_{t}^{k}-\eta_{1}^{k}g_{t,k} & \text{if }t+1\notin\mathcal{I}_{E}\\
\sum_{k=1}^{N}p_{k}\left[u_{t}^{k}-\eta_{1}^{k}g_{t,k}\right] & \text{if }t+1\in\mathcal{I}_{E}
\end{cases}\\
u_{t+1}^{k} & =w_{t+1}^{k}+\gamma^{k}(w_{t+1}^{k}-w_{t}^{k})+\eta_{2}^{k}g_{t,k}
\end{align*}
where we note that the natural parameter is $w_{t}$, while $u_{t}$
is an auxiliary parameter, which we initialize to be $u_{0}^{k}$,
$g_{t,k}=\nabla F_{k}(u_{t}^{k},x_{t}^{k})$ is the stochastic gradient
and $g_{t}=\sum_{k=1}^{N}p_{k}g_{t,k}$ is the averaged stochastic
gradient. When $\eta_{2}^{k}\equiv0$, this reduces to the accelerated
FedAvg algorithm with Nesterov updates.

Also note that the update can equivalently be written as 
\begin{align*}
v_{t+1}^{k} & =(1-\alpha^{k})v_{t}^{k}+\alpha^{k}u_{t}^{k}-\delta^{k}g_{t,k}\\
w_{t+1}^{k} & =\begin{cases}
u_{t}^{k}-\eta^{k}g_{t,k} & \text{if }t+1\notin\mathcal{I}_{E}\\
\sum_{k=1}^{N}p_{k}\left[u_{t}^{k}-\eta^{k}g_{t,k}\right] & \text{if }t+1\in\mathcal{I}_{E}
\end{cases}\\
u_{t+1}^{k} & =\frac{\alpha^{k}}{1+\alpha^{k}}v_{t+1}^{k}+\frac{1}{1+\alpha^{k}}w_{t+1}^{k}
\end{align*}
where there is a bijection between the parameters $\frac{1-\alpha^{k}}{1+\alpha^{k}}=\gamma^{k},\eta^{k}=\eta_{1}^{k},\frac{\eta^{k}-\alpha^{k}\delta^{k}}{1+\alpha^{k}}=\eta_{2}^{k}$,
and we further introduce an auxiliary parameter $v_{t}^{k}$, which
is initialized at $v_{0}^{k}$. We also note that when $\delta^{k}=\frac{\eta^{k}}{\alpha^{k}}$,
the update reduces to the Nesterov accelerated SGD. This version of
the FedAvg algorithm with local MaSS updates is used for analyzing
the exponential convergence. 

As before, define the virtual sequences $\overline{w}_{t}=\sum_{k=1}^{N}p_{k}w_{t}^{k}$,
$\overline{v}_{t}=\sum_{k=1}^{N}p_{k}v_{t}^{k}$, $\overline{u}_{t}=\sum_{k=1}^{N}p_{k}u_{t}^{k}$,
and $\overline{g}_{t}=\sum_{k=1}^{N}p_{k}\mathbb{E}g_{t,k}$. We have
$\mathbb{E}g_{t}=\overline{g}_{t}$ and $\overline{w}_{t+1}=\overline{u}_{t}-\eta_{t}g_{t}$,
$\overline{v}_{t+1}=(1-\alpha^{k})\overline{v}_{t}+\alpha^{k}\overline{w}_{t}-\delta^{k}g_{t}$,
and $\overline{u}_{t+1}=\frac{\alpha^{k}}{1+\alpha^{k}}\overline{v}_{t+1}+\frac{1}{1+\alpha^{k}}\overline{w}_{t+1}$. 

We now present the exponential convergence result in overparameterized
quadratic problems using FedAvg with local MaSS updates. 
\begin{theorem}
For the overparamterized quadratic problem, FedAvg with local MaSS
updates and communication every $E$ iterations with constant step
sizes 
\begin{align*}
\eta_{t}=\frac{1}{4E}\frac{N}{l\nu_{\max}+\mu(N-\nu_{\min})}, & \alpha_{t}=\frac{1}{\sqrt{\kappa_{1}\tilde{\kappa}}},\delta_{t}=\frac{\eta_{t}}{\alpha_{t}\tilde{\kappa}}
\end{align*}
gives the exponential convergence guarantee 
\begin{align*}
\mathbb{E}F(\overline{w}_{t}) & \leq C(1-\frac{1}{4E}\frac{N}{l\nu_{\max}\sqrt{\kappa_{1}\tilde{\kappa}}+(N-\nu_{\min})})^{t}\|w_{0}-w^{\ast}\|^{2}
\end{align*}
\end{theorem}
Compared to the convergence rate of FedAvg with local SGD updates,
we see that when $N=O(\sqrt{\kappa_{1}\tilde{\kappa}})$, the convergence
rate is $O((1-\frac{N}{E\sqrt{\kappa_{1}\tilde{\kappa}}})^{t})=O(\exp(-\frac{N}{E\sqrt{\kappa_{1}\tilde{\kappa}}}t))$
as opposed to $O(\exp(-\frac{N}{E\kappa_{1}}t))$. Since $\kappa_{1}\geq\tilde{\kappa}$,
this implies a speedup factor of $\sqrt{\frac{\kappa_{1}}{\tilde{\kappa}}}$
for the FedAvg with local MaSS updates. On the other hand, the same
linear speedup in the number of workers holds for $N$ in a smaller
range of values. 