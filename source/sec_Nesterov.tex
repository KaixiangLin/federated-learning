% !TEX ROOT=./main.tex



\section{Linear Speedup Analysis of Nesterov Accelerated FedAvg}

A natural extension of the FedAvg algorithm is to use momentum-based
local updates instead of local SGD updates. To our knowledge, the
only formal analyses of accelerated FedAvg with momentum-based stochastic
updates {[}CITE kdd papaer{]} focus on the non-convex smooth case.
In this section, we complete the picture by providing convergence
results of Nesterov-accelerated FedAvg for convex objectives. We show
that for strongly convex and smooth objectives, the convergence of
the optimality gap of averaged parameters across devices is $O(1/NT)$
where $N$ is the number of active devices during each communication
round, while for convex and smooth objectives, the rate is $O(1/\sqrt{NT})$.
Thus Nesterov accelerated FedAvg algorithms enjoy the same convergence
rates as the original FedAvg algorithm. As we know from stochastic
optimization that Nesterov updates may fail to accelerate over SGD,
in general we cannot hope to obtain acceleration results in the federated
learning setting. However, in the following section we specialize
to the overparameterized setting where we demonstrate that a particular
accelerated FedAvg algorithm with momentum-based updates is able to
improve on the original FedAvg algorithm. 

\subsection{Strongly Convex and Smooth Objectives}

We first show that the Nesterov accelerated FedAvg has $O(1/NT)$
convergence rate for $\mu$-strongly convex and $L$-smooth objectives.
The Nesterov Accelerated FedAvg algorithm follows the updates
\begin{align*}
y_{t+1}^{k} & =w_{t}^{k}-\alpha_{t}g_{t,k}\\
w_{t+1}^{k} & =\begin{cases}
y_{t+1}^{k}+\beta_{t}(y_{t+1}^{k}-y_{t}^{k}) & \text{if }t+1\notin\mathcal{I}_{E}\\
\sum_{k=1}^{N}p_{k}\left[y_{t+1}^{k}+\beta_{t}(y_{t+1}^{k}-y_{t}^{k})\right] & \text{if }t+1\in\mathcal{I}_{E}
\end{cases}
\end{align*}
where $g_{t,k}:=\nabla F_{k}(w_{t}^{k},\xi_{t}^{k})$ is the stochastic
gradient. 
\begin{theorem}
	Assume that $F_{k}$ is $L$-smooth and $\mu$-strongly convex for
	all $k$, and let $\nu_{\max}=\max_{k}\frac{1}{N}p_{k}$. Let $\kappa=\frac{L}{\mu}$,
	$\gamma=\max\{32\kappa,E\}$ where $E$ is the communication delay,
	and diminishing learning rates $\alpha_{t}=\beta_{t}=\frac{1}{4\mu(\gamma+t)}$.
	Let $\overline{w}_{T}=\sum_{k=1}^{N}p_{k}w_{T}^{k}$ be the average
	of local parameters at an arbitrary time $T$. 
	
	With full device participation, 
	\begin{align*}
	\mathbb{E}F(\overline{y}_{T})-F^{\ast}=O(\frac{\kappa\nu_{\max}^{2}\sigma^{2}/\mu}{NT}+\frac{\kappa^{2}E^{2}G^{2}/\mu}{T^{2}})
	\end{align*}
	and with partial device participation with $K$ sampled devices at
	each communication round, 
	\begin{align*}
	\mathbb{E}F(\overline{y}_{T})-F^{\ast}\leq O(\frac{\kappa\nu_{\max}^{2}\sigma^{2}/\mu}{NT}+\frac{\kappa E^{2}G^{2}/\mu}{KT}+\frac{\kappa^{2}E^{2}G^{2}/\mu}{T^{2}})
	\end{align*}
\end{theorem}
%
\begin{remark}
To our knowledge, this is the first convergence result for Nesterov accelerated FedAvg in the convex and smooth setting. The same discussion about linear speedup and choice of $E$ from the
	previous section applies to the Nesterov accelerated FedAvg algorithm. 
\end{remark}

\subsection{Convex Smooth Objectives}

In this section, we provide linear speedup analyses of Nesterov Accelerated
FedAvg with convex and smooth objectives and show that the optimality
gap is $O(1/\sqrt{NT})$ where $N$ is the number of participating
devices. This result complements the strongly convex case in the previous
part, as well as the non-convex smooth setting in {[}CITE kdd and
others{]}, where a similar $O(1/\sqrt{NT})$ rate is given in terms
of averaged gradient norm. 
\begin{theorem}
	Suppose $F_{k}$ is $L$-smooth and convex for all $k$, and let $\nu_{\max}=\max_{k}\frac{1}{N}p_{k}$.
	Set learning rate $\alpha_{t}=O(\sqrt{\frac{N}{T}})$.
	
	With full device participation, 
	\begin{align*}
	\min_{t\leq T}F(\overline{w}_{t})-F(w^{\ast}) & =O(\sqrt{\frac{\nu_{\max}^{2}\sigma^{2}}{NT}+\frac{E^{2}LG^{2}}{T^{4/3}}})
	\end{align*}
	and with partial device participation with $K$ sampled devices at
	each communication round, 
	\begin{align*}
	\min_{t\leq T}F(\overline{w}_{t})-F(w^{\ast}) & =O(\sqrt{\frac{\nu_{\max}^{2}\sigma^{2}}{NT}+\frac{E^{2}G^{2}}{KT}+\frac{E^{2}LG^{2}}{T^{4/3}}})
	\end{align*}
\end{theorem}
%
\begin{remark}
	It is possible to extend the analysis to accelerated FedAvg algorithms
	with other momentum-based updates. However, in line with the stochastic
	optimization setting, none of these methods will be able to achieve
	a better rate than the original FedAvg algorithm in general. For this
	reason, we will instead turn to the overparameterized setting {[}CITE
	Belkin{]} where we show that it is possible to improve the convergence
	rate of FedAvg with momentum based local updates. 
\end{remark}