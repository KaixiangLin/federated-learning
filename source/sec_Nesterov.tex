% !TEX ROOT=./main.tex



\section{Linear Speedup Analysis of Nesterov Accelerated FedAvg}
\label{sec:Nesterov}

A natural extension of the FedAvg algorithm is to use momentum-based
local updates instead of local SGD updates. To our knowledge, the
only convergence analyses of FedAvg with momentum-based stochastic
updates focus on the non-convex smooth case \cite{huo2020faster,yu2019linear,li2018federated}.
In this section, we complete the picture with $\mathcal{O}(1/NT)$
and $\mathcal{O}(1/\sqrt{NT})$ convergence results for Nesterov-accelerated
FedAvg for convex objectives that match the original FedAvg results.
As we know from stochastic optimization, Nesterov and other momentum
updates may fail to accelerate over SGD \cite{liu2018accelerating,kidambi2018insufficiency,liu2018toward,yuan2016influence}.
Therefore in Section~\ref{sec:overparameterized} we will specialize to overparameterized
problems where we demonstrate that a particular FedAvg variant with
momentum updates is able to accelerate over the original FedAvg algorithm. 
The convergence proof of Nesterov-accelerated FedAvg is deferred in Appendix Section~\ref{sec:app:Nesterovfedavg}.

\subsection{Strongly Convex and Smooth Objectives}

We first show that the Nesterov accelerated FedAvg has $\mathcal{O}(1/NT)$
convergence rate for $\mu$-strongly convex and $L$-smooth objectives.
The Nesterov Accelerated FedAvg algorithm follows the updates
\begin{align*}
\mathbf{v}_{t+1}^{k} & =\mathbf{w}_{t}^{k}-\alpha_{t}\mathbf{g}_{t,k},\\
\mathbf{w}_{t+1}^{k} & =\begin{cases}
\mathbf{v}_{t+1}^{k}+\beta_{t}(\mathbf{v}_{t+1}^{k}-\mathbf{v}_{t}^{k}) & \text{if }t+1\notin\mathcal{I}_{E},\\
\sum_{k=1}^{N}p_{k}\left[\mathbf{v}_{t+1}^{k}+\beta_{t}(\mathbf{v}_{t+1}^{k}-\mathbf{v}_{t}^{k})\right] & \text{if }t+1\in\mathcal{I}_{E}.
\end{cases}
\end{align*}
where $g_{t,k}:=\nabla F_{k}(\mathbf{w}_{t}^{k},\xi_{t}^{k})$ is
the stochastic gradient sampled on the $k$-th device at time $t$.  
\begin{theorem}
	\label{thm:nesterov_scvx}Let $\overline{\mathbf{v}}_{T}=\sum_{k=1}^{N}p_{k}\mathbf{v}_{T}^{k}$
	be the average of local parameters of Nesterov accelerated FedAvg
	with learning rates $\alpha_{t}=\frac{6}{\mu}\frac{1}{t+\gamma}$
	and $\beta_{t-1}=\frac{3}{14(t+\gamma)(1-\frac{6}{t+\gamma})\max\{\mu,1\}}$,
	$\gamma=\max\{32\kappa,E\}$ and $\kappa=\frac{L}{\mu}$. Then with
	full device participation, 
	\begin{align*}
	\mathbb{E}F(\overline{\mathbf{v}}_{T})-F^{\ast}=\mathcal{O}\left(\frac{\kappa\nu_{\max}^{2}\sigma^{2}/\mu}{NT}+\frac{\kappa^{2}E^{2}G^{2}/\mu}{T^{2}}\right),
	\end{align*}
	and with partial device participation with $K$ sampled devices at
	each communication round, 
	\begin{align*}
	\mathbb{E}F(\overline{\mathbf{v}}_{T})-F^{\ast}\leq\mathcal{O}\left(\frac{\kappa\nu_{\max}^{2}\sigma^{2}/\mu}{NT}+\frac{\kappa E^{2}G^{2}/\mu}{KT}+\frac{\kappa^{2}E^{2}G^{2}/\mu}{T^{2}}\right).
	\end{align*}
\end{theorem}
%
To our knowledge, this is the first convergence result for Nesterov
accelerated FedAvg in the strongly convex and smooth setting. The
same discussion about linear speedup of FedAvg applies to the Nesterov
accelerated variant. In particular, to achieve $\mathcal{O}(1/NT)$
linear speedup, $T$ iterations of the algorithm requires only $\mathcal{O}(\sqrt{NT})$
communication rounds with full participation. 


\subsection{Convex Smooth Objectives}

We now provide linear speedup analysis of Nesterov Accelerated FedAvg
with convex and smooth objectives and show that the optimality gap
is $\mathcal{O}(1/\sqrt{NT})$ where $N$ is the number of participating
devices. This result complements the strongly convex case in the previous
part, as well as the non-convex smooth setting in \cite{huo2020faster,yu2019linear,li2018federated},
where a similar $\mathcal{O}(1/\sqrt{NT})$ rate is given in terms
of averaged gradient norm. 
\begin{theorem}
	\label{thm:Nesterov_cvx}Set learning rates $\alpha_{t}=\beta_{t}=\mathcal{O}(\sqrt{\frac{N}{T}})$
	for Accelerated FedAvg with convex and $L$-smooth objectives. With
	full device participation, 
	\begin{align*}
	\min_{t\leq T}F(\overline{\mathbf{v}}_{t})-F^{\ast} & =\mathcal{O}\left(\frac{\nu_{\max}^{2}\sigma^{2}}{\sqrt{NT}}+\frac{NE^{2}LG^{2}}{T}\right),
	\end{align*}
	and with partial device participation with $K$ sampled devices at
	each communication round, 
	\begin{align*}
	\min_{t\leq T}F(\overline{\mathbf{v}}_{t})-F^{\ast} & =\mathcal{O}\left(\frac{\nu_{\max}^{2}\sigma^{2}}{\sqrt{KT}}+\frac{E^{2}G^{2}}{\sqrt{KT}}+\frac{KE^{2}LG^{2}}{T}\right).
	\end{align*}
\end{theorem}
%
	It is possible to extend the analyses in this section to accelerated
    FedAvg algorithms with other momentum-based updates. However, in the
    stochastic optimization setting, none of these methods will be able
    to achieve a better rate than the original FedAvg with SGD updates in general.
    For this reason, we will instead turn to the overparameterized setting
    \cite{ma2017power,liu2018accelerating,canziani2016analysis} in the
    next section where we show that FedAvg enjoys geometric convergence
    and it is possible to improve its convergence rate with momentum-based updates.
