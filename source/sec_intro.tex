% !TEX ROOT=./main.tex


\section{Introduction}
% Motivate FL from the security and privacy perspective. FedAvg
Federated learning (FL) is a distributed learning paradigm that can collaboratively solve a machine learning problem by leveraging the data
from a large amount of workers without compromising the users' (aka workers/local devices) privacy~\cite{kairouz2019advances}. 
Among all the leading algorithms in FL, Federated Averaging~\cite{mcmahan2016communication} (FedAvg)
is perhaps the most widely adopted optimization algorithm, which runs local
Stochastic Gradient Descent (SGD) updates on a subset of devices
and synchronize the local models once in a while. 
The empirical success of FedAvg and its variants have
attracted lots of efforts~\cite{li2018federated,stich2018local,khaled2019first,yu2019parallel,haddadpour2019convergence,li2019convergence,huo2020faster} on analyzing its convergence properties. 
There are few key challenges that differentiates the theoretical analysis
of FL from the tradition distribution optimization: 
1) The data is non-identically distributed across the workers, which means the
data in each local device cannot be regarded as samples drawn from
a same distribution. 
2) The workers are not active at every communication
round. In FL, the central server has no control over the local devices. 
It is more practical to assume only a subset of workers is active during
each communication round. 

% The current gaps in FL.
However, most of prior works~\cite{li2018federated,stich2018local,khaled2019first,yu2019parallel,haddadpour2019convergence} either assume
the data is identically distributed or the all devices are active, which
violates the practical characteristic in FL. The most recent work~\cite{li2019convergence} firstly presents $O(1/T)$ convergence guarantee without 
making those unrealistic assumptions while their analysis focused 
on strongly convex case only and the relation between the number of active
workers and the convergence rate is not clearly discussed. 

% Our contributions
\subsection{Contributions}
Our work aims to systematically fill the gap in the convergence analysis
of FedAvg and its accelerated variants. 
Our theoretical results not only cover all general convex objectives, including strongly convex cases, convex smooth cases and convex non-smooth cases, but also provide tighter convergence guarantee for all convex smooth objectives, which shows the convergence rate enjoys a linear speedup w.r.t.
the number of workers.
Furthermore, we studied a popular over-parameterized setting~\cite{liu2018accelerating} where the optimal solution can obtain zero training loss. In this scenario, we provide a novel accelerated FL algorithm improving
the convergence rate of FedAvg. Last but not least, we conduct extensive
evaluation on both synthetic and real-world dataset, which demonstrates that our theoretical indications is well-aligned with the empirical observations.


