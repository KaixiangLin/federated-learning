\documentclass{article}

% if you need to pass options to natbib, use, e.g.:
%     \PassOptionsToPackage{numbers, compress}{natbib}
% before loading neurips_2020

% ready for submission
% \usepackage{neurips_2020}

% to compile a preprint version, e.g., for submission to arXiv, add add the
% [preprint] option:
%     \usepackage[preprint]{neurips_2020}

% to compile a camera-ready version, add the [final] option, e.g.:
%     \usepackage[final]{neurips_2020}

% to avoid loading the natbib package, add option nonatbib:
     \usepackage[nonatbib]{neurips_2020}


\usepackage[utf8]{inputenc} % allow utf-8 input
\usepackage[T1]{fontenc}    % use 8-bit T1 fonts
\usepackage{hyperref}       % hyperlinks
\usepackage{url}            % simple URL typesetting
\usepackage{booktabs}       % professional-quality tables
\usepackage{amsfonts}       % blackboard math symbols
\usepackage{nicefrac}       % compact symbols for 1/2, etc.
\usepackage{microtype}      % microtypography
\usepackage{amsthm}     % for theorems
\usepackage{algorithm,algorithmic}
\usepackage{mathtools}
\usepackage{amsmath}
\allowdisplaybreaks
\usepackage{amssymb}
\DeclarePairedDelimiter{\norm}{\lVert}{\rVert}
\usepackage{balance}
\usepackage{comment}
\newcommand{\eq}[1]{{Eq~(#1)}}
\newcommand{\figref}[1]{Figure~\ref{#1}}
\newtheorem{theorem}{Theorem}
\newtheorem{thm}{Theorem}
\newtheorem{lemma}[theorem]{Lemma}
\newtheorem{remark}{Remark}
\newcommand{\lkxcom}[1]{{\color{red}{#1}}}
\usepackage{pifont}% http://ctan.org/pkg/pifont
\newcommand{\cmark}{\ding{51}}%
\newcommand{\xmark}{\ding{55}}%

\newcommand{\ov}[1]{{\overline{\mathbf{#1}}}}
%% common packages
\usepackage{amsbsy}
\usepackage{amsmath}
\usepackage{graphicx}
\usepackage{subfigure}
\usepackage{color}
\usepackage{booktabs}

%% to allow citation as footnote
\usepackage{natbib}
% and to reduce footnote font size
\usepackage{etoolbox}
\makeatletter
\patchcmd{\@makefntext}{\insertfootnotetext{#1}}{\insertfootnotetext{\scriptsize#1}}{}{}
\makeatother


%% macros for commenting
\usepackage[normalem]{ulem} % to use \sout
\newcommand{\remove}[1]{{\color{Gray}\sout{#1}}}
\newcommand{\revise}[1]{{\color{blue}#1}}
\newcommand{\commwy}[1]{{\color{red}(wy: #1)}} % Wotao Yin

%% template for beamer

%\mode<presentation>
%{
%  % page number
%  %\setbeamertemplate{footline}{\insertframenumber/\inserttotalframenumber}
%  \setbeamertemplate{footline}[frame number]
%
%  % background and theme
%  \setbeamertemplate{background canvas}[vertical shading][bottom=white!10,top=white!10]
%  \usetheme{default}
%
%  % section in table of contents has numbers
%  \setbeamertemplate{sections/subsections in toc}[sections numbered]
%
%  % no navigation bottoms
%  \setbeamertemplate{navigation symbols}{}
%
%  % itemize, black bullet, %150 spacing between items using "witemize"
%  \setbeamertemplate{itemize items}[circle]
%  \setbeamercolor{itemize item}{fg=black}
%  \setbeamercolor{enumerate item}{fg=black}
%  \setbeamercolor{itemize subitem}{fg=black}
%  \setbeamercolor{enumerate subitem}{fg=black}
%  \newenvironment{witemize}{\itemize\addtolength{\itemsep}{0.0\baselineskip}}{\enditemize}
%
%  % block, black over gray with no shadow
%  \setbeamertemplate{blocks}[rounded][shadow=false]
%  \setbeamercolor{block title}{fg=black,bg=gray!40}
%  \setbeamercolor{block body}{fg=black,bg=gray!10}
%
%  % frametitle, bold, black, centered
%  \setbeamertemplate{frametitle}[default][center]
%  \setbeamercolor{frametitle}{fg=black}
%  \setbeamerfont{frametitle}{shape=\bfseries}
%
%  % line spacing
%  \linespread{1.2}
%
%  \setlength{\parskip}{\smallskipamount}
%}

%% macros for letters

\newcommand{\va}{{\mathbf{a}}}
\newcommand{\vb}{{\mathbf{b}}}
\newcommand{\vc}{{\mathbf{c}}}
\newcommand{\vd}{{\mathbf{d}}}
\newcommand{\ve}{{\mathbf{e}}}
\newcommand{\vf}{{\mathbf{f}}}
\newcommand{\vg}{{\mathbf{g}}}
\newcommand{\vh}{{\mathbf{h}}}
\newcommand{\vi}{{\mathbf{i}}}
\newcommand{\vj}{{\mathbf{j}}}
\newcommand{\vk}{{\mathbf{k}}}
\newcommand{\vl}{{\mathbf{l}}}
\newcommand{\vm}{{\mathbf{m}}}
\newcommand{\vn}{{\mathbf{n}}}
\newcommand{\vo}{{\mathbf{o}}}
\newcommand{\vp}{{\mathbf{p}}}
\newcommand{\vq}{{\mathbf{q}}}
\newcommand{\vr}{{\mathbf{r}}}
\newcommand{\vs}{{\mathbf{s}}}
\newcommand{\vt}{{\mathbf{t}}}
\newcommand{\vu}{{\mathbf{u}}}
\newcommand{\vv}{{\mathbf{v}}}
\newcommand{\vw}{{\mathbf{w}}}
\newcommand{\vx}{{\mathbf{x}}}
\newcommand{\vy}{{\mathbf{y}}}
\newcommand{\vz}{{\mathbf{z}}}

\newcommand{\vA}{{\mathbf{A}}}
\newcommand{\vB}{{\mathbf{B}}}
\newcommand{\vC}{{\mathbf{C}}}
\newcommand{\vD}{{\mathbf{D}}}
\newcommand{\vE}{{\mathbf{E}}}
\newcommand{\vF}{{\mathbf{F}}}
\newcommand{\vG}{{\mathbf{G}}}
\newcommand{\vH}{{\mathbf{H}}}
\newcommand{\vI}{{\mathbf{I}}}
\newcommand{\vJ}{{\mathbf{J}}}
\newcommand{\vK}{{\mathbf{K}}}
\newcommand{\vL}{{\mathbf{L}}}
\newcommand{\vM}{{\mathbf{M}}}
\newcommand{\vN}{{\mathbf{N}}}
\newcommand{\vO}{{\mathbf{O}}}
\newcommand{\vP}{{\mathbf{P}}}
\newcommand{\vQ}{{\mathbf{Q}}}
\newcommand{\vR}{{\mathbf{R}}}
\newcommand{\vS}{{\mathbf{S}}}
\newcommand{\vT}{{\mathbf{T}}}
\newcommand{\vU}{{\mathbf{U}}}
\newcommand{\vV}{{\mathbf{V}}}
\newcommand{\vW}{{\mathbf{W}}}
\newcommand{\vX}{{\mathbf{X}}}
\newcommand{\vY}{{\mathbf{Y}}}
\newcommand{\vZ}{{\mathbf{Z}}}

\newcommand{\cA}{{\mathcal{A}}}
\newcommand{\cB}{{\mathcal{B}}}
\newcommand{\cC}{{\mathcal{C}}}
\newcommand{\cD}{{\mathcal{D}}}
\newcommand{\cE}{{\mathcal{E}}}
\newcommand{\cF}{{\mathcal{F}}}
\newcommand{\cG}{{\mathcal{G}}}
\newcommand{\cH}{{\mathcal{H}}}
\newcommand{\cI}{{\mathcal{I}}}
\newcommand{\cJ}{{\mathcal{J}}}
\newcommand{\cK}{{\mathcal{K}}}
\newcommand{\cL}{{\mathcal{L}}}
\newcommand{\cM}{{\mathcal{M}}}
\newcommand{\cN}{{\mathcal{N}}}
\newcommand{\cO}{{\mathcal{O}}}
\newcommand{\cP}{{\mathcal{P}}}
\newcommand{\cQ}{{\mathcal{Q}}}
\newcommand{\cR}{{\mathcal{R}}}
\newcommand{\cS}{{\mathcal{S}}}
\newcommand{\cT}{{\mathcal{T}}}
\newcommand{\cU}{{\mathcal{U}}}
\newcommand{\cV}{{\mathcal{V}}}
\newcommand{\cW}{{\mathcal{W}}}
\newcommand{\cX}{{\mathcal{X}}}
\newcommand{\cY}{{\mathcal{Y}}}
\newcommand{\cZ}{{\mathcal{Z}}}

\newcommand{\ri}{{\mathrm{i}}}
\newcommand{\rr}{{\mathrm{r}}}


%% macros for math notions and operators

\newcommand{\RR}{\mathbb{R}}
\newcommand{\EE}{\mathbb{E}}
\newcommand{\CC}{\mathbb{C}}
\newcommand{\ZZ}{\mathbb{Z}}
\renewcommand{\SS}{{\mathbb{S}}}
\newcommand{\SSp}{\mathbb{S}_{+}}
\newcommand{\SSpp}{\mathbb{S}_{++}}
\newcommand{\sign}{\mathrm{sign}}
\newcommand{\Sign}{\mathrm{Sign}}
\newcommand{\vzero}{\mathbf{0}}
\newcommand{\vone}{{\mathbf{1}}}
\newcommand{\Null}{{\mathrm{Null}}}
\newcommand{\dist}{{\mathrm{dist}}}
\newcommand{\Co}{{\mathbf{Co}}}

\newcommand{\op}{{\mathrm{op}}} % subscript for operator norm
\newcommand{\opt}{{\mathrm{opt}}} % subscript for optimal solution
\newcommand{\supp}{{\mathrm{supp}}} % support
\newcommand{\Prob}{{\mathrm{Prob}}} % probability
\newcommand{\Diag}{{\mathrm{Diag}}} % vector -> diagonal matrix
\newcommand{\diag}{{\mathrm{diag}}} % matrix diagonal -> vector
\newcommand{\dom}{{\mathrm{dom}}} % domain
\newcommand{\grad}{{\nabla}}    % gradient
\newcommand{\tr}{{\mathrm{tr}}} % trace
\newcommand{\TV}{{\mathrm{TV}}} % total variation
\newcommand{\Proj}{{\mathrm{Proj}}}
\DeclareMathOperator{\shrink}{shrink} % shrinkage
\DeclareMathOperator*{\argmin}{arg\,min}
\DeclareMathOperator*{\argmax}{arg\,max}
\DeclareMathOperator*{\mini}{minimize}
\DeclareMathOperator*{\maxi}{maximize}
\DeclareMathOperator*{\Min}{minimize}
\DeclareMathOperator*{\Max}{maximize}
\newcommand{\prox}{{\mathbf{prox}}}
\newcommand{\st}{{\quad\text{s.t.}~}}

%% macros for environments math equations

\newcommand{\MyFigure}[1]{../fig/#1}

\newcommand{\bc}{\begin{center}}
\newcommand{\ec}{\end{center}}

\newcommand{\bdm}{\begin{displaymath}}
\newcommand{\edm}{\end{displaymath}}

\newcommand{\beq}{\begin{equation}}
\newcommand{\eeq}{\end{equation}}

\newcommand{\bfl}{\begin{flushleft}}
\newcommand{\efl}{\end{flushleft}}

\newcommand{\bt}{\begin{tabbing}}
\newcommand{\et}{\end{tabbing}}

\newcommand{\beqn}{\begin{eqnarray}}
\newcommand{\eeqn}{\end{eqnarray}}

\newcommand{\beqs}{\begin{align*}} % no equation numbers
\newcommand{\eeqs}{\end{align*}}  % no equation numbers

%% macros for theorem-like environments

% \newtheorem{theorem}{Theorem}
% \newtheorem{condition}{Condition}
\newtheorem{assumption}{Assumption}
\newtheorem{definition}{Definition}
% \newtheorem{corollary}{Corollary}
% \newtheorem{remark}{Remark}
% \newtheorem{lemma}{Lemma}
% \newtheorem{proof}{Proof}
% \newtheorem{proof*}{Proof}
% \newtheorem{proposition}{Proposition}
%\newtheorem{example}{Example}

% \newtheorem{example}[remark]{Example}



\newtheorem*{assumption*}{Assumption}

\title{Federated Learning's Blessing:\\
FedAvg has Linear Speedup}

\author{}

\begin{document}

\maketitle

\begin{abstract}
Federated learning learns
a model jointly from a set of distributed workers without sharing their private
data. The characteristics of non-\emph{i.i.d.} data
across the network and low device participation bring significant challenges
in understanding the convergence of federated learning algorithms. Prior work
analyzed the convergence with restrained settings,
%either \emph{i.i.d.} data, or with full device participation, 
and often with suboptimal rates. 
In this work, we provide a
comprehensive convergence analysis on Federated Averaging (FedAvg),
a widely used framework in which SGD or Nesterov SGD updates are used for general convex problems with arbitrarily
heterogeneous data. 
We first show that the convergence rates using both updates enjoy a linear speedup w.r.t. the number of participating workers for all convex smooth problems. 
In addition, in the overparamterized setting, we show that the convergence of FedAvg is in fact exponential, also with a speedup factor linear in the number of workers. 
Furthermore, we propose a new accelerated federated learning algorithm that further improves this convergence rate with provable guarantees in the linear regression setting. 
Empirical studies of the algorithms in various settings have supported our theoretical results.
\end{abstract}

% !TEX ROOT=./main.tex


\section{Introduction}
Federated learning (FL) is a machine learning paradigm where many clients (e.g., mobile devices or organizations) collaboratively train a model under the orchestration of a central server (e.g., service provider), while keeping the training data decentralized (\cite{smith2017federated, kairouz2019advances}). In recent years, FL has swiftly emerged as an important learning paradigm (\cite{mcmahan2016communication,li2018federated})--one that enjoys widespread success in applications such as personalized recommendation (\cite{chen2018federated}), virtual assistant (\cite{lam2019protecting}), and keyboard prediction (\cite{47586}), to name a few--for at least three reasons: First, the rapid proliferation of smart devices that are equipped with both computing power and data-capturing capabilities provided the infrastructure core for FL. Second, the rising awareness of privacy and the explosive growth of computational power in mobile devices have made it increasingly attractive to push the computation to the edge. Third, the empirical success of communication-efficient FL algorithms has enabled increasingly larger-scale parallel computing and learning with less communication overhead.

Despite its promise and broad applicability in our current era, the potential value FL delivers is coupled with the unique challenges it brings forth. In particular, when FL learns a single statistical model using data from across all the devices while keeping each individual device's data isolated (\cite{kairouz2019advances}), it faces two challenges that are absent in centralized optimization and distributed (stochastic) optimization (\cite{zhou2017convergence, stich2018local,khaled2019first,liang2019variance,wang2018cooperative,woodworth2018graph,wang2019adaptive,jiang2018linear,yu2019parallel,yu2019linear, khaled2020tighter,koloskova2020unified,woodworth2020local,woodworth2020minibatch}):

1) \textbf{Data (statistical) heterogeneity:} data distributions in devices are different (and data cannot be shared);

2) \textbf{System heterogeneity:} only a subset of devices may access the central server at each time both because the communications bandwidth profiles vary across devices and because there is no central server that has control over when a device is active (the presence of ``stragglers''). 
% Gaps in the theoretical understanding. 

To address these challenges, Federated Averaging (FedAvg)~\cite{mcmahan2016communication} was proposed as a particularly effective heuristic, which has enjoyed great empirical success. This success has since motivated a growing line of research efforts \cite{haddadpour2019convergence,li2019convergence,karimireddy2019scaffold,huo2020faster} into understanding its theoretical convergence guarantees in various settings. 
% For instance, \cite{haddadpour2019convergence} analyzed FedAvg (for non-convex smooth problems satisfying PL conditions) under the assumption that each local device's minimizer is the same as the minimizer of the joint problem (if all devices' data is aggregated together), an overly restrictive assumption that restricts the extent of data heterogeneity.
% Very recently, \cite{li2019convergence} furthered the progress and established an $\cO(\frac{1}{T})$ convergence rate for FedAvg for strongly convex smooth Federated learning problems with both data and system heterogeneity. A similar result in the same setting~\cite{karimireddy2019scaffold} also established an $\cO(\frac{1}{T})$ result that allows for a linear speedup when the number of participating devices is large.
% At the same time, \cite{huo2020faster} studied the Nesterov accelerated FedAvg for non-convex smooth problems and established 
% an $\cO(\frac{1}{\sqrt{T}})$ convergence rate to stationary points. 
% (Prior to that, \cite{liu2019accelerating} studied Nesterov accelerated FedAvg using full gradient--rather than stochastic gradient as in both the above-mentioned works and our setting--for strongly convex smooth problems and established an $\cO(\frac{1}{T})$ convergence rate).
However, despite these very recent fruitful pioneering efforts into understanding the theoretical convergence properties of FedAvg, it remains open as to how the number of devices--particularly the number of devices that participate in the computation--affects the convergence speed.
In particular, is linear speedup of FedAvg a universal phenomenon across different settings and for any number of devices? What about when FedAvg is accelerated with momentum updates? Does the presence of both data and system heterogeneity in FL imply different communication complexities and require technical novelties over results in distributed and decentralized optimization? These aspects are currently unexplored or underexplored in FL. We fill in the gaps here by providing affirmative answers.

\begin{table}[t!]
\hspace{-1em}
{\small
\centering
\begin{tabular}{|c|c|c|c|c|}\hline 
	\multirow{2}{*}{\backslashbox{{\tiny Participation} }{{\tiny Objective function}}} & \multirow{2}{*}{Strongly Convex}     &\multirow{2}{*}{Convex}    & Overparameterized & Overparameterized \\ 
	                                &                        &         &     general case                 & linear regression   \\ \hline 
	Full                         & $\cO(\frac{1}{NT}+\frac{E^{2}}{T^{2}})$    &  $\mathcal{O}\left(\frac{1}{\sqrt{NT}}+\frac{NE^{2}}{T}\right)$   & $\cO(\exp(-\frac{NT}{E\kappa_1}))$ & $\cO(\exp(-\frac{NT}{E\kappa }))^{\dagger}$    \\ \hline
	Partial                      &  $\cO\left(\frac{E^{2}}{KT}+\frac{E^{2}}{T^{2}}\right)$   &  $\cO\left(\frac{E^2}{\sqrt{KT}}+\frac{KE^2}{T} \right)$ &  $\cO(\exp(-\frac{KT}{E\kappa_1}))$ & $\cO(\exp(-\frac{KT}{E\kappa }))^{\dagger}$     \\ \hline
\end{tabular}
}
\caption{\small Our convergence results for FedAvg and accelerated FedAvg in this paper. Throughout the paper, $N$ is the total
number of local devices, and $K \leq N$ is the maximal number of devices that are accessible to the central server. $T$ is the total number of stochastic updates performed by each local device, $E$ is the local steps between two consecutive server communications (and hence $T/E$ is the number of communications). $^{\dagger}$ In the linear regression setting, we have $\kappa=\kappa_1$ for FedAvg and $\kappa=\sqrt{\kappa_1\tilde{\kappa}}$ for momentum accelerated FedAvg (FedMaSS), where $\kappa_1$ and $\sqrt{\kappa_1\tilde{\kappa}}$ are condition numbers defined in Section \ref{sec:app:overparameterized}. Since
$\kappa_{1}\geq\tilde{\kappa}$, this implies a speedup factor of
$\sqrt{\frac{\kappa_{1}}{\tilde{\kappa}}}$ for accelerated FedAvg.}
% {\raggedright 
% $^{*}$ The fourth column presents the results for the general overparaterized setting. The fifth column presents the results for overparaterized linear regression.
        %   \par}
\label{tb:convergencerateintro}
\vspace{-2.5em}
\end{table}


\textbf{Our Contributions}
We provide a comprehensive and unified convergence analysis
of FedAvg and its accelerated variants considering both data and system heterogeneity. 
Our contributions are threefold.
First, we establish an {\small{$\cO(1/KT)$}} convergence rate  under FedAvg for strongly convex and smooth problems and  an
{\small{$\cO(1/\sqrt{KT})$}} convergence rate for convex
and smooth problems (where $K$ is the number of participating devices), thereby establishing that FedAvg enjoys the desirable linear speedup convergence, which improves the best known results\cite{li2019convergence,karimireddy2019scaffold} .
\begin{comment}
Prior to our work here, the best and the most related convergence analysis is given by \cite{li2019convergence} and~\cite{karimireddy2019scaffold}, which established an $\cO(\frac{1}{T})$ convergence rate for strongly convex smooth problems under FedAvg. Our rate matches the same (and optimal) dependence on $T$, but also completes the picture by establishing the linear dependence on $K$, for any $K\leq N$, where $N$ is the total number of devices, whereas~\cite{li2019convergence} does not have linear speedup analysis, and~\cite{karimireddy2019scaffold} only allows linear speedup close to full participation ($K=\mathcal{O}(N)$). As for convex and smooth problems, there was no prior work that established the {\small{$\cO(\frac{1}{\sqrt{T}})$}} rate under both system and data heterogeneity. Our unified analysis highlights the common elements and distinctions between the strongly and convex settings.
\end{comment}
Second, we establish the same convergence rates--{\small{$\cO(1/KT)$}} for strongly convex and smooth problems and {\small{$\cO(1/\sqrt{KT})$}} for convex and smooth problems--for Nesterov accelerated FedAvg. 
\begin{comment}
We analyze the accelerated version of FedAvg here because empirically it tends to perform better; yet, its theoretical convergence guarantee is unknown. To the best of our knowledge, these are the first results that provide a linear speedup characterization of Nesterov accelerated FedAvg in those two problem classes (that FedAvg and Nesterov accelerated FedAvg share the same convergence rate is to be expected: this is the case even for centralized stochastic optimization). Prior to our results here, the most relevant results~\cite{yu2019linear,li2018federated,huo2020faster} only concern the non-convex setting, where convergence is measured with respect to stationary points (vanishing of gradient norms, rather than optimality gaps). Our unified analysis of Nesterov FedAvg also illustrates the technical similarities and distinctions compared to the original FedAvg algorithm, whereas prior works (in the non-convex setting) were scattered and used different notations.
\end{comment}
Third, we study a subclass of strongly convex smooth problems where the objective is over-parameterized and establish 
a faster $\cO(\exp(-\frac{KT}{\kappa}))$ convergence rate for FedAvg, in contrast to the $\cO(\exp(-\frac{T}{\kappa}))$ rate for individual solvers~\cite{ma2017power}. 
\begin{comment}
Within this class, we further consider the linear regression problem and establish an even sharper rate under FedAvg. In addition, we propose a new variant of accelerated FedAvg based on a momentum update of~\cite{liu2018accelerating}--MaSS accelerated FedAvg--and establish a faster convergence rate (compared to if no acceleration is used). This stands in contrast to generic (strongly) convex stochastic problems where theoretically no rate improvement is obtained when one accelerates FedAvg.
The detailed convergence results are summarized in Table~\ref{tb:convergencerateintro}.
\end{comment}

\begin{comment}
	

\textbf{Connections with Distributed and Decentralized Optimization}
Federated learning is closely related to distributed and decentralized optimization, and as such it is important to discuss connections and distinctions between our work and related results from that literature. First, when there is neither system heterogeneity, i.e. all devices participate in parameter averaging during a communication round, nor statistical heterogeneity, i.e. all devices have access to a common set of stochastic gradients, FedAvg coincides with the ``Local SGD'' of~\cite{stich2018local}, which showed the linear speedup rate $\mathcal{O}(1/NT)$ for strongly convex and smooth functions. \cite{woodworth2020local} and \cite{woodworth2020minibatch} further improved the communication complexity that guarantees the linear speedup rate. When there is only data heterogeneity, some works have continued to use the term Local SGD to refer to FedAvg, while others subsume it in more general frameworks that include decentralized model averaging based on a network topology or a mixing matrix. They have provided linear speedup analyses for strongly convex and convex problems, e.g.~\cite{khaled2020tighter,koloskova2020unified} as well as non-convex problems, e.g.~\cite{jiang2018linear,yu2019parallel,wang2018cooperative}. However, these results do not consider system heterogeneity, i.e. the presence of stragglers in the device network. Even with decentralized model averaging, the assumptions usually imply that model averages over all devices is the same as decentralized model averages based on network topology (e.g.~\cite{koloskova2020unified} Proposition 1), which precludes system heterogeneity as defined in this paper and prevalent in FL problems. For momentum accelerated FedAvg,~\cite{yu2019linear} provided linear speedup analysis for non-convex problems, while results for strongly convex and convex settings are entirely lacking, even without system heterogeneity.~\cite{karimireddy2019scaffold} considers both types of heterogeneities for FedAvg, but their rate implies a linear speedup only when the number of stragglers is negligible. In contrast, our linear speedup analyses consider both types of heterogeneity present in the full federated learning setting, and are valid for any number of participating devices. We also highlight a distinction in communication efficiency when system heterogeneity is present. Moreover, our results for Nesterov accelerated FedAvg completes the picture for strongly convex and convex problems. For a detailed comparison with related works, please refer to Table~\ref{tb:convergenceratev3} in Appendix Section~\ref{sec:app:comparison}.

\end{comment}

% The detailed convergence results are summarized in Table~\ref{tb:convergencerateintro}, which also include aspects we did not have space to dicuss (i.e. how the rate scales with communications frequency etc.). Note that the full participation case corresponds to $K=N$. 




% Our theoretical results not only cover all general convex objectives, including strongly convex cases, convex smooth cases and convex non-smooth cases, but also provide tighter convergence guarantee for all convex smooth objectives, which shows the convergence rate enjoys a linear speedup w.r.t.
% the number of workers.
% Furthermore, we studied a popular over-parameterized setting~\cite{liu2018accelerating} where the optimal solution can obtain zero training loss. In this scenario, we provide a novel accelerated FL algorithm improving
% the convergence rate of FedAvg. Last but not least, we conduct extensive
% evaluation on both synthetic and real-world dataset, which demonstrates that our theoretical indications is well-aligned with the empirical observations.




% \textbf{Organization} 
% The reminader of this paper is organized as follows. In Section 




% Intro and related work.
% sec 4. linear speedup of FedAvg,  sec 5. linear speedup of Accelerated FedAvg.
% sec 6 linear regression in 
% 1. this is the first result on the exponential convergence of FedAvg algorithms in the
% interpolation setting with linear speedup in the number of workers and explicit dependence on the
% communication interval E.  
% 2. Improved convergence rate over FedAvg for For the overparamterized quadratic problem. 

% without considering statistical heterogeneity
% or system heterogeneity~\cite{stich2018local,khaled2019first,wang2018cooperative,yu2019parallel,yu2019linear} or suboptimal 
% convergence rate~\cite{li2019convergence}.

% is perhaps the most widely adopted optimization algorithm, which runs local
% Stochastic Gradient Descent (SGD) updates on a subset of devices
% and synchronize the local models once in a while. 
% The empirical success of FedAvg and its variants have
% attracted lots of efforts~\cite{li2018federated,stich2018local,khaled2019first,yu2019parallel,haddadpour2019convergence,li2019convergence,huo2020faster} on analyzing its convergence properties. 
% There are few key challenges that differentiates the theoretical analysis
% of FL from the tradition distribution optimization: 
% 1) The data is non-identically distributed across the workers, which means the
% data in each local device cannot be regarded as samples drawn from
% a same distribution. 
% 2) The workers are not active at every communication
% round. In FL, the central server has no control over the local devices. 
% It is more practical to assume only a subset of workers is active during
% each communication round. 

% The current gaps in FL.
% However, most of prior works~\cite{li2018federated,stich2018local,khaled2019first,yu2019parallel,haddadpour2019convergence} either assume
% the data is identically distributed or the all devices are active, which
% violates the practical characteristic in FL. The most recent work~\cite{li2019convergence} firstly presents $O(1/T)$ convergence guarantee without 
% making those unrealistic assumptions while their analysis focused 
% on strongly convex case only and the relation between the number of active
% workers and the convergence rate is not clearly discussed. 


%% !TEX ROOT=./main.tex

\section{Related Works}

% related work on convergence rate
Federated learning (FL) was originally proposed
in~\cite{mcmahan2016communication} for learning a single, global statistical
model from the isolated data stored in a massive number of devices.  The
empirical success of FL~\cite{chen2018federated,47586} has attracted much attention to the rigorous theoretical understanding of the leading algorithm in the field: Federated Averaging (FedAvg) and its accelerated variants~\cite{liu2019accelerating,haddadpour2019convergence,khaled2019first,li2019convergence,huo2020faster,yu2019linear,yu2019parallel}.

% Non-convex linear speedup
In particular, we focus on the convergence analysis of FedAvg, which
has been discussed in various settings. 
In~\cite{yu2019parallel,wang2019adaptive}, the authors provide the convergence rate of FedAvg for the non-convex problem, given all the devices are active at each communication round. Recently, \citep{huo2020faster} provide 
convergence guarantee for Nesterov accelerated FedAvg for non-convex 
setting in a partial participation setting. Although the empirical
results are improved over FedAvg, the accelerated version admitting a similar convergence rate as FedAvg. In this work, we provide a novel accelerated
FedAvg with improved convergence rates under a popular overparameterized setting. In addition, \cite{yu2019linear} provides $O(1/\sqrt{NT})$ 
convergence (i.e., linear speedup) of FedAvg for the non-convex problem in the full participation setting. 
For convex smooth problems, \cite{khaled2019first} prove the 
convergence rate of local GD (a full batch version of FedAvg with full participation) on heterogeneous data. 
Concurrently, \cite{li2019convergence} firstly provides exponential convergence of FedAvg for the strongly convex problem,
under the more practical setting of partial participation and Non-IID data while their results did not achieve linear speedup. 
\cite{stich2018local} prove linear speedup convergence of FedAvg for the strongly convex problem, while they assume full participation and identically distributed data. Comparing to previous works, our results provide the linear speedup on strongly convex problems in the practical partial participation setting and heterogeneous data. In \cite{liu2019accelerating}, the authors provide a convergence rate of accelerated FedAvg for strongly convex problems, while it is considered in full participation setting. To the best of our knowledge,
we provide the first linear speedup convergence results for accelerated FedAvg in the practical setting.










% !TEX ROOT=./main.tex


\section{Problem Setting}

\begin{align}
	\min _{\mathbf{w}}\left\{F(\mathbf{w}) \triangleq \sum_{k=1}^{N} p_{k} F_{k}(\mathbf{w})\right\}
	\label{eq:problem}
\end{align}
where $N$ is the number of devices, and $p_k$ is the weight of the k-th device
such that $p_k \geq 0$ and $\sum_{k=1}^N p_k = 1$. Suppose the k-th device
holds the nk training data: $x_{k,1}, x_{k,2}, \dots, x_{k,n_k}$ . The local
$k \in S_t^K$ objective $F_k(\cdot)$ is defined by

\begin{align}
F_{k}(\mathbf{w}) \triangleq \frac{1}{n_{k}} \sum_{j=1}^{n_{k}} \ell\left(\mathbf{w} ; x_{k, j}\right)	
\label{eq:localloss}
\end{align}

% The summary of all assumptions:

% \begin{assumption}
% $F_{1}, \cdots, F_{N}$ are all $L$-smooth: for all  $\mathbf{v}$  and $\mathbf{w}$, $F_{k}(\mathbf{v}) \leq F_{k}(\mathbf{w})+(\mathbf{v}- \\ \mathbf{w})^{T} \nabla F_{k}(\mathbf{w})+\frac{L}{2}\|\mathbf{v}-\mathbf{w}\|_{2}^{2}$.
% \end{assumption}

% \begin{assumption}
% $F_1,\dots, F_N$ are all convex: for all $\vv$ and $\vw$, 
% $F_k(\vv) \geq F_k(\vw)+(\vv -\vw)^T \grad F_k(\vw)$. \label{ass:cvx}
% \end{assumption}
% \begin{assumption}
% $	F_{1}, \cdots, F_{N} \text { are all } \mu \text { -strongly convex: for all v and } \mathbf{w}, F_{k}(\mathbf{v}) \geq F_{k}(\mathbf{w})+(\mathbf{v}- \\ \mathbf{w})^{T} \nabla F_{k}(\mathbf{w})+\frac{\mu}{2}\|\mathbf{v}-\mathbf{w}\|_{2}^{2}$
% \end{assumption}

% \begin{assumption}
% Let $\xi_{t}^{k}$ be sampled from the k-th device's local data uniformly at random. The variance of stochastic gradients in each device is bounded:
% $\mathbb{E}\left\|\nabla F_{k}\left(\mathbf{w}_{t}^{k}, \xi_{t}^{k}\right)-\nabla F_{k}\left(\mathbf{w}_{t}^{k}\right)\right\|^{2} \leq \sigma_{k}^{2}$, for $k = 1,..., N$.
% \end{assumption}

% \begin{assumption}
% The expected squared norm of stochastic gradients is uniformly bounded. i.e.,
% $\mathbb{E}\left\|\nabla F_{k}\left(\mathbf{w}_{t}^{k}, \xi_{t}^{k}\right)\right\|^{2} \leq G^{2}$, for all $k = 1,..., N$ and $t=0, \dots, T-1$.
% \end{assumption}

\section{Notations}


Let $\vw_t^k$ be the model parameter maintained in the k-th device at the t-th step. $\cI_E$ is a set of global synchronization steps, i.e., $\cI_E = \{n E|n = 1, 2, 3,\dots \}$. If $t+1 \in \cI_E$, which means we communicate the server with (all) clients at time step $t+1$. $\vv^k_{t+1}$ is the immediate result of one step SGD update from $\vw^k_{t}$.
$\xi_{t}^{k}$ denotes the data sampled from k-th device’s local data uniformly at random.
Follow the common practice, we define two virtual sequences $\overline{\mathbf{v}}_{t}$ and $\overline{\mathbf{w}}_{t}$. For full device participation and $t \notin \cI_E$,
$\ov{v}_t = \ov{w}_t =\sum_{k=1}^{N} p_{k} \mathbf{v}_{t}^{k}$. In partial participation, $t \in \cI_E$, $\ov{w}_t \neq \ov{v}_t$ since $\ov{v}_t=\sum_{k=1}^{N} p_{k} \mathbf{v}_{t}^{k}$ while $\sum_{k\in \cS_t}\mathbf{w}_{t}^{k}$. However, we can
set unbiased sampling strategy such that $ \EE_{\cS_t} \ov{w}_t = \ov{v}_t$.
$\overline{\mathbf{v}}_{t+1}$ is one-step SGD from $\overline{\mathbf{w}}_{t}$. 
\begin{align}
\overline{\mathbf{v}}_{t+1}=\overline{\mathbf{w}}_{t}-\eta_{t} \mathbf{g}_{t}	\label{eq:vbar}
\end{align}
% $t+1 \in \cI_E$, we can fetch $\overline{\vw}_{t+1}$, we can communicate $\overline{\vw}_{t+1}$ to all devices.
where $\vg_{t} = \sum_{k=1}^{N} p_{k} \vg_{t,k} $ is one-step stochastic gradient, averaged over all devices. 
\begin{align}
\vg_{t,k} \left\{\begin{array}{ll} 
 = \nabla F_{k}\left(\mathbf{w}_{t}^{k},\xi_{t}^{k} \right)  &  \text{smooth}\\
 \in \partial F_{k}\left(w_{t}^{k}, \xi_{t}^{k}\right)  & \text{non-smooth}
 \end{array}\right.
\end{align}
Similarly, we denote the expected one-step gradient $\ov{g}_{t}= \EE_{\xi_t}[\vg_t] = \sum_{k=1}^{N} p_{k} \EE_{\xi_{t}^{k}} \vg_{t,k}$, where
\begin{align}
\EE_{\xi_{t}^{k}} \vg_{t,k}  \left\{\begin{array}{ll} 
 = \nabla F_{k}\left(\mathbf{w}_{t}^{k}\right)  &  \text{smooth}\\
 \in \partial F_{k}\left(w_{t}^{k}\right)  & \text{non-smooth}
 \end{array}\right.
\end{align}
and we use $\xi_t = \{\xi_t^k\}_{k=1}^N$ denotes samples at all devices at time step $t$. 

The updates of FedAve with partial device activation is given by: 
\begin{align} 
\mathbf{v}_{t+1}^{k} &=\mathbf{w}_{t}^{k}-\eta_{t} \nabla F_{k}\left(\mathbf{w}_{t}^{k}, \xi_{t}^{k}\right) \\ \mathbf{w}_{t+1}^{k} &=\left\{\begin{array}{ll}\mathbf{v}_{t+1}^{k} & \text { if } t+1 \notin \mathcal{I}_{E}, \\ 
\sum_{k \in \cS_{t+1}} \mathbf{v}_{t+1}^{k} & \text { if } t+1 \in \mathcal{I}_{E}\end{array}\right.
\end{align}

In~\cite{li2019convergence}, two types of unbiased sampling strategies are considered in Lemma 5. 
The sampling scheme I establishes $\cS_{t+1}$ by i.i.d. sampling the devices with replacement,
in this case the upper bound of expected square norm of $\ov{w}_{t+1} - \ov{v}_{t+1}$ is given by:
\begin{align}
\EE_{\cS_{t+1}}\left\|\ov{w}_{t+1} - \ov{v}_{t+1}\right\|^2	\leq \frac{4}{K} \eta_t^2 E^2G^2
\end{align}
The sampling scheme II establishes $\cS_{t+1}$ by uniformly sampling all devices without
replacement, in which we have the 
\begin{align}
\EE_{\cS_{t+1}}\left\|\ov{w}_{t+1} - \ov{v}_{t+1}\right\|^2	\leq \frac{4(N - K)}{K(N-1)} \eta_t^2 E^2G^2
\end{align}
We denote this upper bound as follows for concise presentation. 
\begin{align}
	\EE_{\cS_{t+1}}\left\|\ov{w}_{t+1} - \ov{v}_{t+1}\right\|^2 \leq  \eta_t^2 C
	\label{eq:partialsample}
\end{align}





% !TEX ROOT=./main.tex



\section{Linear Speedup Analysis of FedAvg}

In this section, we provide convergence analysis of FedAvg with local
SGD updates. We show that for strongly convex and smooth objectives,
the convergence of the optimality gap of averaged parameters across
devices is $O(1/NT)$ where $N$ is the number of active devices during
each communication round, while for convex and smooth objectives,
the rate is $O(1/\sqrt{NT})$. 

\subsection{Strongly Convex and Smooth Objectives}

We first show that FedAvg with local SGD updates has $O(1/NT)$ convergence
rate for $\mu$-strongly convex and $L$-smooth objectives. The result
improves on the $O(1/T)$ result of {[}ICLR{]} with a linear speedup
in the number of devices $N$. Moreover, it implies a distinction
in the choice of $E$ that guarantees this linear speedup for FedAvg
with full and partial device participation. With full participation,
$E$ can be chosen as large as $O(\sqrt{\frac{T}{N}})$ without degrading
the linear speedup in the number of workers. On the other hand, with
partial participation, $E$ must be $O(1)$. 
\begin{theorem}
	Suppose $F_{k}$ is $L$-smooth and $\mu$-strongly convex for all
	$k$, and let $\nu_{\max}=\max_{k}\frac{1}{N}p_{k}$. Let $\kappa=\frac{L}{\mu}$,
	$\gamma=\max\{32\kappa,E\}$ where $E$ is the communication delay,
	and diminishing learning rates $\alpha_{t}=\frac{1}{4\mu(\gamma+t)}$.
	Let $\overline{w}_{T}=\sum_{k=1}^{N}p_{k}w_{T}^{k}$ be the average
	of local parameters at an arbitrary time $T$.
	
	With full device participation, 
	\begin{align*}
	\mathbb{E}F(\overline{w}_{T})-F^{\ast}=O\left(\frac{\kappa\nu_{\max}^{2}\sigma^{2}/\mu}{NT}+\frac{\kappa^{2}E^{2}G^{2}/\mu}{T^{2}}\right)
	\end{align*}
	and with partial device participation with $K$ sampled devices at
	each communication round, 
	\begin{align*}
	\mathbb{E}F(\overline{w}_{T})-F^{\ast}\leq O\left(\frac{\kappa\nu_{\max}^{2}\sigma^{2}/\mu}{NT}+\frac{\kappa E^{2}G^{2}/\mu}{KT}+\frac{\kappa^{2}E^{2}G^{2}/\mu}{T^{2}}\right)
	\end{align*}
\end{theorem}
%
\begin{remark}
	\textbf{Linear speedup. }We first compare our bound with that in {[}ICLR{]},
	which is $O(\frac{1}{NT}\nu_{\max}^{2}\sigma^{2}\kappa/\mu+\frac{\kappa E^{2}G^{2}/\mu}{KT}+\frac{E^{2}G^{2}}{T}\kappa/\mu)$.
	Because the third term $\frac{E^{2}G^{2}}{T}\kappa/\mu$ is also $O(1/T)$
	without a dependence on $N$, for any choice of $E$ their bound cannot
	achieve linear speedup. The improvement of our bound comes from the
	term $\frac{\kappa^{2}E^{2}G^{2}/\mu}{T^{2}}$, which now is $O(1/T^{2})$.
	As a result, all leading terms scale with $1/N$ in the full device
	particiaption setting, and with $1/K$ in the partial participation
	setting. This implies that in both settings, there is a linear speedup
	in the number of active workers during the communication round.
\end{remark}
%
\begin{remark}
	\textbf{Choice of $E$.} Our bound implies a distinction in the choice
	of $E$ between the full and partial participation settings. In full
	participation, as long as $E=O(\sqrt{\frac{T}{N}})$, the term $\frac{\kappa^{2}E^{2}G^{2}/\mu}{T^{2}}=O(\frac{1}{NT})$,
	which is on the same order as the leading term $\frac{\kappa\nu_{\max}^{2}\sigma^{2}/\mu}{NT}$.
	Thus to achieve a linear speedup in the number of workers $N$, it
	suffices to communicate every $O(\sqrt{\frac{T}{N}})$ iterations.
	In contrast, the bound in {[}ICLR{]} does not allow $E$ to scale
	with $\sqrt{T}$, even for full participation. Thus our bound yields
	a more efficient communication complexity, while at the same time
	providing a linear speedup in the convergence. On the other hand,
	with partial participation, the term $\frac{\kappa E^{2}G^{2}/\mu}{KT}$
	is also a leading term, and so $E$ must be $O(1)$. In this setting,
	our bound still yields a linear speedup in $K$, which is not the
	case for previous analyses. 
\end{remark}

\subsection{Convex Smooth Objectives}

In this section, we provide linear speedup analyses of FedAvg with
convex and smooth objectives and show that the optimality gap is $O(1/\sqrt{NT})$
where $N$ is the number of participating devices. This result complements
the strongly convex case in the previous part, as well as the non-convex
smooth setting in {[}CITE Yu and others{]}, where a similar $O(1/\sqrt{NT})$
rate is given in terms of averaged gradient norm. 
\begin{theorem}
	Suppose $F_{k}$ is $L$-smooth and convex for all $k$, and let $\nu_{\max}=\max_{k}\frac{1}{N}p_{k}$.
	Set learning rate $\alpha_{t}=O\left(\sqrt{\frac{N}{T}}\right)$. With full device participation, 
	\begin{align*}
	\min_{t\leq T}F(\overline{w}_{t})-F(w^{\ast}) & =O\left(\sqrt{\frac{\nu_{\max}^{2}\sigma^{2}}{NT}+\frac{E^{2}LG^{2}}{T^{4/3}}}\right)
	\end{align*}
	and with partial device participation with $K$ sampled devices at
	each communication round, 
	\begin{align*}
	\min_{t\leq T}F(\overline{w}_{t})-F(w^{\ast}) & =O\left(\sqrt{\frac{\nu_{\max}^{2}\sigma^{2}}{NT}+\frac{E^{2}G^{2}}{KT}+\frac{E^{2}LG^{2}}{T^{4/3}}}\right)
	\end{align*}
\end{theorem}

\begin{remark}
	\textbf{Choice of $E$ and linear speedup. }Similar in the strongly
	convex case, we see that in the full participation setting, as long
	as $E=O(\sqrt{\frac{T^{1/3}}{N}})$, the convergence benefits from
	a linear speedup in the number of active workers. In the partial participation
	setting, $E$ must be $O(1)$ in order to allow for linear speedup.
\end{remark}
%
\begin{remark}
	\textbf{Learning rate. }The learning rate now depends on the final
	horizon $T$ of the convergence statement, whereas before the learning
	rate was set to $O(1/t)$ for the $t$-th iteration. We also note
	that the requirement on $\alpha_{t}=O(\sqrt{\frac{N}{T}})$ is also
	present in the work of {[}Yu{]} where they derive similar $O(1/\sqrt{NT})$
	linear speedup convergence results for nonconvex smooth federated
	learning problems.
\end{remark}
For non-smooth objectives, we also have convergence results with rates
$O(1/T)$ and $O(1/\sqrt{T})$ for strongly convex and convex cases,
which can be found in the appendix. Without smoothness, however, the
linear speedup is no longer guaranteed. Together these results complete
the picture of convergence analyses for FedAvg with local SGD updates
for general convex heterogeneous objectives. We next investigate the
convergence of FedAvg with Nesterov updates, and provide similar linear
speedup analyses for convex smooth objectives. 
% !TEX ROOT=./main.tex



\section{Linear Speedup Analysis of Nesterov Accelerated FedAvg}

A natural extension of the FedAvg algorithm is to use momentum-based
local updates instead of local SGD updates. To our knowledge, the
only formal analyses of accelerated FedAvg with momentum-based stochastic
updates {[}CITE kdd papaer{]} focus on the non-convex smooth case.
In this section, we complete the picture by providing convergence
results of Nesterov-accelerated FedAvg for convex objectives. We show
that for strongly convex and smooth objectives, the convergence of
the optimality gap of averaged parameters across devices is $O(1/NT)$
where $N$ is the number of active devices during each communication
round, while for convex and smooth objectives, the rate is $O(1/\sqrt{NT})$.
Thus Nesterov accelerated FedAvg algorithms enjoy the same convergence
rates as the original FedAvg algorithm. As we know from stochastic
optimization that Nesterov updates may fail to accelerate over SGD,
in general we cannot hope to obtain acceleration results in the federated
learning setting. However, in the following section we specialize
to the overparameterized setting where we demonstrate that a particular
accelerated FedAvg algorithm with momentum-based updates is able to
improve on the original FedAvg algorithm. 

\subsection{Strongly Convex and Smooth Objectives}

We first show that the Nesterov accelerated FedAvg has $O(1/NT)$
convergence rate for $\mu$-strongly convex and $L$-smooth objectives.
The Nesterov Accelerated FedAvg algorithm follows the updates
\begin{align*}
y_{t+1}^{k} & =w_{t}^{k}-\alpha_{t}g_{t,k}\\
w_{t+1}^{k} & =\begin{cases}
y_{t+1}^{k}+\beta_{t}(y_{t+1}^{k}-y_{t}^{k}) & \text{if }t+1\notin\mathcal{I}_{E}\\
\sum_{k=1}^{N}p_{k}\left[y_{t+1}^{k}+\beta_{t}(y_{t+1}^{k}-y_{t}^{k})\right] & \text{if }t+1\in\mathcal{I}_{E}
\end{cases}
\end{align*}
where $g_{t,k}:=\nabla F_{k}(w_{t}^{k},\xi_{t}^{k})$ is the stochastic
gradient. 
\begin{theorem}
	Assume that $F_{k}$ is $L$-smooth and $\mu$-strongly convex for
	all $k$, and let $\nu_{\max}=\max_{k}\frac{1}{N}p_{k}$. Let $\kappa=\frac{L}{\mu}$,
	$\gamma=\max\{32\kappa,E\}$ where $E$ is the communication delay,
	and diminishing learning rates $\alpha_{t}=\frac{9}{\mu}\frac{1}{t+\gamma}$
	and $\beta_{t-1}=\frac{9}{14\max\{\mu,1\}}\frac{1}{t+\gamma}$. Let
	$\overline{y}_{T}=\sum_{k=1}^{N}p_{k}y_{T}^{k}$ be the average of
	local parameters at an arbitrary time $T$. 
	
	With full device participation, 
	\begin{align*}
	\mathbb{E}F(\overline{y}_{T})-F^{\ast}=O(\frac{\kappa\nu_{\max}^{2}\sigma^{2}/\mu}{NT}+\frac{\kappa^{2}E^{2}G^{2}/\mu}{T^{2}})
	\end{align*}
	and with partial device participation with $K$ sampled devices at
	each communication round, 
	\begin{align*}
	\mathbb{E}F(\overline{y}_{T})-F^{\ast}\leq O(\frac{\kappa\nu_{\max}^{2}\sigma^{2}/\mu}{NT}+\frac{\kappa E^{2}G^{2}/\mu}{KT}+\frac{\kappa^{2}E^{2}G^{2}/\mu}{T^{2}})
	\end{align*}
\end{theorem}
%
\begin{remark}
To our knowledge, this is the first convergence result for Nesterov accelerated FedAvg in the convex and smooth setting. The same discussion about linear speedup and choice of $E$ from the
	previous section applies to the Nesterov accelerated FedAvg algorithm. 
\end{remark}

\subsection{Convex Smooth Objectives}

In this section, we provide linear speedup analyses of Nesterov Accelerated
FedAvg with convex and smooth objectives and show that the optimality
gap is $O(1/\sqrt{NT})$ where $N$ is the number of participating
devices. This result complements the strongly convex case in the previous
part, as well as the non-convex smooth setting in {[}CITE kdd and
others{]}, where a similar $O(1/\sqrt{NT})$ rate is given in terms
of averaged gradient norm. 
\begin{theorem}
	Suppose $F_{k}$ is $L$-smooth and convex for all $k$, and let $\nu_{\max}=\max_{k}\frac{1}{N}p_{k}$.
	Set learning rate $\alpha_{t},\beta_{t}=O(\sqrt{\frac{N}{T}})$.
	
	With full device participation, 
	\begin{align*}
	\min_{t\leq T}F(\overline{y}_{t})-F(w^{\ast}) & =O(\sqrt{\frac{\nu_{\max}^{2}\sigma^{2}}{NT}+\frac{E^{2}LG^{2}}{T^{4/3}}})
	\end{align*}
	and with partial device participation with $K$ sampled devices at
	each communication round, 
	\begin{align*}
	\min_{t\leq T}F(\overline{y}_{t})-F(w^{\ast}) & =O(\sqrt{\frac{\nu_{\max}^{2}\sigma^{2}}{NT}+\frac{E^{2}G^{2}}{KT}+\frac{E^{2}LG^{2}}{T^{4/3}}})
	\end{align*}
\end{theorem}
%
\begin{remark}
	It is possible to extend the analysis to accelerated FedAvg algorithms
	with other momentum-based updates. However, in line with the stochastic
	optimization setting, none of these methods will be able to achieve
	a better rate than the original FedAvg algorithm in general. For this
	reason, we will instead turn to the overparameterized setting {[}CITE
	Belkin{]} where we show that it is possible to improve the convergence
	rate of FedAvg with momentum based local updates. 
\end{remark}
% !TEX ROOT=./main.tex



\section{Exponential Convergence of FedAvg in the Overparameterized Setting}

So far we have shown $O(1/\sqrt{NT})$ and $O(1/NT)$ convergence
rates for FedAvg with local SGD and Nesterov udpates in the convex
and strongly convex settings, respectively. In this section, we turn
to the special setting of overparameterized problems, where a non-negative
loss function admits a zero-loss global minimizer. Such problems are
very common in machine learning systems where the number of parameters
far exceeds the number of data points, and perfect fitting is possible.
There have been a line of recent works {[}CITE{]} showing that SGD
and accelerated methods achieve exponential convergence in this case,
thanks to the property of ``automatic variance reduction'' property
{[}CITE{]}. The natural question is whether such a result also holds
in the federated learning setting. We first show that this is indeed
the case and establish the exponential (linear) convergence of FedAvg
with local SGD updates with constant step size for general strongly
convex and smooth overparameterized problems. In addition, we show
that the convergence rate speeds up linearly in the number of workers
$N$ when $N$ is below some problem-dependent threhold, while decreases
with $E$ through $1/E$. To our knowledge, this is the first result
on the exponential convergence of FedAvg algorithms in the interpolation
setting with linear speedup in the number of workers and explicit
dependence on the communication interval $E$. We next sharpen this
exponential convergence rate in the special case of linear regression,
and show that it is $O(\exp(-\frac{NT}{E\kappa_{1}}))$, for $\kappa_{1}$
an appropriately defined condition number of the system. Lastly, we
turn to the question of whether FedAvg with momentum-based local updates
can outperform FedAvg with SGD updates. In contrast to the gradient
descent setting, in the single-agent stochastic gradient setting,
Nesterov and Heavy Ball updates are known to fail to accelerate over
SGD, both in the overparameterized setting and standard convex setting{[}CITE{]}.
Thus in general we cannot hope to obtain acceleration results for
the FedAvg algorithm with Nesterov and Heavy Ball updates. On the
other hand, in the overparameterized setting, {[}CITE Belkin{]} introduced
the MaSS algorithm, which is a modification of the Nesterov update,
where in the update a non-negative multiple of the gradient is added
to the Nesterov parameter update to correct for the ``over-descent''
of the Nesterov update. For quadratic objectives, the authors show
that the MaSS algorithm is able to achieve acceleration over the exponential
convergence of SGD both theoretically and empirically. In the last
part of this section, we introduce a new FedAvg algorithm with momentum
updates by adapting the MaSS algorithm to the federated learning setting.
For this new FedAvg algorithm with local MaSS updates, we show that
it achieves exponential convergence for overparameterized quadratic
problems with rate $O(\exp(-\frac{NT}{E\sqrt{\kappa_{1}\tilde{\kappa}}}))$,
where $\tilde{\kappa}$ is the ``statistical condition number''
introduced in {[}CITE{]} that satisfies $\tilde{\kappa}\leq\kappa_{1}$.
Thus the FedAvg algorithm with local MaSS updates preserves the linear
speedup property in the number of workers and the $1/E$ dependence
on $E$, while achieving a speedup of factor $\sqrt{\frac{\kappa_{1}}{\tilde{\kappa}}}$
over FedAvg with local SGD updates. 

\subsection{Exponential Convergence of FedAvg with local SGD: Overparameterized
	Strongly Convex Smooth Objective}

In {[}CITE Belkin{]}, the authors show that in the overparameterized
setting, SGD achieves exponential convergence with linear speedup
in batch size that is below some critical threshold. In this section,
we extend their result to the federated learning setting and show
that a similar exponential convergence with linear speedup in the
number of workers using FedAvg with local SGD updates holds. 

Recall the federated learning problem 
\begin{align*}
\min_{w}\sum_{k=1}^{N}p_{k}F_{k}(w)\\
F_{k}(w)=\frac{1}{n_{k}}\sum_{j=1}^{n_{k}}\ell(w;\xi_{k,j})
\end{align*}
In this section, we consider the standard ERM setting where each
$\ell(w;\xi_{k,j})$ is non-negative, $l$-smooth, and convex, and
as before, we assume that each $F_{k}(w)=\frac{1}{n_{k}}\sum_{j=1}^{n_{k}}\ell(w;\xi_{k,j})$
is $L$-smooth and $\mu$-strongly convex. Note that $l\geq L$. This
setting includes many important problems in practice, such as linear
regression with a full rank sample covariance matrix. In the overparameterized
setting, we assume that there exists $w^{\ast}\in\arg\min_{w}\sum_{k=1}^{N}p_{k}F_{k}(w)$
such that $\ell(w^{\ast};\xi_{k,j})=0$ for all $\xi_{k,j}$. Given
these assumptions, we first show that FedAvg with local SGD updates
and communication every $E$ iterations achieves exponential convergence
with linear speedup in the number of workers.
\begin{theorem}
	For the overparameterized setting with general strongly convex and
	smooth objectives, FedAvg with local SGD updates and communication
	every $E$ iterations with constant step size $\overline{\alpha}=\frac{1}{4E}\frac{N}{l\nu_{\max}+L(N-\nu_{\min})}$
	gives the exponential convergence guarantee 
	\begin{align*}
	\mathbb{E}F(\overline{w}_{t}) & \leq\frac{L}{2}(1-\mu\overline{\alpha})^{t}\|w_{0}-w^{\ast}\|^{2}=O(L\exp(-\frac{\mu}{4E}\frac{N}{l\nu_{\max}+L(N-\nu_{\min})}t)\cdot\|w_{0}-w^{\ast}\|^{2})
	\end{align*}
\end{theorem}
%
\begin{remark}
	We see that when $l>L$, the speedup factor is on the order of $\frac{N}{E(l\nu_{\max}+L(N-\nu_{\min}))}/\frac{1}{l\nu_{\max}+L(1-\nu_{\min})}\approx\frac{Nl}{E(l+L(N-1))}=O(N/E)$
	for $N\leq\frac{l}{L}+1$, i.e. FedAvg with $N$ workers and communication
	every $E$ iterations provides an exponential convergence speedup
	factor of $O(N/E)$, for $N\leq\frac{l}{L}+1$. When $N$ is above
	this threhold, however, the speedup is almost constant in the number
	of woerkers. Our bound also illustrates the tradeoff between increasing
	the number of workers and the communication latency. When the cost
	of communication is high, i.e. it is costly to aggregate local parameters
	and broadcast the average to all active devices, increasing the number
	of workers by some factor achieves the same order of speedup as decreasing
	the communication lag by the same factor. 
\end{remark}
In the next part, we specialize to the problem of linear regression,
and demonstrate how the above bound can be sharpened to explicitly
depend on the condition number of the system. This result also paves
the way for the analysis of a new FedAvg with local momentum-based
updates in the following part. 

\subsection{Overparameterized Quadratic Problems}

We now consider the federated learning problem with overparameterized
quadratic objectives. In other words, the local device objectives
are given by the sum of squares. 
\begin{align*}
F_{k}(w) & =\frac{1}{2n_{k}}\sum_{j=1}^{n_{k}}(w^{T}x_{k,j}-z_{k,j})^{2}
\end{align*}
and there exists $w^{\ast}$ such that $F(w^{\ast})=\sum_{k}p_{k}F_{k}(w^{\ast})\equiv0$.
In this section, we gather some necessary definitions and observations
useful in the analysis of exponential convergence of FedAvg algorithms. 

Define the local Hessian matrix as 
\begin{align*}
H^{k} & :=\frac{1}{n_{k}}\sum_{j=1}^{n_{k}}x_{k,j}(x_{k,j})^{T}
\end{align*}
and the averaged Hessian matrix as 
\begin{align*}
H & :=\sum_{k=1}^{N}p_{k}H^{k}
\end{align*}

In general $H$ has zero eigevalues. However, because the null space
of $H$ and range of $H$ are orthogonal, in our subsequence analysis
it suffices to project $\overline{w}_{t}-w^{\ast}$ onto the range
of $H$, thus we may restrict to the non-zero eigenvalue of $H$. 

We can use $w^{\ast T}x_{k,j}-z_{k,j}\equiv0$ to rewrite the local
objectives 
\begin{align*}
F_{k}(w) & =\frac{1}{2n_{k}}\sum_{j=1}^{n_{k}}(w^{T}x_{k,j}-z_{k,j}-(w^{\ast T}x_{k,j}-z_{k,j}))^{2}\\
& =\frac{1}{2n_{k}}\sum_{j=1}^{n_{k}}((w-w^{\ast})^{T}x_{k,j})^{2}=\frac{1}{2}\langle w-w^{\ast},H^{k}(w-w^{\ast})\rangle=\frac{1}{2}\|w-w^{\ast}\|_{H^{k}}^{2}
\end{align*}
so that 
\begin{align*}
F(w) & =\sum_{k=1}^{N}p_{k}F_{k}(w)=\sum_{k=1}^{N}p_{k}\frac{1}{2}(w-w^{\ast})^{T}H^{k}(w-w^{\ast})=\frac{1}{2}(w-w^{\ast})^{T}H(w-w^{\ast})=\frac{1}{2}\|w-w^{\ast}\|_{H}^{2}
\end{align*}

Let $x_{t}^{k}$ be the stochastic sample on the $k$th device during
iteration $t$, and define $\tilde{H}_{t}^{k}:=x_{t}^{k}(x_{t}^{k})^{T}$
as the stochastic Hessian matrix corresponding to the sample $x_{t}^{k}$.
Note that $\mathbb{E}\tilde{H}_{t}^{k}=\frac{1}{n_{k}}\sum_{j=1}^{n_{k}}x_{k,j}(x_{k,j})^{T}=H^{k}$
so that $\tilde{H}_{t}^{k}$ is an unbiased estimate of $H^{k}$.
Moreover, we have
\begin{align*}
g_{t,k} & =\nabla F_{k}(w_{t}^{k},x_{t}^{k})=\tilde{H}_{t}^{k}(w_{t}^{k}-w^{\ast})\\
g_{t} & =\sum_{k=1}^{N}p_{k}\nabla F_{k}(w_{t}^{k},x_{t}^{k})=\sum_{k=1}^{N}p_{k}\tilde{H}_{t}^{k}(w_{t}^{k}-w^{\ast})
\end{align*}

Define $l$ to be the smallest positive number such that $\mathbb{E}\|x_{t}^{k}\|^{2}$$x_{t}^{k}$($x_{t}^{k})^{T}\preceq lH^{k}$
for all $k,t$. Note that $l\leq\max_{k=1,\dots,N}\max_{1\leq j\leq n_{k}}\|x_{k,j}\|^{2}$,
i.e. $l$ is upper bounded by the maximum norm squared among all feature
vectors on all devices. 

Let $L^{k}$ and $\mu^{k}$ to be the largest and smallest non-zero
eigenvalues of $H^{k}$, and $L:=\max_{k}L^{k}$, $\mu:=\min_{k}\mu^{k}$.
Define $\kappa_{1}:=l/\mu$ and $\kappa:=L/\mu$. 

Following {[}CITE: Liu\&Belkin and Jain et al{]}, we define the statistical
condition number $\tilde{\kappa}^{k}$ as the smallest positive real
number such that 
\begin{align*}
\mathbb{E}\left[\langle x_{t}^{k}(H^{k})^{-1},x_{t}^{k}\rangle x_{t}^{k}(x_{t}^{k})^{T}\right] & \preceq\tilde{\kappa}^{k}H^{k}
\end{align*}
and define $\tilde{\kappa}:=\max_{k}\tilde{\kappa}^{k}$. The condition
numbers $\kappa_{1}$ and $\tilde{\kappa}$ are important in the characterization
of convergence rates for FedAvg algorithms. 

Note that $\kappa_{1}>\kappa$ and $\kappa_{1}>\tilde{\kappa}$. 

\subsection{Exponential Convergence of FedAvg with local SGD: Over-parameterized
	Quadratic Objective}

Next we show that specialized to overparameterized quadratic objectives,
the exponential convergence bound can be further sharpened, with a
similar linear speedup in the number of workers. Recall that in this
setting, we have 
\begin{align*}
F(w) & =\sum_{k=1}^{N}p_{k}F_{k}(w)\\
F_{k}(w) & =\frac{1}{2n_{k}}\sum_{j=1}^{n_{k}}(w^{T}x_{k,j}-z_{k,j})^{2}
\end{align*}
and there exists $w^{\ast}$ such that $F_{k}(w^{\ast})\equiv0$
for all $k$. 
\begin{theorem}
	For the overparamterized linear regression problem, FedAvg with local
	SGD updates and communication every $E$ iterations with constant
	step size $\alpha_{t}=\frac{1}{4E}\frac{N}{l\nu_{\max}+\mu(N-\nu_{\min})}$
	gives the exponential convergence guarantee 
	\begin{align*}
	\mathbb{E}F(\overline{w}_{t}) & \leq L(1-\frac{N}{4E(\nu_{\max}\kappa_{1}+(N-\nu_{\min}))})^{t}\|w_{0}-w^{\ast}\|^{2}
	\end{align*}
	where $\kappa_{1}=l/\mu$. 
\end{theorem}
%
We see that when $N=O(\kappa_{1})$, the convergence rate is $O((1-\frac{N}{E\kappa_{1}})^{t})=O(\exp(-\frac{N}{E\kappa_{1}}t))$,
which exhibits linear speedup in the number of workers, as well as
a $1/\kappa_{1}$ dependence on the condition number $\kappa_{1}$.
In the last part of this section, we will demonstrate that a new FedAvg
algorithm based on a local momentum update {[}CITE Belkin{]} improves
upon this dependence to $1/\sqrt{\kappa_{1}\tilde{\kappa}}$.

 \subsection{Exponential Convergence of FedAvg with local MaSS: Over-parameterized
 	Quadratic Objective}
 
 Following {[}CITE{]}, we propose the FedAvg algorithm with local MaSS
 updates: 
 \begin{align*}
 w_{t+1}^{k} & =\begin{cases}
 u_{t}^{k}-\eta_{1}^{k}g_{t,k} & \text{if }t+1\notin\mathcal{I}_{E}\\
 \sum_{k=1}^{N}p_{k}\left[u_{t}^{k}-\eta_{1}^{k}g_{t,k}\right] & \text{if }t+1\in\mathcal{I}_{E}
 \end{cases}\\
 u_{t+1}^{k} & =w_{t+1}^{k}+\gamma^{k}(w_{t+1}^{k}-w_{t}^{k})+\eta_{2}^{k}g_{t,k}
 \end{align*}
 where we note that the natural parameter is $w_{t}$, while $u_{t}$
 is an auxiliary parameter, which we initialize to be $u_{0}^{k}$,
 $g_{t,k}=\nabla F_{k}(u_{t}^{k},x_{t}^{k})$ is the stochastic gradient
 and $g_{t}=\sum_{k=1}^{N}p_{k}g_{t,k}$ is the averaged stochastic
 gradient. When $\eta_{2}^{k}\equiv0$, this reduces to the FedAvg
 algorithm with Nesterov updates.
 
 Also note that the update can equivalently be written as 
 \begin{align*}
 v_{t+1}^{k} & =(1-\alpha^{k})v_{t}^{k}+\alpha^{k}u_{t}^{k}-\delta^{k}g_{t,k}\\
 w_{t+1}^{k} & =\begin{cases}
 u_{t}^{k}-\eta^{k}g_{t,k} & \text{if }t+1\notin\mathcal{I}_{E}\\
 \sum_{k=1}^{N}p_{k}\left[u_{t}^{k}-\eta^{k}g_{t,k}\right] & \text{if }t+1\in\mathcal{I}_{E}
 \end{cases}\\
 u_{t+1}^{k} & =\frac{\alpha^{k}}{1+\alpha^{k}}v_{t+1}^{k}+\frac{1}{1+\alpha^{k}}w_{t+1}^{k}
 \end{align*}
 where there is a bijection between the parameters 
 \begin{align*}
 \frac{1-\alpha^{k}}{1+\alpha^{k}} & =\gamma^{k}\\
 \eta^{k} & =\eta_{1}^{k}\\
 \frac{\eta^{k}-\alpha^{k}\delta^{k}}{1+\alpha^{k}} & =\eta_{2}^{k}
 \end{align*}
 and we further introduce an auxiliary parameter $v_{t}^{k}$, which
 is initialized at $v_{0}^{k}$. We also note that when $\delta^{k}=\frac{\eta^{k}}{\alpha^{k}}$,
 the update reduces to the Nesterov accelerated SGD. This version of
 the FedAvg algorithm with local MaSS updates is used for analyzing
 the exponential convergence. 
 
 As before, define the virtual sequences $\overline{w}_{t}=\sum_{k=1}^{N}p_{k}w_{t}^{k}$,
 $\overline{v}_{t}=\sum_{k=1}^{N}p_{k}v_{t}^{k}$, $\overline{u}_{t}=\sum_{k=1}^{N}p_{k}u_{t}^{k}$,
 and $\overline{g}_{t}=\sum_{k=1}^{N}p_{k}\mathbb{E}g_{t,k}$. We have
 $\mathbb{E}g_{t}=\overline{g}_{t}$ and $\overline{w}_{t+1}=\overline{u}_{t}-\eta_{t}g_{t}$,
 $\overline{v}_{t+1}=(1-\alpha^{k})\overline{v}_{t}+\alpha^{k}\overline{w}_{t}-\delta^{k}g_{t}$,
 and $\overline{u}_{t+1}=\frac{\alpha^{k}}{1+\alpha^{k}}\overline{v}_{t+1}+\frac{1}{1+\alpha^{k}}\overline{w}_{t+1}$. 
 
 We now present the exponential convergence result in overparameterized
 quadratic problems using FedAvg with local MaSS updates. 
 \begin{theorem}
 	(FedAvg with MaSS, Linear Regression) For the overparamterized quadratic
 	problem, FedAvg with local MaSS updates and communication every $E$
 	iterations with constant step sizes 
 	\begin{align*}
 	\eta_{t}=\frac{1}{4E}\frac{N}{l\nu_{\max}+\mu(N-\nu_{\min})}, & \alpha_{t}=\frac{1}{\sqrt{\kappa_{1}\tilde{\kappa}}},\delta_{t}=\frac{\eta_{t}}{\alpha_{t}\tilde{\kappa}}
 	\end{align*}
 	gives the exponential convergence guarantee 
 	\begin{align*}
 	\mathbb{E}F(\overline{w}_{t}) & \leq C(1-\frac{1}{4E}\frac{N}{l\nu_{\max}\sqrt{\kappa_{1}\tilde{\kappa}}+(N-\nu_{\min})})^{t}\|w_{0}-w^{\ast}\|^{2}
 	\end{align*}
 \end{theorem}
 Compared to the convergence rate of FedAvg with local SGD updates,
 we see that when $N=O(\sqrt{\kappa_{1}\tilde{\kappa}})$, the convergence
 rate is $O((1-\frac{N}{E\sqrt{\kappa_{1}\tilde{\kappa}}})^{t})=O(\exp(-\frac{N}{E\sqrt{\kappa_{1}\tilde{\kappa}}}t))$
 as opposd to $O(\exp(-\frac{N}{E\kappa_{1}}t))$. Since $\kappa_{1}\geq\tilde{\kappa}$,
 this implies a speedup factor of $\sqrt{\frac{\kappa_{1}}{\tilde{\kappa}}}$
 for the FedAvg with local MaSS updates. On the other hand, the same
 linear speedup in the number of workers holds for $N$ in a smaller
 range of values. 
% !TEX ROOT=./main.tex



\section{Numerical Experiments}


% \begin{figure}
% \begin{tabular}{cc}
% 	\includegraphics[width=0.5\textwidth]{fig/synthetic_balance_0_0_loss.pdf} & 
% 	\includegraphics[width=0.5\textwidth]{fig/synthetic_balance_0_0_accuracy.pdf} \\
% 	\includegraphics[width=0.5\textwidth]{fig/synthetic_balance_0_0_lossuser30TuneE.pdf} & 
% 	\includegraphics[width=0.5\textwidth]{fig/synthetic_balance_0_0_lossuser40TuneE.pdf} \\
% \end{tabular}
% \caption{Each user has 250 samples, non-iid setting, full device participation. Constant learning rate 0.01.}
% \end{figure}

We examine the practical speedup on a linear regression problem, 
$ F(\vw) = \sum_{k=1}^N p_kF_k(\vw)$, the objective function on each local device $k$ is given by, $F_k(\vw) = \frac{1}{N_k} \sum_{i=1}^{N_k} (\vw^T\vx_i^k + b  - y_i^k)^2$, where we generated i.i.d. data $\vx_i^k$ for all devices. At each communication round, all devices participated
in synchronization. 

\begin{figure}
\centering
	\includegraphics[width=0.5\textwidth]{fig/synthetic_linear_regression_1k_6k-epsilon01-logTrue.pdf}
	\caption{The linear speed up w.r.t the number of nodes. The synthetic dataset has $6000$ samples, evenly distributed on $10, 100, 1000$ devices. The figure shows the number of iterations needed to converge to $\epsilon=0.01$. The learning rate is decayed as the $\eta_t = \frac{1}{c + t \times a}$, where we extensively search the best learning rate $c \in \{1, 10\}$ and $a \in \{1\mathrm{e}{-2}, 1\mathrm{e}{-3}, 1\mathrm{e}{-4}, 1\mathrm{e}{-5}, 1\mathrm{e}{-6}\}$ for each configuration.}
\end{figure}



% !TEX ROOT=./main.tex

\section{Conclusions}
This paper provides a comprehensive analysis of the convergence rate of FedAvg
and its accelerated variants. We show that both accelerated FedAvg and FedAvg
can achieve speedup when the number of nodes increases, i.e., $O(1/\sqrt{NT})$
convergence for convex smooth problems and $O(1/NT)$ convergence for strongly 
convex problems. Furthermore, we show FedAvg can achieve exponential 
convergence for overparameterized strongly convex smooth problems, and we propose Mass accelerated Fedavg, which show improved convergence rate over
FedAvg on the linear regression problem. Last but not least, we empirically
verify the linear speedup of FedAvg and Nesterov accelerated FedAvg for strongly convex, convex smooth, and linear regression problems. The empirical results are well-aligned with our theories. 


\newpage
% !TEX ROOT=./main.tex

% Broader impact: Authors are asked to include a section in their submissions discussing the broader impact of their work, including possible societal consequences — -both positive and negative.
\section{Broader impact}
With the widespread use of mobile devices, the powerful sensors
collected enormous amount of data empowered with machine learning
solutions can transform people's life in many aspects. The proper
data integration can produce positive social
impact in terms of better personalization~\cite{fallah2020personalized}, more efficient information solicitation~\cite{chen2018federated}, more robust and stable service~\cite{47586}.

In the meanwhile, the increasing concerns of security and
privacy leaking from both large-companies and the entire society
present significant challenges of proper utilization of local
data. This trade-off between efficiency and privacy or security
lies in the central of many industrial applications, including 
but not limited to recommender systems~\cite{chen2018federated}, virtual assistants~\cite{lamautonomy}, and mobile keyboard prediction~\cite{47586}

In this work, we establish solid theoretical foundation for federated
learning that achieves efficiency and privacy at the same time. 
On one hand, the data is isolated in each local devices, thus prevent
the potential leaking of private information. On the otherhand, the
convergence rates improved substantially if the number of
active local devices increases. In conclusion, our work 
shows a faithful pathway to simultaneously secure people's privacy and improve
users' experience.




\bibliographystyle{unsrt}
\bibliography{ref.bib}

\appendix
% !TEX ROOT=./main.tex





\section{Proof for Convergence Results on Strongly Convex and Smooth Objectives}

\subsection{SGD}
We first summarize some properties of $L$-smooth and $\mu$-strongly
convex functions. 
\begin{lemma}
	Let $F$ be a convex $L$-smooth function. Then we have the following
	inequalities:
	
	1. Quadratic upper bound: $0\leq F(w)-F(w')-\langle\nabla F(w'),w-w'\rangle\leq\frac{L}{2}\|w-w'\|^{2}$. 
	
	2. Coercivity: $\frac{1}{L}\|\nabla F(w)-\nabla F(w')\|^{2}\leq\langle\nabla F(w)-\nabla F(w'),w-w'\rangle$.
	
	3. Lower bound: $F(w)\geq F(w')+\langle\nabla F(w'),w-w'\rangle+\frac{1}{2L}\|\nabla F(w)-\nabla F(w')\|^{2}$.
	In particular, $\|\nabla F(w)\|^{2}\leq2L(F(w)-F(w^{\ast}))$.
	
	4. Optimality gap: $F(w)-F(w^{\ast})\leq$$\langle\nabla F(w),w-w^{\ast}\rangle$.
\end{lemma}
%
\begin{lemma}
	Let $F$ be a $\mu$-strongly convex function. Then 
	\begin{align*}
	F(w) & \leq F(w')+\langle\nabla F(w'),w-w'\rangle+\frac{1}{2\mu}\|\nabla F(w)-\nabla F(w')\|^{2}\\
	F(w)-F(w^{\ast}) & \leq\frac{1}{2\mu}\|\nabla F(w)\|^{2}
	\end{align*}
\end{lemma}
%
\begin{assumption}
	The stochastic gradients have bounded variance 
	\begin{align*}
	\mathbb{E}\|\nabla F_{k}(w_{t}^{k},\xi_{t}^{k})-\nabla F_{k}(w_{t}^{k})\|^{2} & \leq\sigma_{k}^{2}
	\end{align*}
	for all $k$. Moreover, the expected squared norm of stochastic gradients
	is uniformly bounded, 
	\begin{align*}
	\mathbb{E}\|\nabla F_{k}(w_{t}^{k},\xi_{t}^{k})\|^{2} & \leq G^{2}
	\end{align*}
\end{assumption}
\begin{theorem}
	Suppose $F_{k}$ is $L$-smooth and $\mu$-strongly convex for all
	$k$. Let $\kappa=\frac{L}{\mu}$, $\gamma=\max\{8\kappa-1,E\}$ where
	$E$ is the communication interval, and learning rates 
	\begin{align*}
	\alpha_{t} & =\frac{c}{\mu(\gamma+t)}
	\end{align*}
	so that $\alpha_{t}\leq2\alpha_{t+E}$, and where $0\leq c\leq1$
	is small enough such that
	$\alpha_{t} \leq\frac{1}{4L}$ and 
	$\alpha_{t} \leq\frac{1}{N}$ for all $t\geq0$. 
	
	Then with full device participation, if $\nu_{max}=\max_{k}\frac{1}{N}p_{k}$,
	we have 
	\begin{align*}
	\mathbb{E}F(\overline{w}_{T})-F^{\ast}\leq C\frac{\kappa}{\gamma+T}\cdot(8E^{2}LG^{2}/T+\frac{1}{N}\nu_{max}^{2}\sigma^{2})
	\end{align*}
	so that as long as $E=O(\sqrt{\frac{N}{T}})$, 
	\begin{align*}
	\mathbb{E}F(w_{T})-F^{\ast} & \leq C(8LG^{2}+\nu_{max}^{2}\sigma^{2})\frac{\kappa}{N(\gamma+T)}
	\end{align*}
	and with partial participation with $K$ sampled devices at each
	communication round, $\frac{4}{K}\alpha_{t}^{2}E^{2}G^{2}$
	\begin{align*}
	\mathbb{E}F(\overline{w}_{T})-F^{\ast}\leq C\frac{\kappa}{\gamma+T}\cdot(8E^{2}LG^{2}/T+\frac{1}{N}\nu_{max}^{2}\sigma^{2}+\frac{4}{K}E^{2}G^{2})
	\end{align*}
	so that if $E=O(1)$, 
	\begin{align*}
	\mathbb{E}F(\overline{w}_{T})-F^{\ast} & \leq C(4E^{2}G^{2}+\nu_{max}^{2}\sigma^{2})\frac{\kappa}{K(\gamma+T)}+\frac{8E^{2}LG^{2}}{T^{2}}
	\end{align*}
\end{theorem}

\begin{proof}
	The idea of the proof is to observe that the $L$-smoothness of $F$
	provides the upper bound
	\begin{align*}
	\mathbb{E}(F(\overline{w}_{t}))-F^{\ast} & =\mathbb{E}(F(\overline{w}_{t})-F(w^{\ast}))\\
	& \leq\frac{L}{2}\mathbb{E}\|\overline{w}_{t}-w^{\ast}\|^{2}
	\end{align*}
	and to show that $\mathbb{E}\|\overline{w}_{T}-w^{\ast}\|^{2}=O(\frac{1}{NT})$. 
	
	Our main step is to prove the bound 
	\begin{align*}
	\mathbb{E}\|\overline{w}_{t+1}-w^{\ast}\|^{2} & \leq(1-\mu\alpha_{t})\mathbb{E}\|\overline{w}_{t}-w^{\ast}\|^{2}+\alpha_{t}^{2}B
	\end{align*}
	where $B$ is define in the statement of the theorem. 
	
	We have 
	\begin{align*}
	\|\overline{w}_{t+1}-w^{\ast}\|^{2} & =\|(\overline{w}_{t}-\alpha_{t}g_{t})-w^{\ast}\|^{2}\\
	& =\|(\overline{w}_{t}-\alpha_{t}\overline{g}_{t}-w^{\ast})-\alpha_{t}(g_{t}-\overline{g}_{t})\|^{2}\\
	& =A_{1}+A_{2}+A_{3}
	\end{align*}
	where 
	\begin{align*}
	A_{1} & =\|\overline{w}_{t}-w^{\ast}-\alpha_{t}\overline{g}_{t}\|^{2}\\
	A_{2} & =2\alpha_{t}\langle\overline{w}_{t}-w^{\ast}-\alpha_{t}\overline{g}_{t},\overline{g}_{t}-g_{t}\rangle\\
	A_{3} & =\alpha_{t}^{2}\|g_{t}-\overline{g}_{t}\|^{2}
	\end{align*}
	$\mathbb{E}A_{2}=0$ by definition of $g_{t}$ and $\overline{g}_{t}$,
	while for $A_{3}$ we have
	\begin{align*}
	\alpha_{t}^{2}\mathbb{E}\|g_{t}-\overline{g}_{t}\|^{2} & =\alpha_{t}^{2}\mathbb{E}\|g_{t}-\mathbb{E}g_{t}\|^{2}=\alpha_{t}^{2}\sum_{k=1}^{N}p_{k}^{2}\|g_{t,k}-\mathbb{E}g_{t,k}\|^{2}\leq\alpha_{t}^{2}\sum_{k=1}^{N}p_{k}^{2}\sigma_{k}^{2}
	\end{align*}
	again by Jensen's inequality and using the independence of $g_{t,k},g_{t,k'}$. 
	
	Next we bound $A_{1}$: 
	\begin{align*}
	\|\overline{w}_{t}-w^{\ast}-\alpha_{t}\overline{g}_{t}\|^{2} & =\|\overline{w}_{t}-w^{\ast}\|^{2}+2\langle\overline{w}_{t}-w^{\ast},-\alpha_{t}\overline{g}_{t}\rangle+\|\alpha_{t}\overline{g}_{t}\|^{2}
	\end{align*}
	and we will show that the third term $\|\alpha_{t}\overline{g}_{t}\|^{2}$
	can be canceled by an upper bound of the second term. %
	\begin{comment}
	The last term is straightforward to bound by the convexity of $\|\cdot\|^{2}$
	and $L$-smoothness of $F_{k}$,
	\begin{align*}
	\alpha_{t}^{2}\|\overline{g}_{t}\|^{2} & \leq\alpha_{t}^{2}\sum_{k=1}^{N}p_{k}\|\nabla F_{k}(w_{t}^{k})\|^{2}\leq2L\alpha_{t}^{2}\sum_{k=1}^{N}p_{k}(F_{k}(w_{t}^{k})-F_{k}^{\ast})
	\end{align*}
	or 
	\begin{align*}
	\alpha_{t}^{2}\|\overline{g}_{t}\|^{2} & \leq\alpha_{t}^{2}\sum_{k=1}^{N}p_{k}\|\nabla F_{k}(w_{t}^{k})\|^{2}\leq\alpha_{t}^{2}\sum_{k=1}^{N}p_{k}\mathbb{E}\|\nabla F_{k}(w_{t}^{k},\xi_{t}^{k})\|^{2}\leq\alpha_{t}^{2}G^{2}
	\end{align*}
	\end{comment}
	
	Now 
	\begin{align*}
	-2\alpha_{t}\langle\overline{w}_{t}-w^{\ast},\overline{g}_{t}\rangle & =-2\alpha_{t}\sum_{k=1}^{N}p_{k}\langle\overline{w}_{t}-w^{\ast},\nabla F_{k}(w_{t}^{k})\rangle\\
	& =-2\alpha_{t}\sum_{k=1}^{N}p_{k}\langle\overline{w}_{t}-w_{t}^{k},\nabla F_{k}(w_{t}^{k})\rangle-2\alpha_{t}\sum_{k=1}^{N}p_{k}\langle w_{t}^{k}-w^{\ast},\nabla F_{k}(w_{t}^{k})\rangle\\
	& \leq-2\alpha_{t}\sum_{k=1}^{N}p_{k}\langle\overline{w}_{t}-w_{t}^{k},\nabla F_{k}(w_{t}^{k})\rangle+2\alpha_{t}\sum_{k=1}^{N}p_{k}(F_{k}(w^{\ast})-F_{k}(w_{t}^{k}))-\alpha_{t}\mu\sum_{k=1}^{N}p_{k}\|w_{t}^{k}-w^{\ast}\|^{2}\\
	& \leq2\alpha_{t}\sum_{k=1}^{N}p_{k}\left[F_{k}(w_{t}^{k})-F_{k}(\overline{w}_{t})+\frac{L}{2}\|\overline{w}_{t}-w_{t}^{k}\|^{2}+F_{k}(w^{\ast})-F_{k}(w_{t}^{k})\right]-\alpha_{t}\mu\|\sum_{k=1}^{N}p_{k}w_{t}^{k}-w^{\ast}\|^{2}\\
	& =\alpha_{t}L\sum_{k=1}^{N}p_{k}\|\overline{w}_{t}-w_{t}^{k}\|^{2}+2\alpha_{t}\sum_{k=1}^{N}p_{k}\left[F_{k}(w^{\ast})-F_{k}(\overline{w}_{t})\right]-\alpha_{t}\mu\|\overline{w}_{t}-w^{\ast}\|^{2}
	\end{align*}
	For the second term, which is negative, we can ignore it, but this
	yields a suboptimal bound that fails to provide the desired linear
	speedup. This is the flaw in the analysis of the ICLR paper. Instead,
	we upper bound it using the following derivation: 
	\begin{align*}
	2\alpha_{t}\sum_{k=1}^{N}p_{k}\left[F_{k}(w^{\ast})-F_{k}(\overline{w}_{t})\right] & \leq2\alpha_{t}\left[F(\overline{w}_{t+1})-F(\overline{w}_{t})\right]\\
	& \leq2\alpha_{t}\mathbb{E}\langle\nabla F(\overline{w}_{t}),\overline{w}_{t+1}-\overline{w}_{t}\rangle+\alpha_{t}L\mathbb{E}\|\overline{w}_{t+1}-\overline{w}_{t}\|^{2}\\
	& =-2\alpha_{t}^{2}\mathbb{E}\langle\nabla F(\overline{w}_{t}),g_{t}\rangle+\alpha_{t}^{3}L\mathbb{E}\|g_{t}\|^{2}\\
	& =-2\alpha_{t}^{2}\mathbb{E}\langle\nabla F(\overline{w}_{t}),\overline{g}_{t}\rangle+\alpha_{t}^{3}L\mathbb{E}\|g_{t}\|^{2}\\
	& =-\alpha_{t}^{2}\left[\|\nabla F(\overline{w}_{t})\|^{2}+\|\overline{g}_{t}\|^{2}-\|\nabla F(\overline{w}_{t})-\overline{g}_{t}\|^{2}\right]+\alpha_{t}^{3}L\mathbb{E}\|g_{t}\|^{2}\\
	& =-\alpha_{t}^{2}\left[\|\nabla F(\overline{w}_{t})\|^{2}+\|\overline{g}_{t}\|^{2}-\|\nabla F(\overline{w}_{t})-\sum_{k}p_{k}\nabla F(w_{t}^{k})\|^{2}\right]+\alpha_{t}^{3}L\mathbb{E}\|g_{t}\|^{2}\\
	& \leq-\alpha_{t}^{2}\left[\|\nabla F(\overline{w}_{t})\|^{2}+\|\overline{g}_{t}\|^{2}-\sum_{k}p_{k}\|\nabla F(\overline{w}_{t})-\nabla F(w_{t}^{k})\|^{2}\right]+\alpha_{t}^{3}L\mathbb{E}\|g_{t}\|^{2}\\
	& \leq-\alpha_{t}^{2}\left[\|\nabla F(\overline{w}_{t})\|^{2}+\|\overline{g}_{t}\|^{2}-L^{2}\sum_{k}p_{k}\|\overline{w}_{t}-w_{t}^{k}\|^{2}\right]+\alpha_{t}^{3}L\mathbb{E}\|g_{t}\|^{2}\\
	& \leq-\alpha_{t}^{2}\|\overline{g}_{t}\|^{2}+\alpha_{t}^{2}L^{2}\sum_{k}p_{k}\|\overline{w}_{t}-w_{t}^{k}\|^{2}+\alpha_{t}^{3}L\mathbb{E}\|g_{t}\|^{2}
	\end{align*}
	where we have used the smoothness of $F$ twice. 
	
	Note that the term $-\alpha_{t}^{2}\|\overline{g}_{t}\|^{2}$ exactly
	cancels the $\alpha_{t}^{2}\|\overline{g}_{t}\|^{2}$ in the bound
	for $A_{1}$, so that plugging in the bound for $-2\alpha_{t}\langle\overline{w}_{t}-w^{\ast},\overline{g}_{t}\rangle$,
	we have so far proved 
	\begin{align*}
	\mathbb{E}\|\overline{w}_{t+1}-w^{\ast}\|^{2} & \leq\mathbb{E}(1-\mu\alpha_{t})\|\overline{w}_{t}-w^{\ast}\|^{2}+\alpha_{t}L\sum_{k=1}^{N}p_{k}\|\overline{w}_{t}-w_{t}^{k}\|^{2}+\alpha_{t}^{2}\sum_{k=1}^{N}p_{k}^{2}\sigma_{k}^{2}\\
	& +\alpha_{t}^{2}L^{2}\sum_{k=1}^{N}p_{k}\|\overline{w}_{t}-w_{t}^{k}\|^{2}+\alpha_{t}^{3}L\mathbb{E}\|g_{t}\|^{2}
	\end{align*}
	The term $\mathbb{E}\|g_{t}\|^{2}\leq G^{2}$ by assumption. 
	
	Now we bound $\mathbb{E}\sum_{k=1}^{N}p_{k}\|\overline{w}_{t}-w_{t}^{k}\|^{2}$.
	Since communication is done every $E$ steps, for any $t\geq0$, we
	can find a $t_{0}\leq t$ such that $t-t_{0}\leq E-1$ and $w_{t_{0}}^{k}=\overline{w}_{t_{0}}$for
	all $k$. Moreover, using $\eta_{t}$ is non-increasing and $\alpha_{t_{0}}\leq\alpha{}_{t}$
	for any $t-t_{0}\leq E-1$, we have 
	\begin{align*}
	\mathbb{E}\sum_{k=1}^{N}p_{k}\|\overline{w}_{t}-w_{t}^{k}\|^{2} & =\mathbb{E}\sum_{k=1}^{N}p_{k}\|w_{t}^{k}-\overline{w}_{t_{0}}-(\overline{w}_{t}-\overline{w}_{t_{0}})\|^{2}\\
	& \leq\mathbb{E}\sum_{k=1}^{N}p_{k}\|w_{t}^{k}-\overline{w}_{t_{0}}\|^{2}\\
	& =\mathbb{E}\sum_{k=1}^{N}p_{k}\|w_{t}^{k}-w_{t_{0}}^{k}\|^{2}\\
	& =\mathbb{E}\sum_{k=1}^{N}p_{k}\|-\sum_{i=t_{0}}^{t-1}\alpha_{i}g_{i,k}\|^{2}\\
	& \leq\sum_{k=1}^{N}p_{k}\mathbb{E}\sum_{i=t_{0}}^{t-1}(E-1)\alpha_{i}^{2}\|g_{i,k}\|^{2}\\
	& \leq\sum_{k=1}^{N}p_{k}\mathbb{E}\sum_{i=t_{0}}^{t-1}(E-1)\alpha_{t_{0}}^{2}G^{2}\\
	& \leq4E^{2}\alpha_{t}^{2}G^{2}
	\end{align*}
	
	Using the bound on $\mathbb{E}\sum_{k=1}^{N}p_{k}\|\overline{w}_{t}-w_{t}^{k}\|^{2}$,
	we can conclude that, with $\nu_{max}:=N\cdot\max_{k}p_{k}$ and $\nu_{min}:=N\cdot\min_{k}p_{k}$, 
	
	\begin{align*}
	\mathbb{E}\|\overline{w}_{t+1}-w^{\ast}\|^{2} & \leq\mathbb{E}(1-\mu\alpha_{t})\|\overline{w}_{t}-w^{\ast}\|^{2}+4E^{2}L\alpha_{t}^{3}G^{2}+4E^{2}L^{2}\alpha_{t}^{4}G^{2}+\alpha_{t}^{2}\sum_{k=1}^{N}p_{k}^{2}\sigma_{k}^{2}+\alpha_{t}^{3}LG^{2}\\
	& =\mathbb{E}(1-\mu\alpha_{t})\|\overline{w}_{t}-w^{\ast}\|^{2}+4E^{2}L\alpha_{t}^{3}G^{2}+4E^{2}L^{2}\alpha_{t}^{4}G^{2}+\alpha_{t}^{2}\frac{1}{N^{2}}\sum_{k=1}^{N}(p_{k}N)^{2}\sigma_{k}^{2}+\alpha_{t}^{3}LG^{2}\\
	& \leq\mathbb{E}(1-\mu\alpha_{t})\|\overline{w}_{t}-w^{\ast}\|^{2}+4E^{2}L\alpha_{t}^{3}G^{2}+4E^{2}L^{2}\alpha_{t}^{4}G^{2}+\alpha_{t}^{2}\frac{1}{N^{2}}\nu_{max}^{2}\sum_{k=1}^{N}\sigma_{k}^{2}+\alpha_{t}^{3}LG^{2}\\
	& \leq\mathbb{E}(1-\mu\alpha_{t})\|\overline{w}_{t}-w^{\ast}\|^{2}+4E^{2}L\alpha_{t}^{3}G^{2}+4E^{2}L^{2}\alpha_{t}^{4}G^{2}+\alpha_{t}^{2}\frac{1}{N}\nu_{max}^{2}\sigma^{2}+\alpha_{t}^{3}LG^{2}
	\end{align*}
	where $\sigma^{2}=\max_{k}\sigma_{k}^{2}$. We show next that $\|\overline{w}_{t}-w^{\ast}\|^{2}=O(\frac{\alpha_{t}}{N})=O(\frac{1}{Nt})$.
	Before we do, we remark briefly on the choice of $E$. The leading
	term among terms other than $\mathbb{E}(1-\mu\alpha_{t})\|\overline{w}_{t}-w^{\ast}\|^{2}$
	is $\alpha_{t}^{2}\frac{1}{N}\nu_{max}^{2}\sigma^{2}$. This means
	we can take $E=O(\sqrt{\frac{1}{N\alpha_{t}}})=O(\sqrt{\frac{t}{N}})$
	which means $4E^{2}L\alpha_{t}^{3}G^{2}=O(\frac{\alpha_{t}}{N})$
	will be on the same order as the leading term. This is in contrast
	with previous results where $E$ cannot scale with $t$. Thus our
	bound yields a more efficient communication complexity than current
	analyses. We note however that this is only possible when there is
	full participation, and when $\nu_{max}$ is close to 1. When there
	is partial participation, however, $E$ has to stay $O(1)$, as we
	will soon see. Thus our result hightlights the difference between
	the communication efficiency of full and partial participation when
	an $O(1/T)$ bound on the optimality gap is desired. 
	
	\textbf{}%
	\begin{comment}
	The leading term on the right hand side the bound is $\alpha_{t}^{2}\frac{1}{N}\nu_{max}^{2}\sigma^{2}$,
	which when summed using the recursive relations gives $O(\alpha_{t})=O(1/t)$,
	but the constant now is scaled down by $1/N$. \textbf{To achieve
	this linear speedup, we need $\sum_{t=0}^{T}(1-\mu\alpha_{t})\|w_{0}-w^{\ast}\|^{2}$
	to be small enough. Note also that the above bound implies that $E$
	can be chosen to be as large as $O(\sqrt{\frac{T}{N}})$ without degrading
	the performance, since $E^{2}\alpha_{t}^{3}=O(\alpha_{t}^{2}/N)$
	which is the order of the leading term. As we will see, in the partial
	participation case, this is no longer true, and $E$ cannot be chosen
	$O(T^{\beta})$ for any $\beta>0$. }
	\end{comment}
	
	To complete the proof of convergence, let $B=$ $B_{t}=4E^{2}L\alpha_{t}G^{2}+4E^{2}L^{2}\alpha_{t}^{2}G^{2}+\frac{1}{N}\nu_{max}^{2}\sigma^{2}+\alpha_{t}LG^{2}$.
	Set the learning rate $\alpha_{t}=\frac{\delta}{t+\gamma}$ for $\delta>\frac{1}{\mu}$
	and $\gamma=\max\{8\kappa-1,E\}$ such that $\alpha_{1}\leq\min\{\frac{1}{\mu},\frac{1}{4L}\}$and
	$\alpha_{t}\leq2\alpha_{t+E}$, we can show that 
	\begin{align*}
	\mathbb{E}\|\overline{w}_{t}-w^{\ast}\|^{2} & \leq\frac{v}{N(\gamma+t)}
	\end{align*}
	where $v=\max\{\frac{\delta^{2}B}{\delta\mu-1},(\gamma+1)\|w_{0}-w^{\ast}\|^{2}\}$,
	by induction. The definition of $v$ ensures that 
	\begin{align*}
	\|w_{0}-w^{\ast}\|^{2} & \leq\frac{v}{\gamma}
	\end{align*}
	and suppose $\mathbb{E}\|\overline{w}_{t}-w^{\ast}\|^{2}\leq\frac{v}{N(\gamma+t)}$
	holds for some $t\geq0$. Then 
	\begin{align*}
	\mathbb{E}\|\overline{w}_{t+1}-w^{\ast}\|^{2}\leq & (1-\alpha_{t}\mu)\mathbb{E}\|\overline{w}_{t}-w^{\ast}\|^{2}+\alpha_{t}^{2}B'\\
	& \leq(1-\frac{\delta\mu}{t+\gamma})\frac{v}{N(t+\gamma)}+\frac{\delta^{2}B'}{N^{2}(t+\gamma)^{2}}\\
	& =\frac{t+\gamma-1}{N(t+\gamma)^{2}}v+\left[\frac{\delta^{2}B'}{N^{2}(t+\gamma)^{2}}-\frac{\delta\mu-1}{N(t+\gamma)^{2}}v\right]\\
	& \leq\frac{t+\gamma-1}{N(t+\gamma)^{2}}v+\left[\frac{\delta^{2}B'}{N(t+\gamma)^{2}}-\frac{\delta\mu-1}{N(t+\gamma)^{2}}v\right]\\
	& \leq\frac{1}{N}\frac{v}{t+\gamma+1}
	\end{align*}
	by definition of $v$.
	
	Finally, the $L$-smoothness of $F$ implies 
	\begin{align*}
	\mathbb{E}(F(\overline{w}_{t}))-F^{\ast} & =\mathbb{E}(F(\overline{w}_{t})-F(w^{\ast}))\\
	& \leq\frac{L}{2}\mathbb{E}\|\overline{w}_{t}-w^{\ast}\|^{2}\leq\frac{L}{2}\frac{v}{\gamma+t}
	\end{align*}
	Now if we choose $\delta=\frac{c}{\mu}$ where $c\leq1$ is small
	enough such that the required conditions 
	\begin{align*}
	\alpha_{t}^{2}+\beta_{t-1}^{2} & \leq\frac{1}{2}\\
	\alpha_{t} & \leq\frac{1}{4L}\\
	4\alpha_{t-1}^{2} & \leq\alpha_{t}
	\end{align*}
	hold for all $t\geq0$, $\gamma=\max\{8\kappa-1,E\}$, then $\alpha_{t}=\frac{c}{\mu}\frac{1}{\gamma+t}$
	and recalling $v=\max\{\frac{\delta^{2}B'}{\delta\mu-c+1},(\gamma+1)\|w_{0}-w^{\ast}\|^{2}\}$,
	we finally have 
	\begin{align*}
	\mathbb{E}(F(\overline{w}_{t}))-F^{\ast} & \leq\frac{L}{2}\frac{v}{\gamma+t}\leq\frac{L/2}{\gamma+t}(\frac{\delta^{2}B'}{\delta\mu-c+1}+(\gamma+1)\|w_{0}-w^{\ast}\|^{2})\\
	& \leq\frac{\kappa}{\gamma+t}(\frac{B'}{\mu}+4L\|w_{0}-w^{\ast}\|^{2})
	\end{align*}
	where $B'=8E^{2}L^{2}G^{2}/t+\frac{1}{N}\nu_{max}^{2}\sigma^{2}$. 
	
	With partial participation, the update at each communication round
	is now given by averages over a subset of sampled devices. The key
	is to bound 
	\begin{align*}
	\mathbb{E}\|\overline{w}_{t}-\overline{y}_{t}\|^{2} & =\frac{1}{K}\sum_{k}p^{k}\|w_{t}^{k}-\sum_{k}p_{k}w_{t}^{k}\|^{2}
	\end{align*}
	This can be bounded by the same argument as before and yield 
	\begin{align*}
	\mathbb{E}\|\overline{w}_{t}-\overline{y}_{t}\|^{2} & \leq\frac{4}{K}\alpha_{t}^{2}E^{2}G^{2}
	\end{align*}
	and so this is also a leading term, and together with the term $\alpha_{t}^{2}\frac{1}{N}\nu_{max}^{2}\sigma^{2}$
	determines the convergence rate, which is now $O(\frac{(\nu_{max}^{2}\sigma^{2}+4E^{2}G^{2})}{KT})$.
	This implies that $E$ canot be chosen $O(T^{\beta})$ for any $\beta>0$
	without degrading the performance. This should be checked in experiments,
	whether with partial participation if the communication round is set
	to scale with $T$, the convergence deteriorates. This is in constrast
	with the full participation case, where $E=O(\sqrt{\frac{T}{N}})$
	is allowed. \textbf{If we can confirm that full participation allows
		linear speedup with $\nu=N\cdot\max_{k}p_{k}\approx1$ and $E=O(\sqrt{\frac{T}{N}})$,
		whereas partial participation only allows $E=O(1)$, then this would
		an interesting phenomenon that is not reported by previous studies!}
\end{proof}

\subsection{Accelerated SGD}
\label{sec:nasgdscvxsmth}



We show that FedAv with Accelerated SGD has $O(1/T)$ rate under $\mu$-strong
convexity and $L$-smoothness. The proof follows the framework of
the ICLR paper. The FedAv algorithm with Nesterov Accelerated SGD
(NASGD) follows the updates
\begin{align*}
y_{t+1}^{k} & =w_{t}^{k}-\alpha_{t}g_{t,k}\\
w_{t+1}^{k} & =\begin{cases}
y_{t+1}^{k}+\beta_{t}(y_{t+1}^{k}-y_{t}^{k}) & \text{if }t+1\notin\mathcal{I}_{E}\\
\sum_{k=1}^{N}p_{k}\left[y_{t+1}^{k}+\beta_{t}(y_{t+1}^{k}-y_{t}^{k})\right] & \text{if }t+1\in\mathcal{I}_{E}
\end{cases}
\end{align*}

and define the virtual sequences $\overline{y}_{t}=\sum_{k=1}^{N}p_{k}y_{t}^{k}$,
$\overline{w}_{t}=\sum_{k=1}^{N}p_{k}w_{t}^{k}$, and $\overline{g}_{t}=\sum_{k=1}^{N}p_{k}\mathbb{E}g_{t,k}$.
We have $\mathbb{E}g_{t}=\overline{g}_{t}$ and $\overline{y}_{t+1}=\overline{w}_{t}-\alpha_{t}g_{t}$,
and $\overline{w}_{t+1}=\overline{y}_{t+1}+\beta_{t}(\overline{y}_{t+1}-\overline{y}_{t})$. 
\begin{theorem}
Let the parameters satisfy the assumptions in the ICLR paper and learning
rate $\alpha_{t}=\frac{2}{\mu(\gamma+t)}$, $\beta_{t}$ such that
$\alpha_{t}^{2}+\beta_{t-1}^{2}\leq\frac{1}{2}$, $\beta_{t}\leq\alpha_{t}$
for all $t$. Then with full device participation, 
\begin{align*}
\mathbb{E}F(w_{T})-F^{\ast} & \leq\frac{2\kappa}{\gamma+T}(\frac{B}{\mu}+2L(\|w_{0}-w^{\ast}\|^{2})\\
B & =\sum_{k=1}^{N}p_{k}^{2}\sigma_{k}^{2}+9L\Gamma+32(E-1)^{2}G^{2}+2+G^{2}+GK
\end{align*}
 and $K$ is such that 
\begin{align*}
\alpha_{0}B+2\sqrt{K}\cdot G & \leq\mu K
\end{align*}
 and
\begin{align*}
\|w_{0}-w^{\ast}\|^{2} & \leq K
\end{align*}
\end{theorem}
\begin{proof}
We have the recursion 
\begin{align*}
y_{t+1}^{k}-y_{t}^{k} & =w_{t}^{k}-w_{t-1}^{k}-(\alpha_{t}g_{t,k}-\alpha_{t-1}g_{t-1,k})\\
w_{t+1}^{k}-w_{t}^{k} & =-\alpha_{t}g_{t,k}+\beta_{t}(y_{t+1}^{k}-y_{t}^{k})
\end{align*}
 so that 
\begin{align*}
y_{t+1}^{k}-y_{t}^{k} & =-\alpha_{t-1}g_{t-1,k}+\beta_{t-1}(y_{t}^{k}-y_{t-1}^{k})-(\alpha_{t}g_{t,k}-\alpha_{t-1}g_{t-1,k})\\
 & =\beta_{t-1}(y_{t}^{k}-y_{t-1}^{k})-\alpha_{t}g_{t,k}
\end{align*}

First, we derive a bound on $\mathbb{E}\|\overline{y}_{t+1}-\overline{y}_{t}\|^{2}$
that is useful in the proof. Since the identity $y_{t+1}^{k}-y_{t}^{k}=\beta_{t-1}(y_{t}^{k}-y_{t-1}^{k})-\alpha_{t}g_{t,k}$
implies 
\begin{align*}
\mathbb{E}\|y_{t+1}^{k}-y_{t}^{k}\|^{2} & \leq2\beta_{t-1}^{2}\mathbb{E}\|y_{t}^{k}-y_{t-1}^{k}\|^{2}+2\alpha_{t}^{2}G^{2}
\end{align*}
\textbf{as long as $\alpha_{t},\beta_{t}$ satisfy $2\beta_{t-1}^{2}+2\alpha_{t}^{2}\leq1$,
and $\mathbb{E}\|w_{0}-\alpha_{t}g_{t,k}\|^{2}\leq G^{2}$,} we can
guarantee that $\mathbb{E}\|y_{t}^{k}-y_{t-1}^{k}\|^{2}\leq G^{2}$.
This together with Jensen implies $\mathbb{E}\|\overline{y}_{t}-\overline{y}_{t-1}\|^{2}\leq G^{2}$. 

Now we turn to $\|\overline{w}_{t+1}-w^{\ast}\|^{2}$. We have 
\begin{align*}
\|\overline{w}_{t+1}-w^{\ast}\|^{2} & =\|(\overline{w}_{t}-\alpha_{t}g_{t})+\beta_{t}(\overline{y}_{t+1}-\overline{y}_{t})-w^{\ast}\|^{2}\\
 & =\|(\overline{w}_{t}-\alpha_{t}\overline{g}_{t}-w^{\ast})+\beta_{t}(\overline{y}_{t+1}-\overline{y}_{t})-\alpha_{t}(\overline{g}_{t}-g_{t})\|^{2}\\
 & =A_{1}+A_{2}+\alpha_{t}^{2}\|g_{t}-\overline{g}_{t}\|^{2}
\end{align*}
where 
\begin{align*}
A_{1} & =\|\overline{w}_{t}-w^{\ast}-\alpha_{t}\overline{g}_{t}+\beta_{t}(\overline{y}_{t+1}-\overline{y}_{t})\|^{2}\\
A_{2} & =2\alpha_{t}\langle\overline{w}_{t}-w^{\ast}-\alpha_{t}\overline{g}_{t}+\beta_{t}(\overline{y}_{t+1}-\overline{y}_{t}),\overline{g}_{t}-g_{t}\rangle
\end{align*}
and $\mathbb{E}A_{2}=0$ by definition of $g_{t}$ and $\overline{g}_{t}$.
Next we bound $A_{1}$: 
\begin{align*}
\|\overline{w}_{t}-w^{\ast}-\alpha_{t}\overline{g}_{t}+\beta_{t}(\overline{y}_{t+1}-\overline{y}_{t})\|^{2} & =\|\overline{w}_{t}-w^{\ast}\|^{2}+2\langle\overline{w}_{t}-w^{\ast},\beta_{t}(\overline{y}_{t+1}-\overline{y}_{t})-\alpha_{t}\overline{g}_{t}\rangle+\|\beta_{t}(\overline{y}_{t+1}-\overline{y}_{t})-\alpha_{t}\overline{g}_{t}\|^{2}\\
 & \leq\|\overline{w}_{t}-w^{\ast}\|^{2}+2\langle\overline{w}_{t}-w^{\ast},\beta_{t}(\overline{y}_{t+1}-\overline{y}_{t})-\alpha_{t}\overline{g}_{t}\rangle+2\|\beta_{t}(\overline{y}_{t+1}-\overline{y}_{t})\|^{2}+2\|\alpha_{t}\overline{g}_{t}\|^{2}
\end{align*}
 and by the convexity of $\|\cdot\|^{2}$ and $L$-smoothness of $F_{k}$,
\begin{align*}
\alpha_{t}^{2}\|\overline{g}_{t}\|^{2} & \leq\alpha_{t}^{2}\sum_{k=1}^{N}p_{k}\|\nabla F_{k}(w_{t}^{k})\|^{2}\leq2L\alpha_{t}^{2}\sum_{k=1}^{N}p_{k}(F_{k}(w_{t}^{k})-F_{k}^{\ast})
\end{align*}
 and if $\beta_{t}=\alpha_{t}$,
\begin{align*}
2\|\beta_{t}(\overline{y}_{t+1}-\overline{y}_{t})\|^{2} & =2\beta_{t}^{2}\|\sum_{k=1}^{N}p_{k}(y_{t+1}^{k}-y_{t}^{k})\|^{2}\\
 & \leq2\beta_{t}^{2}\sum_{k=1}^{N}p_{k}\|y_{t+1}^{k}-y_{t}^{k}\|^{2}\\
 & =2\alpha_{t}^{2}\sum_{k=1}^{N}p_{k}\|y_{t+1}^{k}-y_{t}^{k}\|^{2}
\end{align*}
and taking expectation we get 
\begin{align*}
2\mathbb{E}\|\beta_{t}(\overline{y}_{t+1}-\overline{y}_{t})\|^{2} & \leq2\alpha_{t}^{2}G^{2}
\end{align*}
 Now 
\begin{align*}
2\mathbb{E}\langle\overline{w}_{t}-w^{\ast},\beta_{t}(\overline{y}_{t+1}-\overline{y}_{t})-\alpha_{t}\overline{g}_{t}\rangle & =2\beta_{t}\mathbb{E}\langle\overline{w}_{t}-w^{\ast},(\overline{y}_{t+1}-\overline{y}_{t})\rangle-2\alpha_{t}\langle\overline{w}_{t}-w^{\ast},\overline{g}_{t}\rangle
\end{align*}
 and so 
\begin{align*}
\mathbb{E}\|\overline{w}_{t+1}-w^{\ast}\|^{2} & \leq\mathbb{E}\|\overline{w}_{t}-w^{\ast}\|^{2}-2\alpha_{t}\langle\overline{w}_{t}-w^{\ast},\overline{g}_{t}\rangle+4L\alpha_{t}^{2}\sum_{k=1}^{N}p_{k}(F_{k}(w_{t}^{k})-F_{k}^{\ast})+\alpha_{t}^{2}\mathbb{E}\|g_{t}-\overline{g}_{t}\|^{2}\\
 & +2\alpha_{t}^{2}G^{2}+2\beta_{t}\mathbb{E}\langle\overline{w}_{t}-w^{\ast},(\overline{y}_{t+1}-\overline{y}_{t})\rangle
\end{align*}
 At this point, the exact same argument in the ICLR paper implies
that 
\begin{align*}
\mathbb{E}\|\overline{w}_{t+1}-w^{\ast}\|^{2} & \leq(1-\mu\alpha_{t})+9L\alpha_{t}^{2}\Gamma+\alpha_{t}^{2}\mathbb{E}\|g_{t}-\overline{g}_{t}\|^{2}+2\mathbb{E}\sum_{k=1}^{N}p_{k}\|\overline{w}_{t}-w_{k}^{t}\|^{2}\\
 & +2\alpha_{t}^{2}G^{2}+2\beta_{t}\mathbb{E}\langle\overline{w}_{t}-w^{\ast},(\overline{y}_{t+1}-\overline{y}_{t})\rangle
\end{align*}

Now we bound $\mathbb{E}\sum_{k=1}^{N}p_{k}\|\overline{w}_{t}-w_{t}^{k}\|^{2}$.
Since communication is done every $E$ steps, for any $t\geq0$, we
can find a $t_{0}\leq t$ such that $t-t_{0}\leq E-1$ and $w_{t_{0}}^{k}=\overline{w}_{t_{0}}$for
all $k$. Moreover, using $\eta_{t}$ is non-increasing and $\eta_{t_{0}}\leq2\eta_{t}$
for any $t-t_{0}\leq E-1$, we have 
\begin{align*}
\mathbb{E}\sum_{k=1}^{N}p_{k}\|\overline{w}_{t}-w_{t}^{k}\|^{2} & =\mathbb{E}\sum_{k=1}^{N}p_{k}\|w_{t}^{k}-\overline{w}_{t_{0}}-(\overline{w}_{t}-\overline{w}_{t_{0}})\|^{2}\\
 & \leq\mathbb{E}\sum_{k=1}^{N}p_{k}\|w_{t}^{k}-\overline{w}_{t_{0}}\|^{2}\\
 & =\mathbb{E}\sum_{k=1}^{N}p_{k}\|w_{t}^{k}-w_{t_{0}}^{k}\|^{2}\\
 & =\mathbb{E}\sum_{k=1}^{N}p_{k}\|\sum_{i=t_{0}}^{t-1}\beta_{i}(y_{i+1}^{k}-y_{i}^{k})-\sum_{i=t_{0}}^{t-1}\alpha_{i}g_{i,k}\|^{2}\\
 & \leq2\sum_{k=1}^{N}p_{k}\mathbb{E}\sum_{i=t_{0}}^{t-1}(E-1)\alpha_{i}^{2}\|g_{i,k}\|^{2}+2\sum_{k=1}^{N}p_{k}\mathbb{E}\sum_{i=t_{0}}^{t-1}(E-1)\beta_{i}^{2}\|(y_{i+1}^{k}-y_{i}^{k})\|^{2}
\end{align*}
where we recall that 

\begin{align*}
y_{t+1}^{k} & =w_{t}^{k}-\alpha_{t}g_{t,k}\\
w_{t+1}^{k} & =\begin{cases}
y_{t+1}^{k}+\beta_{t}(y_{t+1}^{k}-y_{t}^{k}) & \text{if }t+1\notin\mathcal{I}_{E}\\
\sum_{k=1}^{N}p_{k}\left[y_{t+1}^{k}+\beta_{t}(y_{t+1}^{k}-y_{t}^{k})\right] & \text{if }t+1\in\mathcal{I}_{E}
\end{cases}
\end{align*}
 The first term $2\sum_{k=1}^{N}p_{k}\mathbb{E}\sum_{i=t_{0}}^{t-1}(E-1)\alpha_{i}^{2}\|g_{i,k}\|^{2}$
is bounded above by $8\alpha_{t}^{2}(E-1)^{2}G^{2}$ following the
ICLR paper. The term $\mathbb{E}\|(y_{i+1}^{k}-y_{i}^{k})\|^{2}$
is bounded above by $G^{2}$ as well, as proved earlier. It follows
that 
\begin{align*}
\mathbb{E}\sum_{k=1}^{N}p_{k}\|\overline{w}_{t}-w_{t}^{k}\|^{2} & \leq16\alpha_{t}^{2}(E-1)^{2}G^{2}
\end{align*}

Using the bound on $\mathbb{E}\sum_{k=1}^{N}p_{k}\|\overline{w}_{t}-w_{t}^{k}\|^{2}$,
we can conclude that 

\begin{align*}
\mathbb{E}\|\overline{w}_{t+1}-w^{\ast}\|^{2} & \leq(1-\mu\alpha_{t})\mathbb{E}\|\overline{w}_{t}-w^{\ast}\|^{2}+9L\alpha_{t}^{2}\Gamma+\alpha_{t}^{2}\sum_{k=1}^{N}p_{k}^{2}\sigma_{k}^{2}+32\alpha_{t}^{2}(E-1)^{2}G^{2}\\
 & +2\alpha_{t}^{2}G^{2}+2\beta_{t}\mathbb{E}\langle\overline{w}_{t}-w^{\ast},(\overline{y}_{t+1}-\overline{y}_{t})\rangle\\
 & =(1-\mu\alpha_{t})\mathbb{E}\|\overline{w}_{t}-w^{\ast}\|^{2}+\alpha_{t}^{2}B+2\beta_{t}\mathbb{E}\langle\overline{w}_{t}-w^{\ast},(\overline{y}_{t+1}-\overline{y}_{t})\rangle
\end{align*}
 where 
\begin{align*}
B & =\sum_{k=1}^{N}p_{k}^{2}\sigma_{k}^{2}+9L\Gamma+32(E-1)^{2}G^{2}+2
\end{align*}
 Our next step is to show that $2\beta_{t}\mathbb{E}\langle\overline{w}_{t}-w^{\ast},(\overline{y}_{t+1}-\overline{y}_{t})\rangle=O(\alpha_{t}^{2})$. 

With appropriate choice of constant $K$ depending on the other constants(to
be detailed), we first show that 
\begin{align*}
\mathbb{E}\|\overline{w}_{t+1}-w^{\ast}\|^{2} & \leq K^{2}
\end{align*}
 for all $t$, i.e. the updates always stay in a large ball around
the optimum during the Nesterov accelerated gradient descent. Note
that
\begin{align*}
\beta_{t}\mathbb{E}\langle\overline{w}_{t}-w^{\ast},(\overline{y}_{t+1}-\overline{y}_{t})\rangle & \leq\beta_{t}\sqrt{\mathbb{E}\|\overline{w}_{t}-w^{\ast}\|^{2}}\cdot\sqrt{\mathbb{E}\|\overline{y}_{t+1}-\overline{y}_{t}\|^{2}}
\end{align*}
 so that 
\begin{align*}
\mathbb{E}\|\overline{w}_{t+1}-w^{\ast}\|^{2} & \leq(1-\alpha_{t}\mu)\mathbb{E}\|\overline{w}_{t}-w^{\ast}\|^{2}+\alpha_{t}^{2}B+2\beta_{t}\sqrt{\mathbb{E}\|\overline{w}_{t}-w^{\ast}\|^{2}}\cdot\sqrt{\mathbb{E}\|\overline{y}_{t+1}-\overline{y}_{t}\|^{2}}\\
 & \leq(1-\alpha_{t}\mu)\mathbb{E}\|\overline{w}_{t}-w^{\ast}\|^{2}+\alpha_{t}^{2}B+2\beta_{t}\sqrt{\mathbb{E}\|\overline{w}_{t}-w^{\ast}\|^{2}}\cdot G
\end{align*}
 where $B=\sum_{k=1}^{N}p_{k}^{2}\sigma_{k}^{2}+9L\Gamma+32(E-1)^{2}G^{2}+2$.
Suppose $\mathbb{E}\|\overline{w}_{t}-w^{\ast}\|^{2}\leq K^{2}$ for
$t\geq0$, then 
\begin{align*}
\mathbb{E}\|\overline{w}_{t+1}-w^{\ast}\|^{2} & \leq(1-\alpha_{t}\mu)K^{2}+\alpha_{t}^{2}B+2\beta_{t}K\cdot G\\
 & \leq K^{2}+(\alpha_{t}^{2}B+2\alpha_{t}K\cdot G-\alpha_{t}\mu K^{2})
\end{align*}
 as long as $\alpha_{0}$ and $K$ are chosen so that 
\begin{align*}
\alpha_{0}B+2\sqrt{K}\cdot G & \leq\mu K
\end{align*}
 and
\begin{align*}
\|w_{0}-w^{\ast}\|^{2} & \leq K
\end{align*}
 then since $\alpha_{t}\leq\alpha_{0}$, we get $\mathbb{E}\|\overline{w}_{t+1}-w^{\ast}\|^{2}\leq K$,
where $K$ only depends on $G,\sigma_{k},L,\mu,\Gamma,\|w_{0}-w^{\ast}\|^{2},E$. 

Now we can finally bound $\beta_{t}\mathbb{E}\langle\overline{w}_{t}-w^{\ast},(\overline{y}_{t+1}-\overline{y}_{t})\rangle$.

Using the recursive relations 
\begin{align*}
y_{t+1}^{k}-y_{t}^{k} & =w_{t}^{k}-w_{t-1}^{k}-(\alpha_{t}g_{t,k}-\alpha_{t-1}g_{t-1,k})\\
w_{t+1}^{k}-w_{t}^{k} & =-\alpha_{t}g_{t,k}+\beta_{t}(y_{t+1}^{k}-y_{t}^{k})
\end{align*}
 so that $y_{t+1}^{k}-y_{t}^{k}=\beta_{t-1}(y_{t}^{k}-y_{t-1}^{k})-\alpha_{t}g_{t,k}$,
we have 
\begin{align*}
\overline{y}_{t+1}-\overline{y}_{t} & =\beta_{t-1}(\overline{y}_{t}-\overline{y}_{t-1})-\alpha_{t}g_{t}
\end{align*}
 and so 
\begin{align*}
\beta_{t}\langle\overline{w}_{t}-w^{\ast},(\overline{y}_{t+1}-\overline{y}_{t})\rangle & =\beta_{t}\langle\overline{w}_{t}-w^{\ast},\beta_{t-1}(\overline{y}_{t}-\overline{y}_{t-1})-\alpha_{t}g_{t,k}\rangle\\
 & =\beta_{t}\langle\overline{w}_{t}-w^{\ast},\beta_{t-1}(\overline{y}_{t}-\overline{y}_{t-1})\rangle-\beta_{t}\langle\overline{w}_{t}-w^{\ast},\alpha_{t}g_{t}\rangle
\end{align*}
 and we further expand the first term: 
\begin{align*}
\beta_{t}\langle\overline{w}_{t}-w^{\ast},\beta_{t-1}(\overline{y}_{t}-\overline{y}_{t-1})\rangle & =\beta_{t}\langle\overline{w}_{t}-\overline{w}_{t-1}+\overline{w}_{t-1}-w^{\ast},\beta_{t-1}(\overline{y}_{t}-\overline{y}_{t-1})\rangle\\
 & =\beta_{t}\langle\overline{w}_{t}-\overline{w}_{t-1},\beta_{t-1}(\overline{y}_{t}-\overline{y}_{t-1})\rangle+\beta_{t}\langle\overline{w}_{t-1}-w^{\ast},\beta_{t-1}(\overline{y}_{t}-\overline{y}_{t-1})\rangle\\
 & =\beta_{t}\beta_{t-1}\langle-\alpha_{t-1}g_{t-1}+\beta_{t-1}(\overline{y}_{t}-\overline{y}_{t-1}),(\overline{y}_{t}-\overline{y}_{t-1})\rangle+\beta_{t}\langle\overline{w}_{t-1}-w^{\ast},\beta_{t-1}(\overline{y}_{t}-\overline{y}_{t-1})\rangle\\
 & =\beta_{t}\beta_{t-1}\langle-\alpha_{t-1}g_{t-1},(\overline{y}_{t}-\overline{y}_{t-1})\rangle+\beta_{t}\beta_{t-1}^{2}\|\overline{y}_{t}-\overline{y}_{t-1}\|^{2}+\beta_{t}\beta_{t-1}\langle\overline{w}_{t-1}-w^{\ast},(\overline{y}_{t}-\overline{y}_{t-1})\rangle
\end{align*}
 and so 
\begin{align*}
\langle\overline{w}_{t}-w^{\ast},(\overline{y}_{t+1}-\overline{y}_{t})\rangle & =-\beta_{t-1}\alpha_{t-1}\langle g_{t-1},\overline{y}_{t}-\overline{y}_{t-1}\rangle+\beta_{t-1}^{2}\|\overline{y}_{t}-\overline{y}_{t-1}\|^{2}+\beta_{t-1}\langle\overline{w}_{t-1}-w^{\ast},(\overline{y}_{t}-\overline{y}_{t-1})\rangle-\alpha_{t}\langle\overline{w}_{t}-w^{\ast},g_{t}\rangle
\end{align*}
 from which we can conclude that $|\beta_{t}\langle\overline{w}_{t}-w^{\ast},(\overline{y}_{t+1}-\overline{y}_{t})\rangle|\leq\alpha_{t}^{2}(G^{2}+GK)$
and so 
\begin{align*}
\mathbb{E}\|\overline{w}_{t+1}-w^{\ast}\|^{2} & \leq(1-\mu\alpha_{t})\mathbb{E}\|\overline{w}_{t}-w^{\ast}\|^{2}+9L\alpha_{t}^{2}\Gamma+\alpha_{t}^{2}\sum_{k=1}^{N}p_{k}^{2}\sigma_{k}^{2}+32\alpha_{t}^{2}(E-1)^{2}G^{2}\\
 & +2\alpha_{t}^{2}G^{2}+2\beta_{t}\mathbb{E}\langle\overline{w}_{t}-w^{\ast},(\overline{y}_{t+1}-\overline{y}_{t})\rangle\\
 & =(1-\mu\alpha_{t})\mathbb{E}\|\overline{w}_{t}-w^{\ast}\|^{2}+\alpha_{t}^{2}B'
\end{align*}
 where 
\begin{align*}
B' & =B+G^{2}+GK\\
 & =\sum_{k=1}^{N}p_{k}^{2}\sigma_{k}^{2}+9L\Gamma+32(E-1)^{2}G^{2}+2+G^{2}+GK
\end{align*}
 and the rest of the proof follows as ICLR paper. 
\end{proof}



\section{Proof for Convergence Results on Convex and Smooth Objectives}

\subsection{SGD}
\label{sec:convexsmoothsgd}
% % !TEX ROOT=./main.tex


\textbf{Full Device Participation}

\begin{itemize}
	\item Stochastic gradient of device $k$ at time step $t$, at point $w_{t,k}$: 
	$$\vg_{t,k} \coloneqq \vg_{t,k}(w_t^k)$$
	$$ \vg_{t, k} = \grad F_{k}\left(w_{t}^{k}, \xi_{t}^{k}\right) $$
	$$\EE \vg_{t, k} = \grad F_{k}\left(w_{t}^{k}\right)$$
\item  One-step stochastic subgradient of all devices.
$$\vg_{t}=\sum_{k=1}^{N} p_{k} \vg_{t, k}\left(w_{t}^{k}\right) $$
\begin{align}
	\EE \vg_{t}= \EE \sum_{k=1}^{N} p_{k} \vg_{t, k}\left(w_{t}^{k}\right) \coloneqq \sum_{k=1}^{N} p_{k} \EE \vg_{t, k}
	\label{eq:egtsgd}
\end{align}
\end{itemize}



$$\Delta_{t+1} = \EE \|\overline{\vw}_{t+1} - \vw^*\|^2 = \EE \|\overline{\vv}_{t+1} - \vw^*\|^2,$$
According to the definition of $\overline{\vv}_{t+1}$ in \eq{\ref{eq:vbar}}, we can expand $\Delta_{t+1}$
$$\begin{aligned}\left\|\bar{\vv}_{t+1}-\vw^{*}\right\|^2 &=\left\|\ov{w}_{t}-\eta_{t} \vg_{t}-\vw^{*}\right\|^2 \\ &=\left\|\ov{w}_{t}-\vw^{*}\right\|^{2}-2 \eta_{t} \vg_{t}^{\top}\left(\ov{w}_{t}-\vw^{*}\right)+\eta_{t}^{2}\left\|\vg_{t}\right\|^{2} \end{aligned}$$

Take the expectation condition on $\vw_t$ over random samples at all devices:
\begin{align}
\Delta_{t+1} = \EE\left[\left\|\bar{\vv}_{t+1}-\vw^{*}\right\|^2| \vw_{t}\right]=\|\ov{w}_{t}-\vw^{*}\|^{2}-2 \eta_{t} \EE \vg_{t}^{\top}\left(\ov{w}_{t}-\vw^{*}\right)+\eta_{t}^{2} \EE\| \vg_{t} \|^{2}	
\label{eq:expandsgd}
\end{align}

Now we focus on bounding $-2 \eta_{t} \EE \vg_{t}^{\top}\left(\ov{w}_{t}-\vw^{*}\right)$ in \eq{\ref{eq:expandsgd}}: 

\begin{align*}
	& -2 \eta_{t} \left<\EE \vg_{t}, \ov{w}_{t}-\vw^{*}\right> \\
 =  & -2 \eta_{t} \left<\EE \vg_{t}, \ov{w}_{t}- \vw^{k}_t \right> -2 \eta_{t} \left<\EE \vg_{t}, \vw^{k}_t - \vw^{*}\right>\\
 \leq & -2 \eta_{t} \left<\EE \vg_{t}, \ov{w}_{t}- \vw^{k}_t \right> + 2 \eta_{t} (F_k(w^*) - F_k(\vw^{k}_t))\\
 \leq &\sum_{k=1}^N p_k \left[2 \eta_{t} (F_k(\vw^{k}_t) - F_k(\ov{w}_t) + \frac{L}{2} \|\ov{w}_t - \vw^{k}_t\|^2 ) + 2 \eta_{t} (F_k(w^*) - F_k(\vw^{k}_t))\right]\\
 = & \sum_{k=1}^N p_k \eta_t L \|\ov{w}_t - \vw^{k}_t\|^2 + 2 \eta_{t} \sum_{k=1}^N p_k (F_k(w^*) - F_k(\ov{w}_t))\\
 = &  \eta_t L \sum_{k=1}^N p_k \|\ov{w}_t - \vw^{k}_t\|^2 + 2 \eta_{t} (F^* - F(\ov{w}_t))
\end{align*}
Plug in this upper bound into \eq{\ref{eq:expandsgd}}, we have

\begin{align}
\EE\left\|\ov{w}_{t+1}-\vw^{*}\right\|^2 &\leq \|\ov{w}_{t}-\vw^{*}\|^{2}+\eta_{t}^{2} \EE\| \vg_{t} \|^{2} + \eta_t L \sum_{k=1}^N p_k \|\ov{w}_t - \vw^{k}_t\|^2 + 2 \eta_{t} (F^* - F(\ov{w}_t))\\
&\leq \|\ov{w}_{t}-\vw^{*}\|^{2}+\eta_{t}^{2} \EE\| \vg_{t} \|^{2} +  4L\eta_t^3(E-1)^2 G^2 + 2 \eta_{t} (F^* - F(\ov{w}_t))\\
&\leq \|\ov{w}_{t}-\vw^{*}\|^{2}+\eta_{t}^{2} G^2 +  4L\eta_t^3(E-1)^2 G^2 + 2 \eta_{t} (F^* - F(\ov{w}_t)) \label{eq:cvxsgd1}
\end{align}

Sum two sides of \eq{\ref{eq:cvxsgd1}}, we assume \lkxcom{$\eta_t < 1$}, set $F^*_t = \min_{t \in [0, T-1]} F(\ov{w}_t)$, 
\begin{align*}
	\sum_{t=0}^{T-1} 2\eta_t (F(\ov{w}_t) - F^* ) & \leq \EE \|\ov{w}_{0}-\vw^{*}\|^{2} - \EE\left\|\ov{w}_{T}-\vw^{*}\right\|^2 + \sum_{t=0}^{T-1} (\eta_t^3 4LG^2(E-1)^2 + \eta_t^2 G^2)\\ 
    & \leq \EE \|\ov{w}_{0}-\vw^{*}\|^{2} - \EE\left\|\ov{w}_{T}-\vw^{*}\right\|^2 + \sum_{t=0}^{T-1} (\eta_t^3 4LG^2(E-1)^2 + \eta_t^2 G^2)\\ 
    & \leq \EE \|\ov{w}_{0}-\vw^{*}\|^{2} + \sum_{t=0}^{T-1}\eta_t^2G^2 ( 4L(E-1)^2 + 1)\\
 F_t^* - F^*  & \leq \frac{\EE \|\ov{w}_{0}-\vw^{*}\|^{2}}{2 \sum_{t=0}^{T-1} \eta_t } + ( 4L(E-1)^2 + 1)G^2 \sum_{t=0}^{T-1} \eta_t^2
\end{align*}
It will converge under three conditions $ \lim_{T \rightarrow \infty }\sum_{t=0}^{T-1} \eta_t = \infty$
$ \lim_{T \rightarrow \infty }\sum_{t=0}^{T-1} \eta_t^2 < \infty$. For example, we can set $\eta_t = \frac{1}{t+1}$.


	
% !TEX ROOT=./main.tex
Stochastic gradient of device $k$ at time step $t$, at point $\vw_t^k$: 
	$\vg_{t,k} \coloneqq \vg_{t,k}(w_t^k)$,
	$ \vg_{t, k} = \grad F_{k}\left(w_{t}^{k}, \xi_{t}^{k}\right) $,
	$\EE \vg_{t, k} = \grad F_{k}\left(w_{t}^{k}\right)$.
One-step stochastic gradient of all devices:
$\vg_{t}=\sum_{k=1}^{N} p_{k} \vg_{t, k}\left(w_{t}^{k}\right) $, $\EE \vg_{t}= \EE \sum_{k=1}^{N} p_{k} \vg_{t, k}\left(w_{t}^{k}\right) \coloneqq \sum_{k=1}^{N} p_{k} \EE \vg_{t, k} = \ov{g}_t$.
We further denote the following constants $\nu_{max}:=N\cdot\max_{k}p_{k}$ and $\nu_{min}:=N\cdot\min_{k}p_{k}$. 

% Let $\widetilde{F}_{k}(\mathbf{w})=p_{k} N F_{k}(\mathbf{w})$, Then the global objective change to the 
% average of all scaled local objectives. 
% $F(\mathbf{w})=\sum_{k=1}^{N} p_{k} F_{k}(\mathbf{w})=\frac{1}{N} \sum_{k=1}^{N} \widetilde{F}_{k}(\mathbf{w})$
% The constants were replaced as follows:
% $\widetilde{L} \triangleq \nu L, \widetilde{\mu} \triangleq_{S \mu,} \widetilde{\sigma}_{k}=\sqrt{\nu} \sigma,$ and $\widetilde{G}=\sqrt{\nu} G$, where $\nu=N \cdot \max _{k} p_{k}$ and $s=N \cdot \min _{k} p_{k}$

\begin{lemma}
Under Assumption~\ref{ass:subgrad2}, we have the following bound
\begin{align}
	 \|\EE \vg_{t,k}\|^2 & \leq G_k^2	\label{eq:g3} \\
	\|\EE \vg_{t,k}(\overline{\vw}_t)\|^2 & \leq  G_k^2 \label{eq:g2} \\
   \EE\| \vg_{t} \|^{2} &  \leq  \sum_{k=1}^N p_k G_k^2  \label{eq:g1}
\end{align}
\label{lma:gradient}
\end{lemma}
The first two inequalities directly follows from Assumption~\ref{ass:subgrad2} and the inequality
can be obtained by applying the convexity of the l2 norm and Jensen's inequality. 

\begin{lemma}
	With unbiased device sampling scheme, we have the following bound: 
	$$ \EE_{\cS_{t+1}, \xi_{t}} \|\overline{\vw}_{t+1} - \vw^*\|^2 \leq \eta_t^2 C + \EE_{\xi_t} \|\overline{\vv}_{t+1} - \vw^*\|^2 $$
\label{lma:wdistance}
\end{lemma}

\begin{proof}
\begin{align*}
& \EE_{\cS_{t+1}, \xi_{t}} \|\overline{\vw}_{t+1} - \vw^*\|^2 \\
=& \EE_{\cS_{t+1}, \xi_{t}} \|\overline{\vw}_{t+1} - \overline{\vv}_{t+1} + \overline{\vv}_{t+1} - \vw^*\|^2\\
=& \EE_{\cS_{t+1}, \xi_{t}} \left[\|\overline{\vw}_{t+1} - \overline{\vv}_{t+1}\|^2 + \|\overline{\vv}_{t+1} - \vw^*\|^2\right] + 2\EE_{\xi_{t}} \left<\EE_{\cS_{t+1}} \overline{\vw}_{t+1} - \overline{\vv}_{t+1},   \overline{\vv}_{t+1} - \vw^*\right> \\
=& \EE_{\cS_{t+1}, \xi_{t}} \|\overline{\vw}_{t+1} - \overline{\vv}_{t+1}\|^2 + \EE_{\xi_t} \|\overline{\vv}_{t+1} - \vw^*\|^2 \\
\leq&  \eta_t^2 C + \EE_{\xi_t} \|\overline{\vv}_{t+1} - \vw^*\|^2 %\label{eq:sgdcvxsmth1}
\end{align*}
where in the second line we use $\EE_{\cS_{t+1}} \overline{\vw}_{t+1}  = \overline{\vv}_{t+1}$. 
The first term can be bounded by consider different sampling scheme as in \eq{\ref{eq:partialsample}}.
\end{proof}

\begin{lemma}[Lemma 3~\cite{li2019convergence}]
Let $\eta_t$ and $E$ satisfy $\eta_0 \leq 2 \eta_t$ for all $t- t_0 \leq E - 1$, we have
$$\mathbb{E}\left[\sum_{k=1}^{N} p_{k}\left\|\overline{\mathbf{w}}_{t}-\mathbf{w}_{k}^{t}\right\|^{2}\right] \leq \sum_{k=1}^N p_k4 \eta_{t}^{2}(E-1)^{2} G_k^{2}$$
\label{lma:l3iclr}
\end{lemma}

\begin{lemma}[Lemma 2~\cite{li2019convergence}]
$\mathbb{E}\left\|\mathbf{g}_{t}-\overline{\mathbf{g}}_{t}\right\|^{2} \leq \sum_{k=1}^{N} p_{k}^{2}\sigma_k^2 \leq \frac{1}{N}\nu_{\max}^2\sigma^2$
\label{lma:iclrvar}
\end{lemma}
\begin{proof}
	\begin{align} 
\mathbb{E}\left\|\mathbf{g}_{t}-\overline{\mathbf{g}}_{t}\right\|^{2} &=\mathbb{E}\left\|\sum_{k=1}^{N} p_{k}\left(\nabla F_{k}\left(\mathbf{w}_{t}^{k}, \xi_{t}^{k}\right)-\nabla F_{k}\left(\mathbf{w}_{t}^{k}\right)\right)\right\|^{2} \\ &=\sum_{k=1}^{N} p_{k}^{2} \mathbb{E}\left\|\nabla F_{k}\left(\mathbf{w}_{t}^{k}, \xi_{t}^{k}\right)-\nabla F_{k}\left(\mathbf{w}_{t}^{k}\right)\right\|^{2} \\ & \leq \sum_{k=1}^{N} p_{k}^{2} \sigma_{k}^{2} \end{align}
\end{proof}

\begin{lemma}
Under Assumption~\ref{ass:lsmooth}, we have the following bound on the 
optimality gap.
$$\eta_t \sum_{k=1}^{N}p_{k}\left[F_{k}(w^{\ast})-F_{k}(\overline{w}_{t})\right] \leq-\eta_{t}^{2}\|\overline{g}_{t}\|^{2}+\eta_{t}^{2}L^{2}\sum_{k}p_{k}\|\overline{w}_{t}-w_{t}^{k}\|^{2}+\eta_{t}^{3}L\mathbb{E}\|g_{t}\|^{2} $$
where $\eta_t$ denotes learning rate.
\label{lma:optgap}
\end{lemma}
\begin{proof}
Note that the expectation in here is taken w.r.t. $\xi_{t+1}^k$ for all $k$.
	\begin{align*}
	2\eta_{t}\EE\sum_{k=1}^{N}p_{k}\left[F_{k}(\vw^{\ast})-F_{k}(\overline{\vw}_{t})\right] & \leq 
	2\eta_{t}\EE \left[F(\overline{\vw}_{t+1})-F(\overline{\vw}_{t})\right]\\
	& \leq2\eta_{t}\mathbb{E}\langle\nabla F(\overline{\vw}_{t}),\overline{\vw}_{t+1}-\overline{\vw}_{t}\rangle+\eta_{t}L\mathbb{E}\|\overline{\vw}_{t+1}-\overline{\vw}_{t}\|^{2}\\
	% & =-2\eta_{t}^{2}\mathbb{E}\langle\nabla F(\overline{\vw}_{t}),\vg_{t}\rangle+\eta_{t}^{3}L\mathbb{E}\|g_{t}\|^{2}\\
	& =-2\eta_{t}^{2}\mathbb{E}\langle\nabla F(\overline{\vw}_{t}),\overline{\vg}_{t}\rangle+\eta_{t}^{3}L\mathbb{E}\|\vg_{t}\|^{2}\\
	& =-\eta_{t}^{2}\left[\|\nabla F(\overline{\vw}_{t})\|^{2}+\|\overline{\vg}_{t}\|^{2}-\|\nabla F(\overline{\vw}_{t}) - \overline{\vg}_{t}\|^{2}\right]+\eta_{t}^{3}L\mathbb{E}\|\vg_{t}\|^{2}\\
	& =-\eta_{t}^{2}\left[\|\nabla F(\overline{\vw}_{t})\|^{2}+\|\overline{\vg}_{t}\|^{2}-\|\nabla F(\overline{\vw}_{t})-\sum_{k}p_{k}\nabla F(\vw_{t}^{k})\|^{2}\right]+\eta_{t}^{3}L\mathbb{E}\|\vg_{t}\|^{2}\\
	& \leq-\eta_{t}^{2}\left[\|\nabla F(\overline{\vw}_{t})\|^{2}+\|\overline{\vg}_{t}\|^{2}-\sum_{k}p_{k}\|\nabla F(\overline{\vw}_{t})-\nabla F(\vw_{t}^{k})\|^{2}\right]+\eta_{t}^{3}L\mathbb{E}\|\vg_{t}\|^{2}\\
	& \leq-\eta_{t}^{2}\left[\|\nabla F(\overline{\vw}_{t})\|^{2}+\|\overline{\vg}_{t}\|^{2}-L^{2}\sum_{k}p_{k}\|\overline{\vw}_{t}-\vw_{t}^{k}\|^{2}\right]+\eta_{t}^{3}L\mathbb{E}\|\vg_{t}\|^{2}\\
	& \leq-\eta_{t}^{2}\|\overline{\vg}_{t}\|^{2}+\eta_{t}^{2}L^{2}\sum_{k}p_{k}\|\overline{\vw}_{t}-\vw_{t}^{k}\|^{2}+\eta_{t}^{3}L\mathbb{E}\|\vg_{t}\|^{2}
	\end{align*}
\end{proof}

\begin{theorem}
Let Assumption~\ref{ass:subgrad2} and Assumption~\ref{ass:lsmooth} hold, suppose we have a bound 
on our starting distance, i.e., $\|\vw_{0} - \vw^*\|^2 \leq \Delta_0$, set learning rate $\eta_t =  \left(\frac{\Delta_0}{ T [G^2( 4L(E-1)^2 + 1) + C]}\right)^{1/2}$, we have,
$$\EE[ F_t^*] - F^*  \leq \left(\frac{ \Delta_0 [G^2( 4L(E-1)^2 + 1) + C] }{T}\right)^{1/2}$$
where we denote $F^*_t = \min_{t \in [0, T-1]} F(\ov{w}_t)$.
% \label{th:sgdcvxsmth}
\end{theorem}

\begin{proof}
	

With Lemma~\ref{lma:wdistance}, 
\begin{align}
	\EE_{\cS_{t+1}, \xi_{t}} \|\overline{\vw}_{t+1} - \vw^*\|^2 \leq \eta_t^2 C + \EE_{\xi_t} \|\overline{\vv}_{t+1} - \vw^*\|^2\label{eq:sgdcvxsmth1}
\end{align}
and according to the definition of $\overline{\vv}_{t+1}$ in \eq{\ref{eq:vbar}}, we can expand the second term in \eq{\ref{eq:sgdcvxsmth1}} as follows:
\begin{align*}
\left\|\ov{v}_{t+1}-\vw^{*}\right\|^2 
 &=\left\|\ov{w}_{t}-\eta_{t} \vg_{t}-\vw^{*}\right\|^2 \\
 &=\left\|\ov{w}_{t}-\eta_{t} \vg_{t}-\vw^{*} - \eta_t \ov{g}_t + \eta_t \ov{g}_t\right\|^2 
\\
& = \left\|\ov{w}_{t}- \eta_t \ov{g}_t  -\vw^{*} + \eta_t \ov{g}_t - \eta_{t} \vg_{t}\right\|^2 \\
& = \left\|\ov{w}_{t}- \eta_t \ov{g}_t  -\vw^{*}\right\|^2  + 2\eta_t \left<\vw_t - \vw^* - \eta_t \ov{g}_t, \ov{g}_t - \vg_{t} \right> + \left\|\eta_t \ov{g}_t - \eta_{t} \vg_{t}\right\|^2 \\
 &=\left\|\ov{w}_{t}-\vw^{*}\right\|^{2}-2 \eta_{t} \left<\ov{g}_t, \ov{w}_{t}-\vw^{*} \right>+\eta_{t}^{2}\left\|\ov{g}_{t}\right\|^{2}  + \underbrace{2\eta_t \left<\vw_t - \vw^* - \eta_t \ov{g}_t, \ov{g}_t - \vg_{t} \right>}_{A_1} + \eta_{t}^2\left\| \ov{g}_t -  \vg_{t}\right\|^2
\end{align*}

We take the expectation condition on $\vw_t$ over $\xi_t$, i.e., random samples at all devices and
note that $\EE A_1 = 0$:
\begin{align}
\EE\left[\left\|\ov{v}_{t+1}-\vw^{*}\right\|^2| \vw_{t}\right]=\|\ov{w}_{t}-\vw^{*}\|^{2}-2 \eta_{t} \left<\ov{g}_t, \ov{w}_{t}-\vw^{*} \right> +\eta_{t}^{2} \| \ov{g}_{t} \|^{2} + \EE \eta_{t}^2\left\| \ov{g}_t -  \vg_{t}\right\|^2
\label{eq:expandsgd}
\end{align}

Now we focus on bounding $-2 \eta_{t} \left< \EE \vg_{t}, \ov{w}_{t}-\vw^{*}\right>$ in \eq{\ref{eq:expandsgd}}: 

\begin{align*}
	& -2 \eta_{t} \left<\EE \vg_{t}, \ov{w}_{t}-\vw^{*}\right> \\
 =  & -2 \eta_{t} \left<\EE \vg_{t}, \ov{w}_{t}- \vw^{k}_t \right> -2 \eta_{t} \left<\EE \vg_{t}, \vw^{k}_t - \vw^{*}\right>\\
 \leq & -2 \eta_{t} \left<\EE \vg_{t}, \ov{w}_{t}- \vw^{k}_t \right> + 2 \eta_{t} (F_k(w^*) - F_k(\vw^{k}_t))\\
 \leq &\sum_{k=1}^N p_k \left[2 \eta_{t} (F_k(\vw^{k}_t) - F_k(\ov{w}_t) + \frac{L}{2} \|\ov{w}_t - \vw^{k}_t\|^2 ) + 2 \eta_{t} (F_k(w^*) - F_k(\vw^{k}_t))\right]\\
 = & \sum_{k=1}^N p_k \eta_t L \|\ov{w}_t - \vw^{k}_t\|^2 + 2 \eta_{t} \sum_{k=1}^N p_k (F_k(w^*) - F_k(\ov{w}_t))\\
 = &  \eta_t L \sum_{k=1}^N p_k \|\ov{w}_t - \vw^{k}_t\|^2 + 2 \eta_{t} (F^* - F(\ov{w}_t))
\end{align*}
Plug in this upper bound into \eq{\ref{eq:expandsgd}}, \eq{\ref{eq:sgdcvxsmth1}} and take totoal expectation over all samples at all iterations, we have
\begin{align}
\EE\left\|\ov{w}_{t+1}-\vw^{*}\right\|^2 &\leq \EE\|\ov{w}_{t}-\vw^{*}\|^{2}+\eta_{t}^{2} \EE\| \ov{g}_{t} \|^{2} + \eta_t L \sum_{k=1}^N p_k \EE \|\ov{w}_t - \vw^{k}_t\|^2 \nonumber \\
& + 2 \eta_{t} (F^* - \EE F(\ov{w}_t)) + \eta_{t}^2\EE\left\| \ov{g}_t -  \vg_{t}\right\|^2+ \eta_t^2 C\\
&\leq \EE\|\ov{w}_{t}-\vw^{*}\|^{2} + 2 \eta_{t} (F^* - \EE F(\ov{w}_t)) + \eta_{t}^{2} G^2 +4 L \eta_{t}^{3}E^{2} G^{2} + \eta_{t}^{2}\sum_{k=1}^{N} p_{k}^{2}\sigma_k^2 + \eta_t^2 C\\
&\leq \EE\|\ov{w}_{t}-\vw^{*}\|^{2} + 2 \eta_{t} (F^* - \EE F(\ov{w}_t)) + \underbrace{\eta_{t}^{2} G^2 +4 L \eta_{t}^{2}E^{2} G^{2} + \eta_{t}^{2}\sum_{k=1}^{N} p_{k}^{2}\sigma_k^2 + \eta_t^2 C}_{\eta_t^2 A}
\label{eq:cvxsgd1}
\end{align}
where we use Lemma~\ref{lma:l3iclr}, Lemma~\ref{lma:iclrvar} and $\eta_t < 1$.

Sum two sides of \eq{\ref{eq:cvxsgd1}}, and set constant learning rate $\eta_t = \eta$, 
\begin{align*}
    \sum_{t=0}^{T-1} 2\eta_t (\EE[F^*_t] - F^* ) & \leq \EE \|\ov{w}_{0}-\vw^{*}\|^{2} + \sum_{t=0}^{T-1}\eta_t^2 A\\
\EE[ F_t^*] - F^*  & \leq \frac{\|\ov{w}_{0}-\vw^{*}\|^{2}}{2 \sum_{t=0}^{T-1} \eta_t } + \frac{A \sum_{t=0}^{T-1} \eta_t^2}{2 \sum_{t=0}^{T-1} \eta_t }
\end{align*}
where $A = (1 +4 L E^{2})G^{2} + \sum_{k=1}^{N} p_{k}^{2}\sigma_k^2 + C$.
It will converge under the following conditions: $ \lim_{T \rightarrow \infty }\sum_{t=0}^{T-1} \eta_t = \infty$, 
$ \lim_{T \rightarrow \infty }\sum_{t=0}^{T-1} \eta_t^2 < \infty$. 
\begin{align*}
	% \EE[ F_t^*] - F^*  & \leq \frac{\|\ov{w}_{0}-\vw^{*}\|^{2}}{2 \sum_{t=0}^{T-1} \eta_t } + \frac{A \sum_{t=0}^{T-1} \eta_t^2}{2 \sum_{t=0}^{T-1} \eta_t } \\
	\EE[ F_t^*] - F^*  & \leq \frac{\Delta_0 + A \sum_{t=0}^{T-1} \eta_t^2 }{2 \sum_{t=0}^{T-1} \eta_t }\\
	& \leq \left(\frac{ \Delta_0 A }{T}\right)^{1/2}
\end{align*}
where in the last line we set the learning rate satisfying $\eta_t =  \left(\frac{\Delta_0}{ TA }\right)^{1/2}$.
\end{proof}




% \begin{align}
% 	\EE\left\|\ov{w}_{t+1}-\vw^{*}\right\|^2 &\leq \EE\|\ov{w}_{t}-\vw^{*}\|^{2}+\eta_{t}^{2} \EE\| \ov{g}_{t} \|^{2} + \eta_t L \sum_{k=1}^N p_k \EE \|\ov{w}_t - \vw^{k}_t\|^2 + 2 \eta_{t} (F^* - \EE F(\ov{w}_t)) + \eta_{t}^2\EE\left\| \ov{g}_t -  \vg_{t}\right\|^2+ \eta_t^2 C\\
% &\leq \EE\|\ov{w}_{t}-\vw^{*}\|^{2}+\eta_{t}^{2} \EE\| \ov{g}_t \|^{2} + \eta_t L \sum_{k=1}^N p_k \EE \|\ov{w}_t - \vw^{k}_t\|^2  \nonumber\\
%  & - \eta_{t}^{2}\EE\|\overline{\vg}_{t}\|^{2}+\eta_{t}^{2}L^{2}\sum_{k}p_{k}\EE\|\overline{\vw}_{t}-\vw_{t}^{k}\|^{2}+\eta_{t}^{3}L\mathbb{E}\|\vg_{t}\|^{2} + \eta_{t}^2\EE\left\| \ov{g}_t -  \vg_{t}\right\|^2 +\eta_t^2 C \\
%  &\leq \EE\|\ov{w}_{t}-\vw^{*}\|^{2} + (\eta_t L + \eta_{t}^{2}L^{2}) \sum_{k=1}^N p_k \EE \|\ov{w}_t - \vw^{k}_t\|^2  +\eta_{t}^{3}L\mathbb{E}\|\vg_{t}\|^{2} + \eta_{t}^2\EE\left\| \ov{g}_t -  \vg_{t}\right\|^2+\eta_t^2 C \\
%   &\leq \EE\|\ov{w}_{t}-\vw^{*}\|^{2} + \underbrace{4(1 + \eta_{t}L)\eta_t^3LE^2G^2  +\eta_{t}^{3}LG^2 + \frac{1}{N}\eta_{t}^2\nu_{\max}^2\sigma^2 +\eta_t^2 C }_{A}
% \end{align}

% \begin{align}
% 		\EE\left\|\ov{w}_{T}-\vw^{*}\right\|^2 & \leq \EE \|\ov{w}_{0}-\vw^{*}\|^{2}  + TA\\ 
%     & \leq \EE \|\ov{w}_{0}-\vw^{*}\|^{2} - \EE\left\|\ov{w}_{T}-\vw^{*}\right\|^2 + \sum_{t=0}^{T-1} (\eta_t^3 4LG^2(E-1)^2 + \eta_t^2 G^2+ \eta_t^2 C)\\ 
%     & \leq \EE \|\ov{w}_{0}-\vw^{*}\|^{2} + \sum_{t=0}^{T-1}\eta_t^2\left[ G^2 ( 4L(E-1)^2 + 1) + C\right]\\
% \end{align}

\subsection{Accelerated SGD}
\label{sec:nasgdcvxsmth}
% !TEX ROOT=./main.tex


\begin{theorem}
	Let Assumption~\ref{ass:lsmooth} and Assumption~\ref{ass:squaregrad} hold,  choose the learning rate $\eta = \frac{\alpha}{1 - \beta} = \sqrt{\frac{\Delta_0}{(D+C)T}}$, $\beta \in (0, 1)$ and $\beta \leq \min\{1, \frac{1}{1 + \sqrt{\frac{\Delta_0}{(D+C)T}}}\}$, then the FedNestrovAve with partial device participation satisfies
	\begin{align}
		 \EE F(\hat{\vw}_T) - F^* &\leq \sqrt{\frac{\Delta_0(D + C)}{T}} + \frac{\Delta_0^{3/2}}{2\sqrt{D+C}}\frac{1}{T^{3/2}} 
	\end{align}
	where $\hat{\vw}_T = \frac{1}{T}\sum_{t=0}^{T-1} \ov{w}_t$, $C$ is defined in \eq{\ref{eq:partialsample}},
$D$ is given by $D =  G^2[6 + (4(E-1)^2+1)L]$.
\end{theorem}

\begin{proof}
Follow the same derivation in \eq{\ref{eq:nagcvx1}}, we have the follow inequality.
\begin{align}
		\EE\|\overline{\vz}_{t+1} - \vw^* \|^2  & \leq  \| \overline{\vz}_{t} - \vw^*\|^2  - 2\eta_t\sum_{k=1}^K p_k\left<\EE g_{t,k}, \overline{\vz}_{t} - \vw^*\right> +  \eta_t^2\EE\|\vg_t\|^2 \label{eq:nasgdcvxsm0}
\end{align}

Bound $- 2\eta_t\sum_{k=1}^K p_k\left<\EE g_{t,k}, \overline{\vz}_{t} - \vw^*\right>$, 
\begin{align}
	& -2 \eta_{t} \left<\EE \vg_{t}, \ov{z}_{t}-\vw^{*}\right> \nonumber \\
 =  & -2 \eta_{t} \left<\EE \vg_{t}, \ov{z}_{t}- \vw^{k}_t \right> -2 \eta_{t} \left<\EE \vg_{t}, \vw^{k}_t - \vw^{*}\right>\nonumber \\
 \leq & -2 \eta_{t} \left<\EE \vg_{t}, \ov{z}_{t}- \vw^{k}_t \right> + 2 \eta_{t} (F_k(w^*) - F_k(\vw^{k}_t))\nonumber \\
 \leq &\sum_{k=1}^N p_k \left[2 \eta_{t} (F_k(\vw^{k}_t) - F_k(\ov{z}_t) + \frac{L}{2} \|\ov{z}_t - \vw^{k}_t\|^2 ) + 2 \eta_{t} (F_k(w^*) - F_k(\vw^{k}_t))\right]\nonumber \\
 = & \eta_t L \sum_{k=1}^N p_k  \|\ov{w}_t - \vw^{k}_t\|^2  + \sum_{k=1}^N p_k\eta_t^2 \|\EE \vg_{t,k}\|^2 	 + 2 \eta_{t} \sum_{k=1}^N p_k (F_k(w^*) - F_k(\ov{z}_t)) 
  \label{eq:nasgdcvxsm1}\\
  = & \eta_t L \sum_{k=1}^N p_k  \|\ov{w}_t - \vw^{k}_t\|^2  + \sum_{k=1}^N p_k\eta_t^2 \|\EE \vg_{t,k}\|^2 	 + 2 \eta_{t} \sum_{k=1}^N p_k (F_k(w^*) - F_k(\ov{w}_t) +  F_k(\ov{w}_t) -  F_k(\ov{z}_t)) \\
  = & \eta_t L \sum_{k=1}^N p_k  \|\ov{w}_t - \vw^{k}_t\|^2  + \sum_{k=1}^N p_k\eta_t^2 \|\EE \vg_{t,k}\|^2 	 + 2 \eta_{t} \sum_{k=1}^N p_k (F_k(w^*) - F_k(\ov{w}_t)) +  2 \eta_{t} \sum_{k=1}^N p_k  (F_k(\ov{w}_t) -  F_k(\ov{z}_t)) \label{eq:nasgdcvxsm1}
\end{align}

Now we bound $2 \eta_{t} \sum_{k=1}^N p_k  (F_k(\ov{w}_t) -  F_k(\ov{z}_t)) $

\begin{align}
	 & 2 \eta_{t} \sum_{k=1}^N p_k  (F_k(\ov{w}_t) -  F_k(\ov{z}_t))  \\
\leq & 2 \eta_{t} \sum_{k=1}^N p_k  (\left<\vg_{t,k}(\ov{z}_t), \ov{w}_t - \ov{z}_t\right> + \frac{L}{2}\|\ov{w}_t - \ov{z}_t\|^2)\\
\leq & 2 \eta_{t} \sum_{k=1}^N p_k (\frac{\eta_t}{2}\|\vg_{t,k}(\ov{z}_t)\|^2 + \frac{1}{2\eta_t}\|\ov{w}_t - \ov{z}_t\|^2+ \frac{L}{2}\|\ov{w}_t - \ov{z}_t\|^2)\\
\leq & \sum_{k=1}^N p_k \eta_t^2\|\vg_{t,k}(\ov{z}_t)\|^2 + (1+ \eta_t L)\|\ov{w}_t - \ov{z}_t\|^2 \label{eq:nasgdcvxsm2}
\end{align}
Plug in \eq{\ref{eq:nasgdcvxsm2}} to \eq{\ref{eq:nasgdcvxsm1}}, 

\begin{align}
	& -2 \eta_{t} \left<\EE \vg_{t}, \ov{z}_{t}-\vw^{*}\right> \nonumber \\
\leq & \eta_t L \sum_{k=1}^N p_k  \|\ov{w}_t - \vw^{k}_t\|^2  + \sum_{k=1}^N p_k\eta_t^2 \|\EE \vg_{t,k}\|^2 	 + 2 \eta_{t} \sum_{k=1}^N p_k (F_k(w^*) - F_k(\ov{w}_t)) \nonumber \\
 & +  2 \eta_{t} \sum_{k=1}^N p_k  (F_k(\ov{w}_t) -  F_k(\ov{z}_t)) \\
\leq & \eta_t L \sum_{k=1}^N p_k  \|\ov{w}_t - \vw^{k}_t\|^2  + \sum_{k=1}^N p_k\eta_t^2 \|\EE \vg_{t,k}\|^2 	 + 2 \eta_{t} \sum_{k=1}^N p_k (F_k(w^*) - F_k(\ov{w}_t)) \nonumber \\
  & +  \sum_{k=1}^N p_k \eta_t^2\|\vg_{t,k}(\ov{z}_t)\|^2 + (1+ \eta_t L)\|\ov{w}_t - \ov{z}_t\|^2 \label{eq:nasgdcvxsm3}
\end{align}

Now we can bound $\|\ov{w}_t - \ov{z}_t\|^2 = \| \ov{p}_t\|^2$ follow the \eq{\ref{eq:nagcvx6}}: 
\begin{align}
	\EE \|\ov{w}_t - \ov{z}_t\|^2 \leq \frac{\beta^4}{(1-\beta)^2}\EE\|\sum_{j=0}^{t_0 - 1}\beta^{t_0-1-j}\vg_j\|^2 
	\label{eq:nasgdcvxsm4}
\end{align}

Follow the proof in \eq{\ref{eq:nagcvx4-1}}, we can bound $\|\ov{w}_t - \vw^{k}_t\|^2$ as follows:
\begin{align}
	\EE \|\ov{w}_t - \vw^{k}_t\|^2  \leq \EE \|\sum_{j=t_0}^{t-1}\eta_j \vg_{j,k}\|^2 + \frac{(\beta^{t-t_0}-1)^2\beta^4}{(1-\beta)^2}\EE\|\sum_{j=0}^{t_0 - 1}\beta^{t_0-1-j}\vg_j\|^2 + \frac{\beta^4}{(1 - \beta)^2} \EE\|\sum_{j=t_0}^{t-1}\beta^{t-j-1}\vg_{j,k}\|^2 \label{eq:nasgdcvxsm5}
\end{align}

Plug in \eq{\ref{eq:nasgdcvxsm4}}, \eq{\ref{eq:nasgdcvxsm5}}, and \eq{\ref{eq:nasgdcvxsm3}} to \eq{\ref{eq:nasgdcvxsm0}}: 
\begin{align}
		\EE\|\overline{\vz}_{t+1} - \vw^* \|^2  & \leq  \| \overline{\vz}_{t} - \vw^*\|^2 +  \eta_t^2\EE\|\vg_t\|^2  \\
 & + \eta_t L \sum_{k=1}^N p_k  \|\ov{w}_t - \vw^{k}_t\|^2  + \sum_{k=1}^N p_k\eta_t^2 \|\EE \vg_{t,k}\|^2 	 + 2 \eta_{t} \sum_{k=1}^N p_k (F_k(w^*) - F_k(\ov{w}_t)) \nonumber \\
  & +  \sum_{k=1}^N p_k \eta_t^2\|\vg_{t,k}(\ov{z}_t)\|^2 + (1+ \eta_t L)\|\ov{w}_t - \ov{z}_t\|^2 \\
 & \leq  \| \overline{\vz}_{t} - \vw^*\|^2 +  \eta_t^2\EE\|\vg_t\|^2 + 2 \eta_{t} \sum_{k=1}^N p_k (F_k(w^*) - F_k(\ov{w}_t)) \\
 & +  \sum_{k=1}^N p_k \eta_t^2\|\vg_{t,k}(\ov{z}_t)\|^2 + \sum_{k=1}^N p_k\eta_t^2 \|\EE \vg_{t,k}\|^2  	  \nonumber \\
  & + \eta_t L \sum_{k=1}^N p_k  \|\ov{w}_t - \vw^{k}_t\|^2    + (1+ \eta_t L)\|\ov{w}_t - \ov{z}_t\|^2 \\
& \leq  \| \overline{\vz}_{t} - \vw^*\|^2 + 2 \eta_{t} \sum_{k=1}^N p_k (F_k(w^*) - F_k(\ov{w}_t)) \\
 & + \eta_t^2\EE\|\vg_t\|^2  +  \sum_{k=1}^N p_k \eta_t^2\|\vg_{t,k}(\ov{z}_t)\|^2 + \sum_{k=1}^N p_k\eta_t^2 \|\EE \vg_{t,k}\|^2  	  \nonumber \\
  & + \eta_t L \sum_{k=1}^N p_k \EE \|\sum_{j=t_0}^{t-1}\eta_j \vg_{j,k}\|^2 + \frac{(\beta^{t-t_0}-1)^2\beta^4}{(1-\beta)^2}\EE\|\sum_{j=0}^{t_0 - 1}\beta^{t_0-1-j}\vg_j\|^2 + \frac{\beta^4}{(1 - \beta)^2} \EE\|\sum_{j=t_0}^{t-1}\beta^{t-j-1}\vg_{j,k}\|^2   \\
  &   + (1+ \eta_t L)\frac{\beta^4}{(1-\beta)^2}\EE\|\sum_{j=0}^{t_0 - 1}\beta^{t_0-1-j}\vg_j\|^2  \label{eq:nasgdcvxsm6}
\end{align}
Denote last three lines in \eq{\ref{eq:nasgdcvxsm6}} as $C_1$, we now give upper bound of $C_1$
\begin{align*}
C_1 &= 	\eta_t^2\EE\|\vg_t\|^2  +  \sum_{k=1}^N p_k \eta_t^2\|\vg_{t,k}(\ov{z}_t)\|^2 + \sum_{k=1}^N p_k\eta_t^2 \|\EE \vg_{t,k}\|^2  	  \nonumber \\
  &  + \eta_t L \sum_{k=1}^N p_k \EE \|\sum_{j=t_0}^{t-1}\eta_j \vg_{j,k}\|^2 + \frac{(\beta^{t-t_0}-1)^2\beta^4}{(1-\beta)^2}\EE\|\sum_{j=0}^{t_0 - 1}\beta^{t_0-1-j}\vg_j\|^2 + \sum_{k=1}^N p_k\frac{\beta^4}{(1 - \beta)^2} \EE\|\sum_{j=t_0}^{t-1}\beta^{t-j-1}\vg_{j,k}\|^2   \\
  & + (1+ \eta_t L)\frac{\beta^4}{(1-\beta)^2}\EE\|\sum_{j=0}^{t_0 - 1}\beta^{t_0-1-j}\vg_j\|^2 \\
  & = \underbrace{\eta_t^2\EE\|\vg_t\|^2  +  \sum_{k=1}^N p_k \eta_t^2\|\vg_{t,k}(\ov{z}_t)\|^2 + \sum_{k=1}^N p_k\eta_t^2 \|\EE \vg_{t,k}\|^2}_{C_2} + \underbrace{\eta_t L \sum_{k=1}^N p_k \EE \|\sum_{j=t_0}^{t-1}\eta_j \vg_{j,k}\|^2}_{C_3}   \nonumber \\
  & + \underbrace{\frac{[1+ \eta_t L + (\beta^{t-t_0}-1)^2]\beta^4}{(1-\beta)^2}\EE\|\sum_{j=0}^{t_0 - 1}\beta^{t_0-1-j}\vg_j\|^2}_{C_4} + \underbrace{\sum_{k=1}^N p_k\frac{\beta^4}{(1 - \beta)^2} \EE\|\sum_{j=t_0}^{t-1}\beta^{t-j-1}\vg_{j,k}\|^2}_{C_5}   \label{eq:nasgdcvxsm7}
\end{align*}

With Assumption~\ref{ass:subgrad2}, the upper bound of first line is given by:
\begin{align*}
		C_2 = \eta_t^2\EE\|\vg_t\|^2  +  \sum_{k=1}^N p_k \eta_t^2\|\vg_{t,k}(\ov{z}_t)\|^2 + \sum_{k=1}^N p_k\eta_t^2 \|\EE \vg_{t,k}\|^2  	  \leq 3 \eta_t^2 G^2
\end{align*}
\begin{align*}
C_3 =	& \eta_t L \sum_{k=1}^N p_k \EE \|\sum_{j=t_0}^{t-1}\eta_j \vg_{j,k}\|^2  \leq \eta_t L (4 (E-1)^2\eta_t^2 G^2) = 4(E-1)^2\eta_t^3L G^2 
\end{align*}
\begin{align*}
C_4 = 	\frac{[1+ \eta_t L + (\beta^{t-t_0}-1)^2]\beta^4}{(1-\beta)^2}\EE\|\sum_{j=0}^{t_0 - 1}\beta^{t_0-1-j}\vg_j\|^2  \leq \eta_t^4(2 + L) G^2 
\end{align*}
\begin{align*}
	C_5 = \sum_{k=1}^N p_k\frac{\beta^4}{(1 - \beta)^2} \EE\|\sum_{j=t_0}^{t-1}\beta^{t-j-1}\vg_{j,k}\|^2 
	\leq \eta_t^4 G^2
\end{align*}

Plug in $C_2 - C_5$ into \eq{\ref{eq:nasgdcvxsm6}}, 

\begin{align}
	\EE\|\overline{\vz}_{t+1} - \vw^* \|^2 & \leq  \| \overline{\vz}_{t} - \vw^*\|^2 + 2 \eta_{t} \sum_{k=1}^N p_k (F_k(w^*) - F_k(\ov{w}_t))  + 3\eta_t G^2 + 4(E-1)^2\eta_t^3L G^2  +  \eta_t^4(2 + L) G^2  + \eta_t^4 G^2 \\
& \leq  \| \overline{\vz}_{t} - \vw^*\|^2 + 2 \eta_{t} \sum_{k=1}^N p_k (F_k(w^*) - F_k(\ov{w}_t))  + \eta_t^2 \underbrace{G^2[6 + (4(E-1)^2+1)L] }_{D}
\end{align}
Sum two sides over time step $t$, take totoal expectation and use constant rate $\eta_t = \eta$ and follow \eq{\ref{eq:nagcvxc4}}:
\begin{align}
	\EE F(\hat{w}_t) - F^* & \leq \sqrt{\frac{\Delta_0D}{T}} + \frac{\Delta_0^{3/2}}{2\sqrt{D}}\frac{1}{T^{3/2}} \nonumber\\
\end{align}
where $D = G^2[6 + (4(E-1)^2+1)L] $, $\hat{w}_t = \frac{1}{T}\sum_{t=0}^{T-1} \ov{w}_t$, $\eta = \sqrt{\frac{\Delta_0}{DT}}$.

\textbf{Consider partial device participation:}
\begin{align*}
	 \EE F(\hat{w}_t) - F^* & \leq \sqrt{\frac{\Delta_0(D + C)}{T}} + \frac{\Delta_0^{3/2}}{2\sqrt{D+C}}\frac{1}{T^{3/2}} \nonumber\\
\end{align*}
where $\eta = \sqrt{\frac{\Delta_0}{(D+C)T}}$. 
\end{proof}

% Approach 1. 
% \begin{align}
% 	& -2 \eta_{t} \left<\EE \vg_{t}, \ov{z}_{t}-\vw^{*}\right> \nonumber \\
%  =  & -2 \eta_{t} \left<\EE \vg_{t}, \ov{z}_{t}- \vw^{k}_t \right> -2 \eta_{t} \left<\EE \vg_{t}, \vw^{k}_t - \vw^{*}\right>\nonumber \\
%  \leq & -2 \eta_{t} \left<\EE \vg_{t}, \ov{z}_{t}- \vw^{k}_t \right> + 2 \eta_{t} (F_k(w^*) - F_k(\vw^{k}_t))\nonumber \\
%  \leq & -2 \eta_{t} \left<\EE \vg_{t}, \ov{w}_{t} - \vw^{k}_t \right> -2 \eta_{t} \left<\EE \vg_{t},  \ov{p}_{t} - \vw^{k}_t \right> + 2 \eta_{t} (F_k(w^*) - F_k(\vw^{k}_t))\nonumber \\
%  \leq &\sum_{k=1}^N p_k \left[2 \eta_{t} (F_k(\vw^{k}_t) - F_k(\ov{w}_t) + \frac{L}{2} \|\ov{w}_t - \vw^{k}_t\|^2 ) + \|\ov{p}_t - \vw_t^k\|^2 + \eta_t^2 \|\EE \vg_{t,k}\|^2  + 2 \eta_{t} (F_k(w^*) - F_k(\vw^{k}_t))\right]\nonumber \\
%  = & \eta_t L \sum_{k=1}^N p_k  \|\ov{w}_t - \vw^{k}_t\|^2 + \sum_{k=1}^N p_k \|\ov{p}_t - \vw_t^k\|^2 + \sum_{k=1}^N p_k\eta_t^2 \|\EE \vg_{t,k}\|^2 	 + 2 \eta_{t} \sum_{k=1}^N p_k (F_k(w^*) - F_k(\ov{w}_t)) \label{eq:nasgdcvxsm1}
% \end{align}

\section{Proof for Exponential Convergence Results on Overparameterized Linear Regression}
\label{sec:interpolation}
In contrast to the gradient descent setting, in the stochastic gradient
setting, Nesterov update is known to fail to accelerate over stochastic
gradient. Thus in general we cannot hope to obtain acceleration results
for the FedAvg algorithm with Nesterov updates. However, in special
problem settings, there have been a number of works showing that accelerated
stochastic gradient descent methods achieves better rates than stochastic
gradient descent. One such case is the so-called interpolation setting,
where the objective has an optimal value of 0, e.g. due to overparameterization.
The works of Belkin et.al have shown that under interpolation, SGD
achieves exponential convergence under constant step-size, thanks
to the automatic variance reduction property that is not available
in generic optimization problems. In this section, we extend their
result to the federated learning problem and prove the exponential
convergence of SGD for a class of convex objectives in the interpolation
setting.

In the non-federated learning setting, {[}TODO: CITE{]} have provided
negative results on the acceleration of Nesterov and Heavy Ball updates
over plain SGD for certain problems. In {[}Liu\&Belkin{]}, the authors
show that in the interpolation setting, stochastic Nesterov update
is not able to achieve accleration over vanilla SGD. They introduce
the MaSS algorithm, which is a variant of the Nesterov acceleration,
where in the update a non-negative multiple of the gradient is added
to the Nesterov parameter update to correct for the ``over-descent''
of the Nesterov update. They show that the MaSS algorithm is able
to achieve acceleration over the exponential convergence of SGD both
theoretically and empirically.

The next natural question is then whether similar conclusions can
be drawn in the federated learning setting. We demonstrate that this
is possible by extending the results of Belkin et al. to show that
for the federated learning problem in the interpolation setting, Nesterov
SGD cannot in general achieve acceleration over plain SGD. We then
adapt the MaSS algorithm to the federated learning setting, and prove
that for certain class of convex objectives with 0 global objective
value, the FedAvg algorithm with MaSS updates achieves exponential
convergence and acceleration of convergence rate over SGD. 

We first show that FedAvg with MaSS has exponential convergence for
linear regression when the global objective has 0 as its global minimum.
Since the global loss is given by 
\begin{align*}
F(w) & =\sum_{k=1}^{N}p_{k}F_{k}(w)\\
F_{k}(w) & =\frac{1}{2n_{k}}\sum_{j=1}^{n_{k}}(w^{T}x_{k,j}-z_{k,j})^{2}
\end{align*}
 and there exists $w^{\ast}$ such that $F(w^{\ast})=0$, in particular
this implies that $F_{k}^{\ast}=F_{k}(w^{\ast})=F^{\ast}=0$ for all
$k$.

\subsection{Notation and Definitions}

Following Liu\&Belkin and Jain et al., we define some condition number
related quantities. Let $H^{k}=\frac{1}{n_{k}}\sum_{j=1}^{n_{k}}x_{k,j}x_{k,j}$
be the Hessian matrix of $F_{k}$. Let $L^{k}$ and $\mu^{k}$ be
any upper bound and lower bound of the non-zero eigenvalues of the
Hessians $H^{k}$. For a mini-batch $\{\tilde{x}_{j}\}_{j=1}^{m}$
of $m$ samples from device $k$, let $\tilde{H}_{m}^{k}=\frac{1}{m}\sum_{j=1}^{m}\tilde{x}_{j}\tilde{x}_{j}^{T}$
be the unbiased mini-batch estimate of $H^{k}$. Let $L_{1}^{k}$
be the smallest positive number such that 
\begin{align*}
\mathbb{E}\left[\|\tilde{x}\|^{2}\tilde{x}\tilde{x}^{T}\right] & \preceq L_{1}^{k}H^{k}
\end{align*}
 and define 
\begin{align*}
L_{m}^{k} & =L_{1}^{k}/m+(m-1)L^{k}/m
\end{align*}

Define the $m$-stochastic condition number as $\kappa_{m}^{k}:=L_{m}^{k}/\mu$.
Define $L_{m}=\max_{k}L_{m}^{k}$ and $\kappa_{m}=\max_{k}\kappa_{m}^{k}$.
Define the statistical condition number $\tilde{\kappa}^{k}$ as the
smallest positive real number such that 
\begin{align*}
\mathbb{E}\left[\langle\tilde{x}(H^{k})^{-1},\tilde{x}\rangle\tilde{x}\tilde{x}^{T}\right] & \preceq\tilde{\kappa}^{k}H^{k}
\end{align*}

In order to use hyperparameters that are universal across devices,
we further define $L=\max_{k}L^{k}$, $\mu=\min_{k}\mu^{k}$, $L_{m}=\max_{k}L_{m}^{k}$
and $\kappa_{m}=\max_{k}\kappa_{m}^{k}$. 

\subsection{FedAvg with Nesterov and MaSS}

The FedAvg algorithm with MaSS follows the updates
\begin{align*}
w_{t+1}^{k} & =\begin{cases}
u_{t}^{k}-\eta_{1}^{k}g_{t,k} & \text{if }t+1\notin\mathcal{I}_{E}\\
\sum_{k=1}^{N}p_{k}\left[u_{t}^{k}-\eta_{1}^{k}g_{t,k}\right] & \text{if }t+1\in\mathcal{I}_{E}
\end{cases}\\
u_{t+1}^{k} & =w_{t+1}^{k}+\gamma^{k}(w_{t+1}^{k}-w_{t}^{k})+\eta_{2}^{k}g_{t,k}
\end{align*}
 where we note that the natural parameter is $w_{t}$, while $u_{t}$
is an auxiliary parameter, which we initialize to be $u_{0}^{k}$,
and 
\begin{align*}
g_{t,k} & :=\nabla F_{k}(u_{t}^{k},\xi_{t}^{k})
\end{align*}
 is the stochastic gradient and 
\begin{align*}
g_{t} & =\sum_{k=1}^{N}p_{k}g_{t,k}
\end{align*}
is the averaged stochastic gradient. When $\eta_{2}^{k}\equiv0$,
this reduces to the FedAvg algorithm with Nesterov updates.

We note that the update can equivalently be written as 
\begin{align*}
v_{t+1}^{k} & =(1-\alpha^{k})v_{t}^{k}+\alpha^{k}w_{t}^{k}-\delta^{k}g_{t,k}\\
w_{t+1}^{k} & =\begin{cases}
u_{t}^{k}-\eta^{k}g_{t,k} & \text{if }t+1\notin\mathcal{I}_{E}\\
\sum_{k=1}^{N}p_{k}\left[u_{t}^{k}-\eta^{k}g_{t,k}\right] & \text{if }t+1\in\mathcal{I}_{E}
\end{cases}\\
u_{t+1}^{k} & =\frac{\alpha^{k}}{1+\alpha^{k}}v_{t+1}^{k}+\frac{1}{1+\alpha^{k}}w_{t+1}^{k}
\end{align*}
 where there is a bijection between the parameters 
\begin{align*}
\frac{1-\alpha^{k}}{1+\alpha^{k}} & =\gamma^{k}\\
\eta^{k} & =\eta_{1}^{k}\\
\frac{\eta^{k}-\alpha^{k}\delta^{k}}{1+\alpha^{k}} & =\eta_{2}^{k}
\end{align*}
 and we further introduce an auxiliary parameter $v_{t}^{k}$, which
is initialized at $v_{0}^{k}$. We also note that when $\delta^{k}=\frac{\eta^{k}}{\alpha^{k}}$,
the update reduces to the Nesterov accelerated SGD. This version of
the FedAvg with MaSS algorithm is used for analyzing the exponential
convergence. 

As before, define the virtual sequences $\overline{w}_{t}=\sum_{k=1}^{N}p_{k}w_{t}^{k}$,
$\overline{v}_{t}=\sum_{k=1}^{N}p_{k}v_{t}^{k}$, $\overline{u}_{t}=\sum_{k=1}^{N}p_{k}u_{t}^{k}$,
and $\overline{g}_{t}=\sum_{k=1}^{N}p_{k}\mathbb{E}g_{t,k}$. We have
$\mathbb{E}g_{t}=\overline{g}_{t}$ and $\overline{w}_{t+1}=\overline{u}_{t}-\eta_{t}g_{t}$,
$\overline{v}_{t+1}=(1-\alpha^{k})\overline{v}_{t}+\alpha^{k}\overline{w}_{t}-\delta^{k}g_{t}$,
and $\overline{u}_{t+1}=\frac{\alpha^{k}}{1+\alpha^{k}}\overline{v}_{t+1}+\frac{1}{1+\alpha^{k}}\overline{w}_{t+1}$. 

For the linear regression problem in the interpolation setting, we
can write 
\begin{align*}
F(w) & =\frac{1}{2}(w-w^{\ast})^{T}H(w-w^{\ast})\\
 & =\frac{1}{2}\|w-w^{\ast}\|_{H}^{2}
\end{align*}
 and similarly $F^{k}(w)=\frac{1}{2}\|w-w^{\ast}\|_{H^{k}}^{2}$,
so that 
\begin{align*}
g_{t,k} & =\tilde{H}_{t}^{k}(w_{t}^{k}-w^{\ast})\\
g_{t} & =\sum_{k=1}^{N}p_{k}\tilde{H}_{t}^{k}(w_{t}^{k}-w^{\ast})
\end{align*}


\subsection{Exponential Convergence of FedAvg with SGD and Nesterov}

\subsection{Exponential Convergence of FedAvg with MaSS}

We now present the exponential convergence result in linear regression
using FedAvg with MaSS updates. On each device, local data is stored
and mini-batch gradient descent with batch size $m$ is performed.
We assume that the batch size is the same across devices. 
\begin{theorem}
(FedAvg with MaSS, Linear Regression) Let $\tilde{\kappa}_{m}:=\tilde{\kappa}/m+(m-1)/m$,
and let the hyperparameters satisfy 
\begin{align*}
\eta^{k}(m)=\frac{1}{L_{m}}, & \alpha^{k}(m)=\frac{1}{\sqrt{\kappa_{m}\tilde{\kappa}_{m}}},\delta^{k}(m)=\frac{\eta^{k}}{\alpha^{k}\tilde{\kappa}_{m}}
\end{align*}
for all $k$. Let $\mathcal{I}_{E}=\{\ell E\mid\ell\in\mathbb{N}\}$
be the set of communication rounds, then the FedAvg algorithm with
MaSS and full participation solves the problem 
\begin{align*}
F(w) & =\sum_{k=1}^{N}p_{k}F_{k}(w)\\
F_{k}(w) & =\frac{1}{2n_{k}}\sum_{j=1}^{n_{k}}(w^{T}x_{k,j}-z_{k,j})^{2}
\end{align*}
with exponential convergence
\begin{align*}
\mathbb{E}F(\overline{w}_{t})\leq\frac{L}{2}\mathbb{E}\|\overline{w}_{t}-w^{\ast}\|^{2} & \leq C\cdot(1-\frac{1}{\sqrt{\kappa_{m}\tilde{\kappa}_{m}}})^{t}=O(\exp(-\frac{t}{\sqrt{\kappa_{m}\tilde{\kappa}_{m}}}))
\end{align*}
 where $w^{\ast}$ is such that $F(w^{\ast})=0$ and $C=\frac{L}{2}\cdot\frac{\alpha}{\delta}\sum_{k=1}^{N}p_{k}(\|v_{0}^{k}-w^{\ast}\|_{H_{k}^{-1}}^{2}+\frac{\delta}{\alpha}\|w_{0}^{k}-w^{\ast}\|^{2})$.
With partial participation of $K$ devices each communication round,
the convergence rate is also 
\begin{align*}
\mathbb{E}F(\overline{w}_{t})\leq\frac{L}{2}\mathbb{E}\|\overline{w}_{t}-w^{\ast}\|^{2} & \leq C\cdot(1+\frac{4}{K}\cdot(\frac{t}{E}-1))\cdot(1-\frac{1}{\sqrt{\kappa_{m}\tilde{\kappa}_{m}}})^{t}=O(\exp(-\frac{t}{\sqrt{\kappa_{m}\tilde{\kappa}_{m}}}))
\end{align*}
\end{theorem}
\begin{proof}
We first proof the convergence for full device participation. Note
that at each communication round we update the $w_{t+1}^{k}$ parameters
to be the average across devices while fixing $v_{t+1}^{k}$. This
automatically adjusts the $u_{t+1}^{k}$ parameter at each device
by the relation 
\begin{align*}
u_{t+1}^{k} & =\frac{\alpha^{k}}{1+\alpha^{k}}v_{t+1}^{k}+\frac{1}{1+\alpha^{k}}w_{t+1}^{k}
\end{align*}
 valid for all $t\geq0$. Note also that the hyperparameters are chosen
the same for all devices: $\delta_{m}\equiv\delta$, $\alpha_{m}\equiv\alpha$,
and $\eta_{m}\equiv\eta$. 

Theorems 2 and 3 of the Liu\&Belkin paper give the bound 
\begin{align*}
\mathbb{E}\left[\|v_{E}^{k}-w^{\ast}\|_{H_{k}^{-1}}^{2}+\frac{\delta}{\alpha}\|u_{E}^{k}-\eta g_{E,k}-w^{\ast}\|^{2}\right] & \leq(1-\alpha)^{E}(\|v_{0}^{k}-w^{\ast}\|_{H_{k}^{-1}}^{2}+\frac{\delta}{\alpha}\|w_{0}^{k}-w^{\ast}\|^{2})
\end{align*}
for all $k$, where $E$ is the first communication round. Note that
$w_{E}^{k}=\overline{w}_{E}^{k}\neq u_{E}^{k}-\eta g_{E,k}$.

It follows from convexity that 
\begin{align*}
\mathbb{E}\left[\sum_{k=1}^{N}p_{k}\|v_{E}^{k}-w^{\ast}\|_{H_{k}^{-1}}^{2}+\frac{\delta}{\alpha}\|\overline{w}_{E}-w^{\ast}\|^{2}\right] & \leq\sum_{k=1}^{N}p_{k}\mathbb{E}\left[\|v_{E}^{k}-w^{\ast}\|_{H_{k}^{-1}}^{2}+\frac{\delta}{\alpha}\|u_{E}^{k}-\eta g_{E,k}-w^{\ast}\|^{2}\right]\\
 & \leq\sum_{k=1}^{N}p_{k}(1-\alpha)^{E}(\|v_{0}^{k}-w^{\ast}\|_{H_{k}^{-1}}^{2}+\frac{\delta}{\alpha}\|w_{0}^{k}-w^{\ast}\|^{2})
\end{align*}
Since $w_{E}^{k}=\overline{w}_{E}$ for all devices, applying the
per-device result again starting at $t=E$ instead of $t=0$, for
each device we have the bound
\begin{align*}
\mathbb{E}\left[\|v_{2E}^{k}-w^{\ast}\|_{H_{k}^{-1}}^{2}+\frac{\delta}{\alpha}\|u_{2E}^{k}-\eta g_{2E,k}-w^{\ast}\|^{2}\right] & \leq(1-\alpha)^{E}\mathbb{E}(\|v_{E}^{k}-w^{\ast}\|_{H_{k}^{-1}}^{2}+\frac{\delta}{\alpha}\|w_{E}^{k}-w^{\ast}\|^{2})\\
 & =(1-\alpha)^{E}\mathbb{E}(\|v_{E}^{k}-w^{\ast}\|_{H_{k}^{-1}}^{2}+\frac{\delta}{\alpha}\|\overline{w}_{E}-w^{\ast}\|^{2})
\end{align*}

Here we emphasize that $w_{E}^{k}$ results from broadcasting and
so is the same across all devices, while $v_{E}^{k}$ remains distinct
on each device (and is only auxiliary). Then by convexity and summing
the above inequalities across devices we have 
\begin{align*}
\mathbb{E}\left[\sum_{k=1}^{N}p_{k}\|v_{2E}^{k}-w^{\ast}\|_{H_{k}^{-1}}^{2}+\frac{\delta}{\alpha}\|\overline{w}_{2E}-w^{\ast}\|^{2}\right] & \leq\sum_{k=1}^{N}p_{k}\mathbb{E}\left[\|v_{2E}^{k}-w^{\ast}\|_{H_{k}^{-1}}^{2}+\frac{\delta}{\alpha}\|u_{2E}^{k}-\eta g_{2E,k}-w^{\ast}\|^{2}\right]\\
 & \leq\sum_{k=1}^{N}p_{k}(1-\alpha)^{E}\mathbb{E}(\|v_{E}^{k}-w^{\ast}\|_{H_{k}^{-1}}^{2}+\frac{\delta}{\alpha}\|\overline{w}_{E}-w^{\ast}\|^{2})\\
 & \leq\sum_{k=1}^{N}p_{k}(1-\alpha)^{2E}(\|v_{0}^{k}-w^{\ast}\|_{H_{k}^{-1}}^{2}+\frac{\delta}{\alpha}\|w_{0}^{k}-w^{\ast}\|^{2})
\end{align*}
 and by induction we can show that 
\begin{align*}
\mathbb{E}\left[\sum_{k=1}^{N}p_{k}\|v_{\ell E}^{k}-w^{\ast}\|_{H_{k}^{-1}}^{2}+\frac{\delta}{\alpha}\|\overline{w}_{\ell E}-w^{\ast}\|^{2}\right] & \leq\sum_{k=1}^{N}p_{k}(1-\alpha)^{\ell E}(\|v_{0}^{k}-w^{\ast}\|_{H_{k}^{-1}}^{2}+\frac{\delta}{\alpha}\|w_{0}^{k}-w^{\ast}\|^{2})
\end{align*}
 and more generally
\begin{align*}
\mathbb{E}\left[\sum_{k=1}^{N}p_{k}\|v_{t}^{k}-w^{\ast}\|_{H_{k}^{-1}}^{2}+\frac{\delta}{\alpha}\|\overline{w}_{t}-w^{\ast}\|^{2}\right] & \leq\sum_{k=1}^{N}p_{k}(1-\alpha)^{t}(\|v_{0}^{k}-w^{\ast}\|_{H_{k}^{-1}}^{2}+\frac{\delta}{\alpha}\|w_{0}^{k}-w^{\ast}\|^{2})
\end{align*}
 In particular, this implies 
\begin{align*}
\mathbb{E}\|\overline{w}_{t}-w^{\ast}\|^{2} & \leq C\cdot(1-\frac{1}{\sqrt{\kappa_{m}\tilde{\kappa}_{m}}})^{t}=O(\exp(-\frac{t}{\sqrt{\kappa_{m}\tilde{\kappa}_{m}}}))
\end{align*}

Now we prove the result for partial device participation. Note that
now 
\begin{align*}
\mathbb{E}\left[\|v_{2E}^{k}-w^{\ast}\|_{H_{k}^{-1}}^{2}+\frac{\delta}{\alpha}\|u_{2E}^{k}-\eta g_{2E,k}-w^{\ast}\|^{2}\right] & \leq(1-\alpha)^{E}\mathbb{E}(\|v_{E}^{k}-w^{\ast}\|_{H_{k}^{-1}}^{2}+\frac{\delta}{\alpha}\|w_{E}^{k}-w^{\ast}\|^{2})\\
 & =(1-\alpha)^{E}\mathbb{E}(\|v_{E}^{k}-w^{\ast}\|_{H_{k}^{-1}}^{2}+\frac{\delta}{\alpha}\|\frac{1}{K}\sum_{j}\tilde{w}_{E,j}-w^{\ast}\|^{2})
\end{align*}
 where $\tilde{w}_{E,j}$'s are i.i.d. drawn from the discrete distribution
on $w_{E}^{k}$ with probability $p_{k}$. We have 
\begin{align*}
\mathbb{E}\|\frac{1}{K}\sum_{j}\tilde{w}_{E,j}-w^{\ast}\|^{2} & =\mathbb{E}\|\frac{1}{K}\sum_{j}\tilde{w}_{E,j}-\overline{w}_{E}\|^{2}+\|\overline{w}_{E}-w^{\ast}\|^{2}
\end{align*}
 since $\mathbb{E}_{E}\tilde{w}_{E,j}=\overline{w}_{E}$. Now 
\begin{align*}
\mathbb{E}\|\frac{1}{K}\sum_{j}\tilde{w}_{E,j}-\overline{w}_{E}\|^{2} & =\mathbb{E}\frac{1}{K}\sum_{k=1}^{N}p_{k}\|u_{E}^{k}-\eta g_{E,k}-\overline{w}_{E}\|^{2}
\end{align*}
 Since 
\begin{align*}
\mathbb{E}\|u_{E}^{k}-\eta g_{E,k}-\overline{w}_{E}\|^{2} & \leq\mathbb{E}\|u_{E}^{k}-\eta g_{E,k}-w^{\ast}+w^{\ast}-\overline{w}_{E}\|^{2}\\
 & \leq2\mathbb{E}\|u_{E}^{k}-\eta g_{E,k}-w^{\ast}\|+\sum_{k}p_{k}\|w^{\ast}-u_{E}^{k}-\eta g_{E,k}\|^{2}
\end{align*}
 we have 
\begin{align*}
\mathbb{E}\|\frac{1}{K}\sum_{j}\tilde{w}_{E,j}-\overline{w}_{E}\|^{2} & \leq\frac{4}{K}\mathbb{E}\sum_{k=1}^{N}p_{k}\|w^{\ast}-u_{E}^{k}-\eta g_{E,k}\|^{2}\\
 & \leq\frac{4}{K}\mathbb{E}\sum_{k=1}^{N}p_{k}\|w^{\ast}-u_{E}^{k}-\eta g_{E,k}\|^{2}\\
 & \le\frac{4}{K}(1-\alpha)^{E}\frac{\alpha}{\delta}\sum_{k=1}^{N}p_{k}(\|v_{0}^{k}-w^{\ast}\|_{H_{k}^{-1}}^{2}+\frac{\delta}{\alpha}\|w_{0}^{k}-w^{\ast}\|^{2})
\end{align*}
where we have used
\begin{align*}
\mathbb{E}\left[\|v_{E}^{k}-w^{\ast}\|_{H_{k}^{-1}}^{2}+\frac{\delta}{\alpha}\|u_{E}^{k}-\eta g_{E,k}-w^{\ast}\|^{2}\right] & \leq(1-\alpha)^{E}(\|v_{0}^{k}-w^{\ast}\|_{H_{k}^{-1}}^{2}+\frac{\delta}{\alpha}\|w_{0}^{k}-w^{\ast}\|^{2})
\end{align*}
Thus 
\begin{align*}
\mathbb{E}\left[\|v_{2E}^{k}-w^{\ast}\|_{H_{k}^{-1}}^{2}+\frac{\delta}{\alpha}\|u_{2E}^{k}-\eta g_{2E,k}-w^{\ast}\|^{2}\right] & \leq(1-\alpha)^{E}\mathbb{E}(\|v_{E}^{k}-w^{\ast}\|_{H_{k}^{-1}}^{2}+\frac{\delta}{\alpha}\|w_{E}^{k}-w^{\ast}\|^{2})\\
 & =(1-\alpha)^{E}\mathbb{E}(\|v_{E}^{k}-w^{\ast}\|_{H_{k}^{-1}}^{2}+\frac{\delta}{\alpha}\|\frac{1}{K}\sum_{j}\tilde{w}_{E,j}-w^{\ast}\|^{2})\\
 & \leq(1-\alpha)^{E}\mathbb{E}(\|v_{E}^{k}-w^{\ast}\|_{H_{k}^{-1}}^{2}+\frac{\delta}{\alpha}\|\overline{w}_{E}-w^{\ast}\|^{2})+\frac{4}{K}(1-\alpha)^{2E}\sum_{k=1}^{N}p_{k}(\|v_{0}^{k}-w^{\ast}\|_{H_{k}^{-1}}^{2}+\frac{\delta}{\alpha}\|w_{0}^{k}-w^{\ast}\|^{2})
\end{align*}
 

Summing over devices 
\begin{align*}
\mathbb{E}\left[\sum_{k=1}^{N}p_{k}\|v_{2E}^{k}-w^{\ast}\|_{H_{k}^{-1}}^{2}+\frac{\delta}{\alpha}\|\overline{w}_{2E}-w^{\ast}\|^{2}\right] & \leq\sum_{k=1}^{N}p_{k}\mathbb{E}\left[\|v_{2E}^{k}-w^{\ast}\|_{H_{k}^{-1}}^{2}+\frac{\delta}{\alpha}\|u_{2E}^{k}-\eta g_{2E,k}-w^{\ast}\|^{2}\right]\\
 & \leq\sum_{k=1}^{N}p_{k}(1-\alpha)^{E}\mathbb{E}(\|v_{E}^{k}-w^{\ast}\|_{H_{k}^{-1}}^{2}+\frac{\delta}{\alpha}\|\overline{w}_{E}-w^{\ast}\|^{2})+\frac{4}{K}(1-\alpha)^{2E}\sum_{k=1}^{N}p_{k}(\|v_{0}^{k}-w^{\ast}\|_{H_{k}^{-1}}^{2}+\frac{\delta}{\alpha}\|w_{0}^{k}-w^{\ast}\|^{2})\\
 & \leq\sum_{k=1}^{N}p_{k}(1-\alpha)^{2E}(\|v_{0}^{k}-w^{\ast}\|_{H_{k}^{-1}}^{2}+\frac{\delta}{\alpha}\|w_{0}^{k}-w^{\ast}\|^{2})+\frac{4}{K}(1-\alpha)^{2E}\sum_{k=1}^{N}p_{k}(\|v_{0}^{k}-w^{\ast}\|_{H_{k}^{-1}}^{2}+\frac{\delta}{\alpha}\|w_{0}^{k}-w^{\ast}\|^{2})\\
 & =(1+\frac{4}{K})\cdot(1-\alpha)^{2E}\sum_{k=1}^{N}p_{k}(\|v_{0}^{k}-w^{\ast}\|_{H_{k}^{-1}}^{2}+\frac{\delta}{\alpha}\|w_{0}^{k}-w^{\ast}\|^{2})
\end{align*}

By induction, we can show in general 
\begin{align*}
\mathbb{E}\left[\sum_{k=1}^{N}p_{k}\|v_{t}^{k}-w^{\ast}\|_{H_{k}^{-1}}^{2}+\frac{\delta}{\alpha}\|\overline{w}_{t}-w^{\ast}\|^{2}\right] & \leq(1+\frac{4}{K}\cdot(\frac{t}{E}-1))\cdot(1-\alpha)^{t}\sum_{k=1}^{N}p_{k}(\|v_{0}^{k}-w^{\ast}\|_{H_{k}^{-1}}^{2}+\frac{\delta}{\alpha}\|w_{0}^{k}-w^{\ast}\|^{2})
\end{align*}
\end{proof}
%
A similar result can be extended to a class of strongly convex objectives.
For a general convex function $F_{k}$, define $L_{1}^{k}=\inf\{c\mid\mathbb{E}\|\nabla F_{k}(w,\xi)\|^{2}\leq2c(F_{k}(w)-F_{k}^{\ast}),\forall w\}$.
All other quantities are defined in the same way as before. 
\begin{theorem}
(FedAvg with MaSS, Strongly convex objective) Suppose there exists
a $\frac{1}{L}$-strongly convex and $\frac{1}{\mu}$-smooth non-negative
function $g$ such that $g(w^{\ast})=0$ and $\langle g(x),\nabla f(z)\rangle\geq(1-\epsilon)\langle x-w^{\ast},z-w^{\ast}\rangle$,
$\forall x,z\in\mathbb{R}^{d}$, for some $\epsilon>0$. Let the hyperparameters
satisfy 
\begin{align*}
\eta^{k}(m)=\frac{1}{2L_{m}}, & \alpha^{k}(m)=\frac{1-\epsilon}{2\kappa_{m}},\delta^{k}(m)=\frac{1}{2L_{m}}
\end{align*}
for all $k$. Then the FedAvg algorithm with MaSS and full participation
solves the problem 
\begin{align*}
F(w) & =\sum_{k=1}^{N}p_{k}F_{k}(w)
\end{align*}
with exponential convergence
\begin{align*}
\mathbb{E}F(\overline{w}_{t})\leq\frac{L}{2}\mathbb{E}\|\overline{w}_{t}-w^{\ast}\|^{2} & \leq C\cdot(1-\frac{1-\epsilon}{2\kappa_{m}})^{t}=O(\exp(-\frac{1-\epsilon}{2\kappa_{m}}t))
\end{align*}
 where $w^{\ast}$ is such that $F(w^{\ast})=0$ and $C=\frac{L}{2}\cdot\frac{1}{\frac{\delta}{2\alpha}(1-\epsilon)}\sum_{k=1}^{N}p_{k}(\|v_{0}^{k}-w^{\ast}\|_{H_{k}^{-1}}^{2}+\frac{\delta}{2\alpha}(1-\epsilon)\|w_{0}^{k}-w^{\ast}\|^{2})$.
With partial participation of $K$ devices each communication round,
the convergence rate is also 
\begin{align*}
\mathbb{E}F(\overline{w}_{t})\leq\frac{L}{2}\mathbb{E}\|\overline{w}_{t}-w^{\ast}\|^{2} & \leq C\cdot(1+\frac{4}{K}\cdot(\frac{t}{E}-1))\cdot(1-\frac{1-\epsilon}{2\kappa_{m}})^{t}=O(\exp(-\frac{1-\epsilon}{2\kappa_{m}}t))
\end{align*}
\end{theorem}

\section{Convergence Results for Non-smooth Objectives}

\subsection{SGD}
\subsubsection{Strongly Convex Non-smooth}
\label{sec:sgdscvxnonsmth}
% !TEX ROOT=./main.tex

In this section, we prove the convergence rate of stochastic for strongly convex
non-smooth problem.


\begin{theorem}
	Let assumption~\ref{ass:stroncvx} and assumption~\ref{ass:subgrad2} hold, choose learning rate $\eta_t = \frac{2}{\mu(t+1)} $.Then,
	\begin{align}
		\EE[F(\hat{\vw}_T)] - F^* \leq \frac{2(B + C)}{\mu(T+1)}.
	\end{align}
	where $B =  G^2 (3  + 8 (E-1)^2)$, for sampling scheme I $C =\frac{4(N - K)}{K(N-1)} \eta_t^2 E^2G^2 $ or
sampling scheme II $C = \frac{4}{K} \eta_t^2 E^2G^2$, for full device participation $C= 0$.
\end{theorem}


\begin{proof}

\textbf{Full Device activation: }  $\ov{w}_{t+1} = \ov{v}_{t+1}, \forall t$ 
\begin{align}
\Delta_{t+1} = \EE\left[\left\|\bar{\vv}_{t+1}-\vw^{*}\right\|^2| \vw_{t}\right]=\|\bar{\vw}_{t}-\vw^{*}\|^{2} \underbrace{ - 2 \eta_{t} \EE \vg_{t}^{\top}\left(\bar{\vw}_{t}-\vw^{*}\right)}_{B_1} + \eta_{t}^{2} \EE\| \vg_{t} \|^{2}	
\label{eq:expand2}
\end{align}


Now we focus on bounding $B_1$ in \eq{\ref{eq:expand}}: 
\begin{align}
	& -2 \eta_{t} \left<\EE \vg_{t}, \bar{\vw}_{t}-\vw^{*} \right>\\
  = & 2 \eta_{t} \sum_{k=1}^N p_k \left(-\left<\EE \vg_{t,k}, \bar{\vw}_{t}-\vw_t^k + \vw_t^k - \vw^{*} \right> \right) \\
  \leq& 2 \eta_{t} \sum_{k=1}^N p_k  \left(\frac{1}{2\eta_t} \|\overline{\vw}_t - \vw_t^k\|^2 + \frac{\eta_t}{2} \|\EE \vg_{t,k}\|^2 - F_k(\vw_t^k) - F_k(\vw^*) - \frac{\mu}{2} \|\vw_t^k - \vw^*\|^2 \right) \\
  =& \sum_{k=1}^N p_k \|\overline{\vw}_t - \vw_t^k\|^2 +  \sum_{k=1}^N p_k \eta_t^2 \|\EE \vg_{t,k}\|^2 
  - 2\eta_t\sum_{k=1}^N p_k (F_k(\vw_t^k) - F^*) - \eta_t \mu \sum_{k=1}^N p_k \|\vw_t^k - \vw^*\|^2 \\
  \leq & \sum_{k=1}^N p_k \|\overline{\vw}_t - \vw_t^k\|^2 +  \sum_{k=1}^N p_k \eta_t^2 \|\EE \vg_{t,k}\|^2 
  - 2\eta_t\sum_{k=1}^N p_k (F_k(\vw_t^k) - F^*) - \eta_t \mu \|\ov{w}_t- \vw^*\|^2  \label{eq:scvxnsmth1}
\end{align}
where in the third line we use Cauchy-Schwarz inequality, AM-GM inequality and strong convexity.
In the last line we use convexity of l2 norm and Jensen's inequality.

Now we plug in $B_1$ into \eq{\ref{eq:expand2}}, we have,
\begin{align}
	\EE\left\|\bar{\vv}_{t+1}-\vw^{*}\right\|^2 & \leq (1 - \eta_t\mu)\|\bar{\vw}_{t}-\vw^{*}\|^{2} + \eta_{t}^{2} \EE\| \vg_{t} \|^{2} +  \sum_{k=1}^N p_k \eta_t^2 \|\EE \vg_{t,k}\|^2 \\
	& \underbrace{-2\eta_t\sum_{k=1}^N p_k (F_k(\vw_t^k) - F^*)}_{C_1} + \sum_{k=1}^N p_k \|\overline{\vw}_t - \vw_t^k\|^2 \label{eq:scvxnsmth2}
\end{align}

Now we bound $C_1$
\begin{align}
	-2\eta_t\sum_{k=1}^N p_k (F_k(\vw_t^k) - F^*) & = -2\eta_t\sum_{k=1}^N p_k \left(F_k(\vw_t^k) - F_k(\ov{w}_t)\right) + F(\ov{w}_t) -  F^*\\
& \leq \sum_{k=1}^{k} p_{k}[\eta_{t}^{2}\|\EE \vg_{t,k}(\ov{w}_{t})\|^{2}+\| \vw_{t}^{k}-\ov{w}_t^{k} \|^{2}]-2 \eta_{t}(F(\ov{w}_{t})- F^{*}). 
\end{align}

Plug in $C_1$ into \eq{\ref{eq:scvxnsmth2}}:
\begin{align}
	\EE\left\|\bar{\vv}_{t+1}-\vw^{*}\right\|^2 \leq
	&  (1 - \eta_t\mu)\|\bar{\vw}_{t}-\vw^{*}\|^{2} + \eta_{t}^{2} \EE\| \vg_{t} \|^{2} +  \sum_{k=1}^N p_k \eta_t^2 \|\EE \vg_{t,k}\|^2 \\& + \sum_{k=1}^{k} p_{k}\eta_{t}^{2}\|\EE \vg_{t,k}(\ov{w}_{t})\|^{2}-2 \eta_{t}(F(\ov{w}_{t})- F^{*})  + 2 \underbrace{\sum_{k=1}^N p_k \|\overline{\vw}_t - \vw_t^k\|^2}_{C_2} \label{eq:scvxnsmth3}
\end{align}
$C_2 \leq  4\eta_t^2 (E-1)^2 G^2$ can be bounded by Lemma 3 in \cite{li2019convergence}, we plug in $C_2$ into \eq{\ref{eq:scvxnsmth3}},


\begin{align}
	\underbrace{\EE\left\|\bar{\vv}_{t+1}-\vw^{*}\right\|^2}_{\Delta_{t+1}} \leq &  (1 - \eta_t\mu)\|\bar{\vw}_{t}-\vw^{*}\|^{2} -2 \eta_{t}(F(\ov{w}_{t})- F^{*})\\
	+ & \underbrace{\eta_{t}^{2} \EE\| \vg_{t} \|^{2} +  \sum_{k=1}^N p_k \eta_t^2 \|\EE \vg_{t,k}\|^2  + \sum_{k=1}^{k} p_{k}\eta_{t}^{2}\|\EE \vg_{t,k}(\ov{w}_{t})\|^{2} + 8\eta_t^2 (E-1)^2 G^2}_{\eta_t^2 B} \label{eq:scvxnsmth4} \\
	\Delta_{t+1}  \leq &  (1 - \eta_t\mu) \Delta_t + \eta^2_t B - 2 \eta_t [F(\ov{w}_{t})- F^{*}] \label{eq:scvxnsmth5}
\end{align}
where 
\begin{align*}
\eta_t^2 B &= \eta_{t}^{2} \EE\| \vg_{t} \|^{2} +  \sum_{k=1}^N p_k \eta_t^2 \|\EE \vg_{t,k}\|^2  + \sum_{k=1}^{k} p_{k}\eta_{t}^{2}\|\EE \vg_{t,k}(\ov{w}_{t})\|^{2} + 8\eta_t^2 (E-1)^2 G^2 \\
& \leq \eta_t^2 G^2 (3  + 8 (E-1)^2) 	
\end{align*}

Rearrange, set $\eta_t = \frac{2}{\mu(t+1)} $, using the weighted average trick in \cite{lacoste2012simpler}, and sum two sides of \eq{\ref{eq:scvxnsmth5}}, we have
\begin{align}
F\left(\ov{w}_{t}\right)-F^{*} &\leqslant \left(\frac{1}{2 \eta_{t}}-\frac{\mu}{2}\right) \Delta t-\frac{\Delta_{t+1}}{2 \eta_{t}}+\frac{\eta_{t}}{2} B\\
t(F\left(\ov{w}_{t}\right)-F^{*}) & \leqslant t\left(\frac{1}{2 \eta_{t}}-\frac{\mu}{2}\right) \Delta t-\frac{t\Delta_{t+1}}{2 \eta_{t}}+\frac{\eta_{t}}{2} Bt\\
t(F\left(\ov{w}_{t}\right)-F^{*}) & \leqslant \frac{\mu t(t-1)}{4} \Delta t- \frac{\mu t(t+1)\Delta_{t+1}}{4}+\frac{t}{\mu(t+1)} B\\
t(F\left(\ov{w}_{t}\right)-F^{*}) & \leqslant \frac{\mu t(t-1)}{4} \Delta t- \frac{\mu t(t+1)\Delta_{t+1}}{4}+\frac{B}{\mu}\\
\sum_{t=1}^T \frac{2t}{T(T+1)}F\left(\ov{w}_{t}\right)-F^{*} & \leqslant \frac{2B}{\mu(T+1)} \\
F\left(\sum_{t=1}^T \frac{2t}{T(T+1)} \ov{w}_{t}\right)-F^{*} & \leqslant \frac{2B}{\mu(T+1)} 
\end{align}

\textbf{Partial Device activation: }

If $t + 1 \notin \cI_{E}$, $\ov{w}_{t+1} = \ov{v}_{t+1}$, 
\begin{align}
\EE\left[\left\|\ov{w}_{t+1}-\vw^{*}\right\|^2| \ov{w}_{t}\right] = \EE\left[\left\|\bar{\vv}_{t+1}-\vw^{*}\right\|^2| \vw_{t}\right]=\|\bar{\vw}_{t}-\vw^{*}\|^{2} \underbrace{ - 2 \eta_{t} \EE \vg_{t}^{\top}\left(\bar{\vw}_{t}-\vw^{*}\right)}_{B_1} + \eta_{t}^{2} \EE\| \vg_{t} \|^{2}	
\label{eq:expandp1}
\end{align}

If $t + 1 \in \cI_{E}$, $\ov{w}_{t+1} = \frac{1}{K}\sum_{l=1}^K \vv_{t+1}^{i_l}$, 
\begin{align}
\EE\left\|\ov{w}_{t+1}-\vw^{*}\right\|^2 & =  \EE\left\|\ov{w}_{t+1} - \ov{v}_{t+1} + \ov{v}_{t+1} -\vw^{*}\right\|^2 \\
& \leq \EE\left\|\ov{w}_{t+1} - \ov{v}_{t+1}\right\|^2 + \EE \left\| \ov{v}_{t+1} -\vw^{*}\right\|^2 \\
&= \|\bar{\vw}_{t}-\vw^{*}\|^{2} \underbrace{ - 2 \eta_{t} \EE \vg_{t}^{\top}\left(\bar{\vw}_{t}-\vw^{*}\right)}_{B_1} + \eta_{t}^{2} \EE\| \vg_{t} \|^{2}	+ \EE\left\|\ov{w}_{t+1} - \ov{v}_{t+1}\right\|^2
\label{eq:expandp2}
\end{align}


Similar to the full device derivation:
\begin{align}
F\left(\sum_{t=1}^T \frac{2t}{T(T+1)} \ov{w}_{t}\right)-F^{*} & \leqslant \frac{2(B + C)}{\mu(T+1)} 	
\end{align}
where $B =  G^2 (3  + 8 (E-1)^2)$ and $C =\frac{4(N - K)}{K(N-1)} \eta_t^2 E^2G^2 $ or
$C = \frac{4}{K} \eta_t^2 E^2G^2$ depends on the sampling scheme.



\end{proof}
\subsubsection{Convex Non-smooth}
\label{sec:sgdcvxnonsmth}
% !TEX ROOT=./main.tex


% \textbf{Full Device Participation}

% \begin{itemize}
% 	\item Stochastic Subgradient of device $k$ at time step $t$, at point $w_{t,k}$: 
% 	$$\vg_{t,k} \coloneqq \vg_{t,k}(w_t^k)$$
% 	$$ \vg_{t, k} \in \partial F_{k}\left(w_{t}^{k}, \xi_{t}^{k}\right) $$
% 	$$\EE \vg_{t, k} \in \partial F_{k}\left(w_{t}^{k}\right)$$
% \item  One-step stochastic subgradient of all devices.
% $$\vg_{t}=\sum_{k=1}^{N} p_{k} \vg_{t, k}\left(w_{t}^{k}\right) $$
% \begin{align}
% 	\EE \vg_{t}= \EE \sum_{k=1}^{N} p_{k} \vg_{t, k}\left(w_{t}^{k}\right) \coloneqq \sum_{k=1}^{N} p_{k} \EE \vg_{t, k}
% 	\label{eq:egt}
% \end{align}
% \end{itemize}

\begin{assumption}
The expected squared norm of stochastic subgradients is uniformly bounded. i.e.,
$\mathbb{E}\|\vg_{t,k}\|^2  \leq G_k^{2}$, for all $k = 1,..., N$ and $t=0, \dots, T-1$.  This also implies $\left\| \mathbb{E}\vg_{t,k}\right\|^2  \leq \mathbb{E}\|\vg_{t,k}\|^2 \leq G_k^2$. Denote $G^2 = \sum_{k=1}^N p_k G_k^2$
\label{ass:subgrad2}
\end{assumption}

\begin{theorem}
	Let Assumption~\ref{ass:subgrad2} hold, 
	then the FedAve with full device participation satisfies
	\begin{align}
		 F^*_t - F^* &\leq \frac{RL}{\sqrt{T}}
	\end{align}
	where $\eta_t = \frac{R}{\sqrt{T}},$
	$F^*_t \coloneqq \min_{t \in [0, T-1]} F(\overline{\vw}_t),$
	$R = \sqrt{ \frac{\Delta_0}{L}},$
	$L=\sum_{k=1}^N \left( G^2 \left(3 + 4(E-1)^2\right)\right).$
	\label{th:cvxnonsmoth}
\end{theorem}
\begin{proof}
% In full device active setting, we have $\overline{\vw}_{t+1}=\overline{\vv}_{t+1}$,
Follow the same reasoning in the proof of Theorem~\ref{th:sgdcvxsmth} \eq{\ref{eq:sgdcvxsmth1}}
$$\EE_{\cS_{t+1}, \xi_{t}} \|\ov{w}_{t+1} - \vw^*\|^2 \leq  \eta_t^2 C + \EE_{\xi_t} \|\overline{\vv}_{t+1} - \vw^*\|^2 $$
where $C$ is defined in \eq{\ref{eq:partialsample}}. For convenience, we denote $\Delta_{t+1} = \EE \|\overline{\vw}_{t+1} - \vw^*\|^2$. We now focus on bounding $\left\|\ov{v}_{t+1}-\vw^{*}\right\|^2$.
According to the definition of $\overline{\vv}_{t+1}$ in \eq{\ref{eq:vbar}}, we have,
$$\begin{aligned}\left\|\bar{\vv}_{t+1}-\vw^{*}\right\|^2 &=\left\|\bar{\vw}_{t}-\eta_{t} \vg_{t}-\vw^{*}\right\|^2 \\ &=\left\|\bar{\vw}_{t}-\vw^{*}\right\|^{2}-2 \eta_{t} \vg_{t}^{\top}\left(\bar{\vw}_{t}-\vw^{*}\right)+\eta_{t}^{2}\left\|\vg_{t}\right\|^{2} \end{aligned}$$

Take the expectation condition on $\vw_t$ over random samples at all devices:
\begin{align}
\EE\left[\left\|\bar{\vv}_{t+1}-\vw^{*}\right\|^2| \vw_{t}\right]=\|\bar{\vw}_{t}-\vw^{*}\|^{2}-2 \eta_{t} \EE \vg_{t}^{\top}\left(\bar{\vw}_{t}-\vw^{*}\right)+\eta_{t}^{2} \EE\| \vg_{t} \|^{2}	
\label{eq:expand}
\end{align}
Note that since $\EE \vg_t = \EE \sum_{k=1}^{N} p_{k} \vg_{t, k}\left(\vw_{t}^{k}\right) \neq \EE \sum_{k=1}^Np_k \vg_{t,k}(\overline{\vw}_t)$, therefore $\EE \vg_t \notin \partial F(\overline{\vw}_t)$, we cannot directly use the convex definition to
upper bound $- \EE \vg_{t}^{\top}\left(\bar{\vw}_{t}-\vw^{*}\right)$ in \eq{\ref{eq:expand}}, as classic SGD proof did. 

Now we focus on bounding $-2 \eta_{t} \EE \vg_{t}^{\top}\left(\bar{\vw}_{t}-\vw^{*}\right)$ in \eq{\ref{eq:expand}}: 
\begin{align}
	& -2 \eta_{t} \left<\EE \vg_{t}, \bar{\vw}_{t}-\vw^{*} \right>\\
  = & 2 \eta_{t} \sum_{k=1}^N p_k \left(-\left<\EE \vg_{t,k}, \bar{\vw}_{t}-\vw^{*} \right> \right) \\
  \leq& 2 \eta_{t} \sum_{k=1}^N p_k  \left(\frac{1}{2\eta_t} \|\overline{\vw}_t - \vw_t^k\|^2 + \frac{\eta_t}{2} \|\EE \vg_{t,k}\|^2 - (F_k(\vw_t^k) - F_k(\vw^*) \right) \\
  =& \sum_{k=1}^N p_k \|\overline{\vw}_t - \vw_t^k\|^2 +  \sum_{k=1}^N p_k \eta_t^2 \|\EE \vg_{t,k}\|^2 
  - 2\eta_t\sum_{k=1}^N p_k (F_k(\vw_t^k) - F_k(\vw^*)) \\
  =& \sum_{k=1}^N p_k \|\overline{\vw}_t - \vw_t^k\|^2 +  \sum_{k=1}^N p_k \eta_t^2 \|\EE \vg_{t,k}\|^2 
  - 2\eta_t\sum_{k=1}^N p_k (F_k(\vw_t^k) - F^*) \label{eq:exp4}
\end{align}
where in the third line we use the following derivation:
\begin{align}
	& -\left<\EE \vg_{t,k}, \bar{\vw}_{t} - \vw^{*} \right>\\
=& - \left<\EE \vg_{t,k}, \bar{\vw}_{t} - \vw_t^k +  \vw_t^k - \vw^{*} \right>\\
=& - \left<\EE \vg_{t,k}, \bar{\vw}_{t} - \vw_t^k\right> - \left<\EE \vg_{t,k}, \vw_t^k - \vw^{*} \right>\\
\leq & - \left<\EE \vg_{t,k}, \bar{\vw}_{t} - \vw_t^k\right> - (F_k(\vw_t^k) - F_k(\vw^*)) \label{eq:3}\\
\leq & \frac{1}{2\eta_t} \|\bar{\vw}_t - \vw_t^k\|^2 + \frac{\eta_t}{2} \|\EE \vg_{t,k}\|^2
- (F_k(\vw_t^k) - F_k(\vw^*)) \label{eq:4}
\end{align}
where in \eq{\ref{eq:3}} we use the definition of convexity and in \eq{\ref{eq:4}} we use Cauchy-Schwarz inequality and AM-GM inequality $-\langle\va, \vb\rangle=\langle\va,-\vb\rangle \leqslant|\va||\vb| \leqslant \frac{1}{2}\left(|\va|^{2}+|\vb|^{2}\right)$.


Now we plug in \eq{\ref{eq:exp4}} into \eq{\ref{eq:expand}}:
\begin{align}
	\Delta_{t+1} \leq& \|\bar{\vw}_{t}-\vw^{*}\|^{2} + \eta_{t}^{2} \EE\| \vg_{t} \|^{2}  + \eta_t^2 C \\
	& + \underbrace{\sum_{k=1}^N p_k \|\overline{\vw}_t - \vw_t^k\|^2}_{A_1} +  \sum_{k=1}^N p_k \eta_t^2 \|\EE \vg_{t,k}\|^2
  - 2\eta_t\underbrace{\sum_{k=1}^N p_k (F_k(\vw_t^k) - F^*)}_{A_2} \label{eq:main2}
\end{align}

$A_1$ can be bounded by Lemma 3 in~\cite{li2019convergence}, 


Now we bound $A_2$:
\begin{align}
	& \sum_{k=1}^N p_k (F_k(\vw_t^k) - F^*) \\
 = & \sum_{k=1}^N p_k (F_k(\vw_t^k) - F_k(\overline{w}_t) + F_k(\overline{w}_t) - F^*) \\
 = & \sum_{k=1}^N p_k (F_k(\vw_t^k) - F_k(\overline{w}_t) ) + (F(\overline{w}_t) - F^*)\\ 
 \geq & \sum_{k=1}^{N} P_{k}\left<\EE g_{t k}\left(\overline{\vw}_{t}\right), \vw_{t}^k-\overline{\vw}_{t}\right> + (F(\overline{w}_t) - F^*)\\
 \geq& -\frac{1}{2} \sum_{k=1}^{N} P_{k} (\eta_t \|\EE \vg_{t,k}(\overline{\vw}_t) \|^2 + \frac{1}{\eta_t} \|\vw_t^k - \overline{\vw}_t\|^2) + (F(\overline{w}_t) - F^*) \label{eq:a24}
\end{align}
where in fourth line we use Cauchy-Schwarz inequality and AM-GM inequality. 


Now we plug in \eq{\ref{eq:a24}} into \eq{\ref{eq:main2}}:
\begin{align}
	\Delta_{t+1} & \leq \Delta_t + 
	\eta_{t}^{2} \EE\| \vg_{t} \|^{2} + \sum_{k=1}^N p_k \|\overline{\vw}_t - \vw_t^k\|^2 +  \sum_{k=1}^N p_k \eta_t^2 \|\EE \vg_{t,k}\|^2 + \eta_t^2 C \\
	 &- 2 \eta_t \left(-\frac{1}{2} \sum_{k=1}^{N} P_{k} (\eta_t \|\EE \vg_{t,k}(\overline{\vw}_t) \|^2 + \frac{1}{\eta_t} \|\vw_t^k - \overline{\vw}_t\|^2) + (F(\overline{w}_t) - F^*) \right)  + \eta_t^2 C \\
	& \leq \Delta_t + 
	\eta_{t}^{2} \EE\| \vg_{t} \|^{2} + \sum_{k=1}^N p_k \|\overline{\vw}_t - \vw_t^k\|^2 +  \sum_{k=1}^N p_k \eta_t^2 \|\EE \vg_{t,k}\|^2  + \eta_t^2 C  \\
	& +\sum_{k=1}^{N} P_{k} (\eta_t^2 \|\EE \vg_{t,k}(\overline{\vw}_t) \|^2 + \|\vw_t^k - \overline{\vw}_t\|^2) - 2\eta_t (F(\overline{w}_t) - F^*) \\
	& = \Delta_t + \underbrace{\eta_{t}^{2} \EE\| \vg_{t} \|^{2} +\sum_{k=1}^N p_k  ( 2\|\overline{\vw}_t - \vw_t^k\|^2  +  \eta_t^2 \|\EE \vg_{t,k}\|^2 + \eta_t^2 \|\EE \vg_{t,k}(\overline{\vw}_t) \|^2 ) + \eta_t^2 C }_{C_1 = \eta_t^2L}  - 2\eta_t (F(\overline{w}_t) - F^*) \label{eq:main3}
\end{align}
Rearrange the last inequality:
\begin{align}
	2\eta_t (F(\overline{w}_t) - F^*) \leq \Delta_t - \Delta_{t+1} + \eta_t^2 L
\end{align}
Sum over two sides from $t=0$ to $t = T- 1$ and define $F^*_t \coloneqq \min_{t \in [0, T-1]} F(\overline{\vw}_t)$
\begin{align}
	\sum_{t=0}^{T-1} 2\eta_t(F(\overline{w}_t) - F^*) & \leq \Delta_0 - \Delta_T + \sum_{t=0}^{T-1} \eta_t^2 L\\
	\sum_{t=0}^{T-1} 2\eta_t(F(\overline{w}_t) - F^*) & \leq \Delta_0 + \sum_{t=0}^{T-1} \eta_t^2 L\\
    F^*_t - F^* &\leq \frac{\Delta_0 + \sum_{t=0}^{T-1} \eta_t^2L}{ \sum_{t=0}^{T-1}  2 \eta_t}\\
    F^*_t - F^* &\leq \frac{RL}{\sqrt{T}}
\end{align}
where in last line we let $R = \sqrt{ \frac{\Delta_0}{L}}, \eta_t = \frac{R}{\sqrt{T}}$

Now we discuss how to bound the term $C_1$ in \eq{\ref{eq:main3}}

Assume Assumption~\ref{ass:subgrad2}, $\eta_t$ is non-increasing, and $\eta_t \leq 2\eta_t+E$ for all $t\geq 0$, according to Lemma 3 in~\cite{li2019convergence}, we have


\begin{equation}
\mathbb{E}\left[\sum_{k=1}^{N} p_{k}\left\|\overline{\mathbf{w}}_{t}-\mathbf{w}_{k}^{t}\right\|^{2}\right] \leq \sum_{k=1}^N p_k4 \eta_{t}^{2}(E-1)^{2} G_k^{2}
	\label{eq:g4}
\end{equation}


Assume Assumption~\ref{ass:subgrad2}, we have 
\begin{equation}
	\eta_t^2 \|\EE \vg_{t,k}\|^2 \leq  \eta_t^2 G_k^2
	\label{eq:g3}
\end{equation}
and
\begin{equation}
	\eta_t^2 \|\EE \vg_{t,k}(\overline{\vw}_t)\|^2 \leq  \eta_t^2 G_k^2
	\label{eq:g2}
\end{equation}


Now the only term left in $C_1$ is $\eta_{t}^{2} \EE\| \vg_{t} \|^{2} $, follow the proof in Lemma 2~\cite{li2019convergence}: 
\begin{align}
	  \eta_{t}^{2} \EE\| \vg_{t} \|^{2} 
	\leq   \eta_{t}^{2} \EE\| \sum_{k=1}^N p_k g_{t,k} \|^{2} 
	\leq   \eta_{t}^{2} \sum_{k=1}^N p_k \EE\| g_{t,k} \|^{2} 
	\leq   \eta_{t}^{2} \sum_{k=1}^N p_k G_k^2  \label{eq:g1}
\end{align}
where in the third inequality we use the convexity of the l2 norm.

Combine \eq{\ref{eq:g4}},\eq{\ref{eq:g3}},\eq{\ref{eq:g2}}, \eq{\ref{eq:g1}},
we have 
\begin{align}
C_1 = \eta_t^2 L  = \eta_t^2 [\sum_{k=1}^N p_k G_k^2 \left(3 + 4(E-1)^2\right) + C]  = \eta_t^2 [C+ G^2 \left(3 + 4(E-1)^2\right)]
\end{align}

\end{proof}


\subsection{Accelerated SGD}
\subsubsection{Strongly Convex Non-smooth}
\label{sec:nasgdscvxnonsmth}
We show that FedAvg with Accelerated SGD has $O(1/T)$ rate under
$\mu$-strong convexity. The FedAv algorithm with Nesterov Accelerated
SGD (NASGD) follows the updates
\begin{align*}
y_{t+1}^{k} & =w_{t}^{k}-\alpha_{t}g_{t,k}\\
w_{t+1}^{k} & =\begin{cases}
y_{t+1}^{k}+\beta_{t}(y_{t+1}^{k}-y_{t}^{k}) & \text{if }t+1\notin\mathcal{I}_{E}\\
\sum_{k=1}^{N}p_{k}\left[y_{t+1}^{k}+\beta_{t}(y_{t+1}^{k}-y_{t}^{k})\right] & \text{if }t+1\in\mathcal{I}_{E}
\end{cases}
\end{align*}
where 
\begin{align*}
g_{t,k} & :=\nabla F_{k}(w_{t}^{k},\xi_{t}^{k})
\end{align*}
is the stochastic gradient. 

Define the virtual sequences $\overline{y}_{t}=\sum_{k=1}^{N}p_{k}y_{t}^{k}$,
$\overline{w}_{t}=\sum_{k=1}^{N}p_{k}w_{t}^{k}$, and $\overline{g}_{t}=\sum_{k=1}^{N}p_{k}\mathbb{E}g_{t,k}$.
We have $\mathbb{E}g_{t}=\overline{g}_{t}$ and $\overline{y}_{t+1}=\overline{w}_{t}-\alpha_{t}g_{t}$,
and $\overline{w}_{t+1}=\overline{y}_{t+1}+\beta_{t}(\overline{y}_{t+1}-\overline{y}_{t})$. 
\begin{thm}
	(Full device participation) Suppose $F_{k}$ is $\mu$-strongly convex
	for all $k$. Let $E$ be the communication interval, and learning
	rates 
	\begin{align*}
	\alpha_{t} & =\frac{c}{\mu(E+t)}\\
	\beta_{t} & \leq\alpha_{t}
	\end{align*}
	so that $\alpha_{t}\leq2\alpha_{t+E}$, and where $c$ is small enough
	such that the following hold: 
	\begin{align*}
	\alpha_{t}^{2}+\beta_{t-1}^{2} & \leq\frac{1}{2}\\
	4\alpha_{t-1}^{2} & \leq\alpha_{t}
	\end{align*}
	for all $t\geq0$. Suppose also that $G$ is a constant satisfying
	$\mathbb{E}\|w_{0}-\alpha_{0}g_{0,k}\|^{2}=\mathbb{E}\|w_{0}-\alpha_{0}\nabla F_{k}(w_{0},\xi_{0}^{k})\|^{2}\leq G^{2}$
	for all $k$, and $\mathbb{E}\|\nabla F_{k}(w,\xi_{t}^{k})\|^{2}\leq G^{2}$
	for $w=\overline{w}_{t}$ or $w=w_{t}^{k}$ and all $t,k$.
	
	Then with full device participation, \textbf{
		\begin{align*}
		F(\sum_{t=1}^{T}\frac{2t}{T(T+1)}\overline{w}_{t})-F^{\ast} & \leq\frac{2B'}{\mu(T+1)}
		\end{align*}
	} where
	\begin{align*}
	B' & =6G^{2}+32(E-1)^{2}G^{2}+2K^{2}
	\end{align*}
	and $K$ is such that 
	\begin{align*}
	\alpha_{0}B+2K\cdot G & \leq\mu K^{2}\\
	B & =6G^{2}+32(E-1)^{2}G^{2}
	\end{align*}
	and
	\begin{align*}
	K & \geq\max\{\|w_{0}-w^{\ast}\|^{2},\frac{G}{2\alpha_{0}}\}
	\end{align*}
\end{thm}
\begin{proof}
	First, we derive a bound on $\mathbb{E}\|\overline{y}_{t+1}-\overline{y}_{t}\|^{2}$
	that is useful in the proof. We have the recursion 
	\begin{align*}
	y_{t+1}^{k}-y_{t}^{k} & =w_{t}^{k}-w_{t-1}^{k}-(\alpha_{t}g_{t,k}-\alpha_{t-1}g_{t-1,k})\\
	w_{t+1}^{k}-w_{t}^{k} & =-\alpha_{t}g_{t,k}+\beta_{t}(y_{t+1}^{k}-y_{t}^{k})
	\end{align*}
	so that 
	\begin{align*}
	y_{t+1}^{k}-y_{t}^{k} & =-\alpha_{t-1}g_{t-1,k}+\beta_{t-1}(y_{t}^{k}-y_{t-1}^{k})-(\alpha_{t}g_{t,k}-\alpha_{t-1}g_{t-1,k})\\
	& =\beta_{t-1}(y_{t}^{k}-y_{t-1}^{k})-\alpha_{t}g_{t,k}
	\end{align*}
	Since the identity $y_{t+1}^{k}-y_{t}^{k}=\beta_{t-1}(y_{t}^{k}-y_{t-1}^{k})-\alpha_{t}g_{t,k}$
	implies 
	\begin{align*}
	\mathbb{E}\|y_{t+1}^{k}-y_{t}^{k}\|^{2} & \leq2\beta_{t-1}^{2}\mathbb{E}\|y_{t}^{k}-y_{t-1}^{k}\|^{2}+2\alpha_{t}^{2}G^{2}
	\end{align*}
	as long as $\alpha_{t},\beta_{t}$ satisfy $2\beta_{t-1}^{2}+2\alpha_{t}^{2}\leq1$,
	and\textbf{ $\mathbb{E}\|w_{0}-\alpha_{0}g_{0,k}\|^{2}\leq G^{2}$},
	we can guarantee that $\mathbb{E}\|y_{t}^{k}-y_{t-1}^{k}\|^{2}\leq G^{2}$.
	This together with Jensen's inequality implies $\mathbb{E}\|\overline{y}_{t}-\overline{y}_{t-1}\|^{2}\leq G^{2}$. 
	
	Our main step is to prove the bound 
	\begin{align*}
	\mathbb{E}\|\overline{w}_{t+1}-w^{\ast}\|^{2} & \leq(1-\mu\alpha_{t})\mathbb{E}\|\overline{w}_{t}-w^{\ast}\|^{2}+9L\alpha_{t}^{2}\Gamma+\alpha_{t}^{2}\sum_{k=1}^{N}p_{k}^{2}\sigma_{k}^{2}+32\alpha_{t}^{2}(E-1)^{2}G^{2}\\
	& +2\alpha_{t}^{2}G^{2}+2\beta_{t}\mathbb{E}|\langle\overline{w}_{t}-w^{\ast},(\overline{y}_{t+1}-\overline{y}_{t})\rangle|\\
	& \leq(1-\mu\alpha_{t})\mathbb{E}\|\overline{w}_{t}-w^{\ast}\|^{2}+\alpha_{t}^{2}B'
	\end{align*}
	where $B'$ is define in the statement of the theorem. We have 
	\begin{align*}
	\|\overline{w}_{t+1}-w^{\ast}\|^{2} & =\|\overline{w}_{t}-w^{\ast}-\alpha_{t}g_{t}+\beta_{t}(\overline{y}_{t+1}-\overline{y}_{t})\|^{2}\\
	& =\|\overline{w}_{t}-w^{\ast}\|^{2}+2\langle\overline{w}_{t}-w^{\ast},\beta_{t}(\overline{y}_{t+1}-\overline{y}_{t})-\alpha_{t}g_{t}\rangle+\|\beta_{t}(\overline{y}_{t+1}-\overline{y}_{t})-\alpha_{t}g_{t}\|^{2}\\
	& \leq\|\overline{w}_{t}-w^{\ast}\|^{2}+2\langle\overline{w}_{t}-w^{\ast},\beta_{t}(\overline{y}_{t+1}-\overline{y}_{t})-\alpha_{t}g_{t}\rangle+2\|\beta_{t}(\overline{y}_{t+1}-\overline{y}_{t})\|^{2}+2\|\alpha_{t}g_{t}\|^{2}
	\end{align*}
	and the last two terms satisfy 
	\begin{align*}
	\mathbb{E}\left(2\|\beta_{t}(\overline{y}_{t+1}-\overline{y}_{t})\|^{2}+2\|\alpha_{t}g_{t}\|^{2}\right) & \leq4\alpha_{t}^{2}G^{2}
	\end{align*}
	sicne $\beta_{t}\leq\alpha_{t}$, $\|(\overline{y}_{t+1}-\overline{y}_{t})\|^{2}\leq G^{2}$,
	and $\mathbb{E}\|g_{t}\|^{2}=\mathbb{E}\|\sum_{k=1}^{N}p_{k}g_{t,k}\|^{2}\leq\mathbb{E}\sum_{k=1}^{N}p_{k}\|g_{t,k}\|^{2}\leq G^{2}$. 
	
	Now 
	\begin{align*}
	2\mathbb{E}\langle\overline{w}_{t}-w^{\ast},\beta_{t}(\overline{y}_{t+1}-\overline{y}_{t})-\alpha_{t}g_{t}\rangle & =2\beta_{t}\mathbb{E}\langle\overline{w}_{t}-w^{\ast},(\overline{y}_{t+1}-\overline{y}_{t})\rangle-2\alpha_{t}\mathbb{E}\langle\overline{w}_{t}-w^{\ast},g_{t}\rangle
	\end{align*}
	and we first bound $-2\alpha_{t}\mathbb{E}\langle\overline{w}_{t}-w^{\ast},g_{t}\rangle$.
	We have
	\begin{align*}
	-2\alpha_{t}\langle\overline{w}_{t}-w^{\ast},g_{t}\rangle & =-2\alpha_{t}\sum_{k=1}^{N}p_{k}\langle\overline{w}_{t}-w^{\ast},g_{t,k}\rangle\\
	& =-2\alpha_{t}\sum_{k=1}^{N}p_{k}\langle\overline{w}_{t}-w_{t}^{k},g_{t,k}\rangle-2\alpha_{t}\sum_{k=1}^{N}p_{k}\langle w_{t}^{k}-w^{\ast},g_{t,k}\rangle
	\end{align*}
	and again we bound the two terms separately. Using $\|\frac{1}{\sqrt{\alpha_{t}}}(\overline{w}_{t}-w_{t}^{k})-\sqrt{\alpha_{t}}g_{t,k}\|^{2}\geq0$,
	we have
	\begin{align*}
	-2\langle\overline{w}_{t}-w_{t}^{k},g_{t,k}\rangle & \leq\frac{1}{\alpha_{t}}\|\overline{w}_{t}-w_{t}^{k}\|^{2}+\alpha_{t}\|g_{t,k}\|^{2}
	\end{align*}
	
	Letting $\mathbb{E}_{t}$ denote the conditional expectation $\mathbb{E}\left[\cdot\mid\{w_{t}^{k}\}_{k=1}^{N}\right]$,
	i.e. the expectation with respect to the randomness of $\xi_{t}^{k}$s
	in the stochastic gradients, the $\mu$-strong convexity of $F_{k}$
	implies 
	\begin{align*}
	-\mathbb{E}_{t}\langle w_{t}^{k}-w^{\ast},g_{t,k}\rangle & =-\mathbb{E}_{t}\langle w_{t}^{k}-w^{\ast},g_{t}\rangle=-\mathbb{E}_{t}\langle w_{t}^{k}-w^{\ast},\nabla F_{k}(w_{t}^{k},\xi_{t}^{k})\rangle=-\langle w_{t}^{k}-w^{\ast},\nabla F_{k}(w_{t}^{k})\rangle\\
	& \leq-(F_{k}(w_{t}^{k})-F_{k}(w^{\ast}))-\frac{\mu}{2}\|w_{t}^{k}-w^{\ast}\|^{2}
	\end{align*}
	
	Combining the above, it follows that 
	\begin{align*}
	-2\alpha_{t}\mathbb{E}\langle\overline{w}_{t}-w^{\ast},g_{t}\rangle & \le\alpha_{t}\mathbb{E}\sum_{k=1}^{N}p_{k}\left[(\frac{1}{\alpha_{t}}\|\overline{w}_{t}-w_{t}^{k}\|^{2}+\alpha_{t}\|g_{t,k}\|^{2})-2(F_{k}(w_{t}^{k})-F_{k}(w^{\ast}))-\mu\|w_{t}^{k}-w^{\ast}\|^{2}\right]\\
	& =\mathbb{E}\sum_{k=1}^{N}p_{k}\|\overline{w}_{t}-w_{t}^{k}\|^{2}-\mu\alpha_{t}\mathbb{E}\sum_{k=1}^{N}p_{k}\|w_{t}^{k}-w^{\ast}\|^{2}+\alpha_{t}^{2}\mathbb{E}\sum_{k=1}^{N}p_{k}\|g_{t,k}\|^{2}\\
	& -2\alpha_{t}\mathbb{E}\sum_{k=1}^{N}p_{k}(F_{k}(w_{t}^{k})-F_{k}(w^{\ast}))\\
	& \leq\mathbb{E}\sum_{k=1}^{N}p_{k}\|\overline{w}_{t}-w_{t}^{k}\|^{2}-\mu\alpha_{t}\mathbb{E}\|\overline{w}_{t}-w^{\ast}\|^{2}+\alpha_{t}^{2}G^{2}-2\alpha_{t}\mathbb{E}\sum_{k=1}^{N}p_{k}(F_{k}(w_{t}^{k})-F_{k}(w^{\ast}))
	\end{align*}
	where we have again used Jensen's inequality on $\sum_{k=1}^{N}p_{k}\|w_{t}^{k}-w^{\ast}\|^{2}$. 
	
	Thus 
	\begin{align*}
	\mathbb{E}\|\overline{w}_{t+1}-w^{\ast}\|^{2} & \leq(1-\mu\alpha_{t})\mathbb{E}\|\overline{w}_{t}-w^{\ast}\|^{2}+\mathbb{E}\sum_{k=1}^{N}p_{k}\|\overline{w}_{t}-w_{k}^{t}\|^{2}-2\alpha_{t}\mathbb{E}\sum_{k=1}^{N}p_{k}(F_{k}(w_{t}^{k})-F_{k}(w^{\ast}))\\
	& +5\alpha_{t}^{2}G^{2}+2\beta_{t}\mathbb{E}\langle\overline{w}_{t}-w^{\ast},(\overline{y}_{t+1}-\overline{y}_{t})\rangle
	\end{align*}
	We note that 
	\begin{align*}
	\mathbb{E}\sum_{k=1}^{N}p_{k}\|\overline{w}_{t}-w_{k}^{t}\|^{2} & \leq16(E-1)^{2}\alpha_{t}^{2}G^{2}
	\end{align*}
	by the same argument as in the proof for the strongly convex and
	smooth case. Now we bound 
	\begin{align*}
	-2\alpha_{t}\mathbb{E}\sum_{k=1}^{N}p_{k}(F_{k}(w_{t}^{k})-F_{k}(w^{\ast})) & \leq-2\alpha_{t}\mathbb{E}\sum_{k=1}^{N}p_{k}(F_{k}(w_{t}^{k})-F_{k}(w^{\ast}))\\
	& =-2\alpha_{t}\mathbb{E}\sum_{k=1}^{N}p_{k}(F_{k}(w_{t}^{k})-F_{k}(\overline{w}_{k})+F(\overline{w}_{k})-F_{k}(w^{\ast}))\\
	& \leq-2\alpha_{t}\mathbb{E}\sum_{k=1}^{N}p_{k}(\langle w_{t}^{k}-\overline{w}_{k},\nabla F_{k}(\overline{w}_{k})\rangle+F(\overline{w}_{k})-F_{k}(w^{\ast}))\\
	& \leq\mathbb{E}\sum_{k=1}^{N}p_{k}\left[\alpha_{t}^{2}\|\nabla F_{k}(\overline{w}_{k})\|^{2}+\|w_{t}^{k}-\overline{w}_{t}^{k}\|^{2}\right]-2\alpha_{t}(F(\overline{w}_{t})-F^{\ast})\\
	& \leq\alpha_{t}^{2}G^{2}+\mathbb{E}\sum_{k=1}^{N}p_{k}\|\overline{w}_{t}-w_{k}^{t}\|^{2}-2\alpha_{t}\mathbb{E}(F(\overline{w}_{t})-F^{\ast})
	\end{align*}
	using convexity of $F_{k}$, the assumption that $\mathbb{E}\|\nabla F_{k}(\overline{w}_{k})\|^{2}\leq G^{2}$. 
	
	Thus 
	\begin{align*}
	\mathbb{E}\|\overline{w}_{t+1}-w^{\ast}\|^{2} & \leq(1-\mu\alpha_{t})\mathbb{E}\|\overline{w}_{t}-w^{\ast}\|^{2}+2\mathbb{E}\sum_{k=1}^{N}p_{k}\|\overline{w}_{t}-w_{k}^{t}\|^{2}-2\alpha_{t}\mathbb{E}(F(\overline{w}_{t})-F^{\ast})\\
	& +6\alpha_{t}^{2}G^{2}+2\beta_{t}\mathbb{E}\langle\overline{w}_{t}-w^{\ast},(\overline{y}_{t+1}-\overline{y}_{t})\rangle\\
	& \leq(1-\mu\alpha_{t})\mathbb{E}\|\overline{w}_{t}-w^{\ast}\|^{2}+\alpha_{t}^{2}B-2\alpha_{t}\mathbb{E}(F(\overline{w}_{t})-F^{\ast})+2\beta_{t}\mathbb{E}\langle\overline{w}_{t}-w^{\ast},(\overline{y}_{t+1}-\overline{y}_{t})\rangle
	\end{align*}
	where 
	\begin{align*}
	B & =6G^{2}+32(E-1)^{2}G^{2}
	\end{align*}
	
	Now we bound $|2\beta_{t}\mathbb{E}\langle\overline{w}_{t}-w^{\ast},(\overline{y}_{t+1}-\overline{y}_{t})\rangle|$.
	As in the proof for the strongly convex and smooth case, with appropriate
	choice of constant $K$ depending on the other constants, we first
	show that 
	\begin{align*}
	\mathbb{E}\|\overline{w}_{t+1}-w^{\ast}\|^{2} & \leq K^{2}
	\end{align*}
	for all $t$, i.e. the updates always stay in a large ball around
	the optimum during the Nesterov accelerated gradient descent. Note
	that $(F(\overline{w}_{t})-F^{\ast})\geq0$ and 
	\begin{align*}
	\beta_{t}\mathbb{E}\langle\overline{w}_{t}-w^{\ast},(\overline{y}_{t+1}-\overline{y}_{t})\rangle & \leq\beta_{t}\sqrt{\mathbb{E}\|\overline{w}_{t}-w^{\ast}\|^{2}}\cdot\sqrt{\mathbb{E}\|\overline{y}_{t+1}-\overline{y}_{t}\|^{2}}
	\end{align*}
	so that 
	\begin{align*}
	\mathbb{E}\|\overline{w}_{t+1}-w^{\ast}\|^{2} & \leq(1-\alpha_{t}\mu)\mathbb{E}\|\overline{w}_{t}-w^{\ast}\|^{2}+\alpha_{t}^{2}B+2\beta_{t}\sqrt{\mathbb{E}\|\overline{w}_{t}-w^{\ast}\|^{2}}\cdot\sqrt{\mathbb{E}\|\overline{y}_{t+1}-\overline{y}_{t}\|^{2}}\\
	& \leq(1-\alpha_{t}\mu)\mathbb{E}\|\overline{w}_{t}-w^{\ast}\|^{2}+\alpha_{t}^{2}B+2\beta_{t}\sqrt{\mathbb{E}\|\overline{w}_{t}-w^{\ast}\|^{2}}\cdot G
	\end{align*}
	and again we can conclude that $\mathbb{E}\|\overline{w}_{t}-w^{\ast}\|^{2}\leq K^{2}$
	for $t\geq0$, for $K$ satisfying 
	\begin{align*}
	\alpha_{0}B+2K\cdot G & \leq\mu K^{2}
	\end{align*}
	and
	\begin{align*}
	\|w_{0}-w^{\ast}\|^{2} & \leq K^{2}
	\end{align*}
	
	Finally, the bound on $\beta_{t}\mathbb{E}\langle\overline{w}_{t}-w^{\ast},(\overline{y}_{t+1}-\overline{y}_{t})\rangle$
	is exactly the same as in the proof for the strongly convex and smooth
	case, and we may conclude that 
	\begin{align*}
	\mathbb{E}\|\overline{w}_{t+1}-w^{\ast}\|^{2} & \leq(1-\mu\alpha_{t})\mathbb{E}\|\overline{w}_{t}-w^{\ast}\|^{2}+\alpha_{t}^{2}B'-2\alpha_{t}\mathbb{E}(F(\overline{w}_{t})-F^{\ast})
	\end{align*}
	where 
	\begin{align*}
	B' & =B+2K^{2}\\
	& =6G^{2}+32(E-1)^{2}G^{2}+2K^{2}
	\end{align*}
	
	To conclude the proof, we apply the averaging trick of the \textbf{TODO:CITE
		LACOSTE paper to get 
		\begin{align*}
		F(\sum_{t=1}^{T}\frac{2t}{T(T+1)}\overline{w}_{t})-F^{\ast} & \leq\frac{2B'}{\mu(T+1)}
		\end{align*}
	}
\end{proof}
%
We now move on to prove the result in the case of partial participation.
Now the FedAvg algorithm with Nesterov Accelerated SGD (NASGD) follows
the updates
\begin{align*}
y_{t+1}^{k} & =w_{t}^{k}-\alpha_{t}g_{t,k}\\
w_{t+1}^{k} & =\begin{cases}
y_{t+1}^{k}+\beta_{t}(y_{t+1}^{k}-y_{t}^{k}) & \text{if }t+1\notin\mathcal{I}_{E}\\
\sum_{k\in\mathcal{S}_{t+1}}\left(y_{t+1}^{k}+\beta_{t}(y_{t+1}^{k}-y_{t}^{k})\right) & \text{if }t+1\in\mathcal{I}_{E}
\end{cases}
\end{align*}
where $\mathcal{S}_{t+1}$ is the multiset obtained by sampling from
$[N]$ according to $p_{k}$, \emph{with replacement, }a total of
$S=|\mathcal{S}_{t+1}|$ times. 

As before we define the virtual sequences $\overline{y}_{t}=\sum_{k=1}^{N}p_{k}y_{t}^{k}$,
$\overline{w}_{t}=\sum_{k=1}^{N}p_{k}w_{t}^{k}$, and $\overline{g}_{t}=\sum_{k=1}^{N}p_{k}\mathbb{E}g_{t,k}$.
We have $\mathbb{E}g_{t}=\overline{g}_{t}$ and $\overline{y}_{t+1}=\overline{w}_{t}-\alpha_{t}g_{t}$,
and as in the full participation case, $\overline{w}_{t+1}=\overline{y}_{t+1}+\beta_{t}(\overline{y}_{t+1}-\overline{y}_{t})$
for $t+1\notin\mathcal{I}_{E}$. When $t+1$ is a communication round,
because of the sampling step, this identity is no longer true, but
is true when we take expectation with respect to the sampling distribution
$\mathbb{P}_{\mathcal{S}_{t+1}}$. 
\begin{thm}
	(Partial device participation) Let the parameters satisfy the assumptions
	in the ICLR paper, $\kappa=\frac{L}{\mu}$, $\gamma=\max\{8\kappa,E\}$
	and learning rate $\alpha_{t}=\frac{c}{\mu(\gamma+t)}$, $\beta_{t}\leq\alpha_{t}$
	where $c$ is small enough such that $\alpha_{t}^{2}+\beta_{t-1}^{2}\leq\frac{1}{2}$
	and $2\alpha_{t-1}^{2}\leq\alpha_{t}$ for all $t$. Then with the
	partial device participation scheme described above,
	\begin{align*}
	\mathbb{E}F(w_{T})-F^{\ast} & \leq\frac{2\kappa}{\gamma+T}(\frac{B'+C}{\mu}+2L(\|w_{0}-w^{\ast}\|^{2})\\
	B' & =\sum_{k=1}^{N}p_{k}^{2}\sigma_{k}^{2}+9L\Gamma+32(E-1)^{2}G^{2}+2+2G^{2}+2GK\\
	C & =\frac{16}{S}E^{2}G^{2}
	\end{align*}
	and $K$ is such that 
	\begin{align*}
	\alpha_{0}B+2\sqrt{K}\cdot G & \leq\mu K\\
	B & =\sum_{k=1}^{N}p_{k}^{2}\sigma_{k}^{2}+9L\Gamma+32(E-1)^{2}G^{2}
	\end{align*}
	and
	\begin{align*}
	\|w_{0}-w^{\ast}\|^{2} & \leq K
	\end{align*}
\end{thm}
%
\begin{proof}
	We have
	\begin{align*}
	\|\overline{w}_{t+1}-w^{\ast}\|^{2} & =\|\overline{w}_{t+1}-(\overline{y}_{t+1}+\beta_{t}(\overline{y}_{t+1}-\overline{y}_{t}))+(\overline{y}_{t+1}+\beta_{t}(\overline{y}_{t+1}-\overline{y}_{t}))-w^{\ast}\|^{2}\\
	& =\|\overline{w}_{t+1}-(\overline{y}_{t+1}+\beta_{t}(\overline{y}_{t+1}-\overline{y}_{t}))\|^{2}+\|(\overline{y}_{t+1}+\beta_{t}(\overline{y}_{t+1}-\overline{y}_{t}))-w^{\ast}\|^{2}\\
	& +2\langle\overline{w}_{t+1}-(\overline{y}_{t+1}+\beta_{t}(\overline{y}_{t+1}-\overline{y}_{t})),(\overline{y}_{t+1}+\beta_{t}(\overline{y}_{t+1}-\overline{y}_{t}))-w^{\ast}\rangle
	\end{align*}
	Note that 
	\begin{align*}
	\mathbb{E}_{\mathcal{S}_{t+1}}\langle\overline{w}_{t+1}-(\overline{y}_{t+1}+\beta_{t}(\overline{y}_{t+1}-\overline{y}_{t})),(\overline{y}_{t+1}+\beta_{t}(\overline{y}_{t+1}-\overline{y}_{t}))-w^{\ast}\rangle & =0
	\end{align*}
	since 
	\begin{align*}
	\mathbb{E}_{\mathcal{S}_{t+1}}\overline{w}_{t+1} & =\mathbb{E}_{\mathcal{S}_{t+1}}\frac{1}{S}\sum_{k\in\mathcal{S}_{t+1}}(y_{t+1}^{k}+\beta_{t}(y_{t+1}^{k}-y_{t}^{k}))\\
	& =\frac{1}{S}\sum_{k\in\mathcal{S}_{t+1}}\mathbb{E}_{\mathcal{S}_{t+1}}(y_{t+1}^{k}+\beta_{t}(y_{t+1}^{k}-y_{t}^{k}))\\
	& =\frac{1}{S}\cdot S\cdot\sum_{k=1}^{N}p_{k}(y_{t+1}^{k}+\beta_{t}(y_{t+1}^{k}-y_{t}^{k}))\\
	& =\overline{y}_{t+1}+\beta_{t}(\overline{y}_{t+1}-\overline{y}_{t})
	\end{align*}
	Moreover, if $t+1\notin\mathcal{I}_{E}$, $\|\overline{w}_{t+1}-(\overline{y}_{t+1}+\beta_{t}(\overline{y}_{t+1}-\overline{y}_{t}))\|^{2}=0$
	as well, while if $t+1\notin\mathcal{I}_{E}$, we show that 
	\begin{align*}
	\mathbb{E}\|\overline{w}_{t+1}-(\overline{y}_{t+1}+\beta_{t}(\overline{y}_{t+1}-\overline{y}_{t}))\|^{2} & =O(\alpha_{t}^{2})
	\end{align*}
	assuming $\eta_{t}$ is non-increasing and $\eta_{t}\leq2\eta_{t+E}$
	for all $t\geq0$. We have 
	\begin{align*}
	\mathbb{E}_{\mathcal{S}_{t+1}}\|\overline{w}_{t+1}-(\overline{y}_{t+1}+\beta_{t}(\overline{y}_{t+1}-\overline{y}_{t}))\|^{2} & =\mathbb{E}_{\mathcal{S}_{t+1}}\|\frac{1}{S}\sum_{k\in\mathcal{S}_{t+1}}(y_{t+1}^{k}+\beta_{t}(y_{t+1}^{k}-y_{t}^{k}))-(\overline{y}_{t+1}+\beta_{t}(\overline{y}_{t+1}-\overline{y}_{t}))\|^{2}\\
	& =\frac{1}{S^{2}}\sum_{k\in\mathcal{S}_{t+1}}\mathbb{E}_{\mathcal{S}_{t+1}}\|y_{t+1}^{k}+\beta_{t}(y_{t+1}^{k}-y_{t}^{k})-(\overline{y}_{t+1}+\beta_{t}(\overline{y}_{t+1}-\overline{y}_{t}))\|^{2}\\
	& =\frac{1}{S}\sum_{k=1}^{N}p_{k}\|y_{t+1}^{k}+\beta_{t}(y_{t+1}^{k}-y_{t}^{k})-(\overline{y}_{t+1}+\beta_{t}(\overline{y}_{t+1}-\overline{y}_{t}))\|^{2}
	\end{align*}
	where we have used the general fact that $\mathbb{E}\|\frac{1}{S}\sum_{k\in\mathcal{S}_{t+1}}x_{k}-\mathbb{E}x_{k}\|^{2}=\frac{1}{S^{2}}\sum_{k\in\mathcal{S}_{t+1}}\mathbb{E}\|x_{k}-\mathbb{E}x_{k}\|^{2}$
	for $x_{k}$ iid. Since $t+1\in\mathcal{I}_{E}$, we know that $t_{0}=t-E+1\in\mathcal{I}_{E}$
	is also a communication round, so that $w_{t_{0}}^{k}\equiv\overline{w}_{t_{0}}$
	does not depend on $k$. Then 
	\begin{align*}
	\sum_{k=1}^{N}p_{k}\|y_{t+1}^{k}+\beta_{t}(y_{t+1}^{k}-y_{t}^{k})-(\overline{y}_{t+1}+\beta_{t}(\overline{y}_{t+1}-\overline{y}_{t}))\|^{2} & =\sum_{k=1}^{N}p_{k}\|y_{t+1}^{k}+\beta_{t}(y_{t+1}^{k}-y_{t}^{k})-\overline{w}_{t_{0}}+\overline{w}_{t_{0}}-(\overline{y}_{t+1}+\beta_{t}(\overline{y}_{t+1}-\overline{y}_{t}))\|^{2}\\
	& \leq\sum_{k=1}^{N}p_{k}\|y_{t+1}^{k}+\beta_{t}(y_{t+1}^{k}-y_{t}^{k})-\overline{w}_{t_{0}}\|^{2}
	\end{align*}
	so that 
	\begin{align*}
	\mathbb{E}\|\overline{w}_{t+1}-(\overline{y}_{t+1}+\beta_{t}(\overline{y}_{t+1}-\overline{y}_{t}))\|^{2} & \leq\frac{1}{S}\sum_{k=1}^{N}p_{k}\mathbb{E}\|y_{t+1}^{k}+\beta_{t}(y_{t+1}^{k}-y_{t}^{k})-\overline{w}_{t_{0}}\|^{2}\\
	& =\frac{1}{S}\sum_{k=1}^{N}p_{k}\mathbb{E}\|y_{t+1}^{k}+\beta_{t}(y_{t+1}^{k}-y_{t}^{k})-w_{t_{0}}^{k}\|^{2}\\
	& =\frac{1}{S}\sum_{k=1}^{N}p_{k}\mathbb{E}\|\sum_{i=t_{0}}^{t}\beta_{i}(y_{i+1}^{k}-y_{i}^{k})-\sum_{i=t_{0}}^{t}\alpha_{i}g_{i,k}\|^{2}\\
	& \leq\frac{2}{S}\left[\sum_{k=1}^{N}p_{k}\mathbb{E}\sum_{i=t_{0}}^{t}E\alpha_{i}^{2}\|g_{i,k}\|^{2}+\sum_{k=1}^{N}p_{k}\mathbb{E}\sum_{i=t_{0}}^{t}E\beta_{i}^{2}\|(y_{i+1}^{k}-y_{i}^{k})\|^{2}\right]\\
	& \leq\frac{16}{S}\alpha_{t}^{2}E^{2}G^{2}
	\end{align*}
	Now we turn to $\mathbb{E}\|(\overline{y}_{t+1}+\beta_{t}(\overline{y}_{t+1}-\overline{y}_{t}))-w^{\ast}\|^{2}$.
	This term is bounded in the proof of the full participation case,
	with 
	\begin{align*}
	\mathbb{E}\|\overline{w}_{t+1}-w^{\ast}\|^{2} & \leq(1-\mu\alpha_{t})\mathbb{E}\|\overline{w}_{t}-w^{\ast}\|^{2}+\alpha_{t}^{2}B'
	\end{align*}
	where 
	\begin{align*}
	B' & =B+2G^{2}+2GK\\
	& =\sum_{k=1}^{N}p_{k}^{2}\sigma_{k}^{2}+9L\Gamma+32(E-1)^{2}G^{2}+2+2G^{2}+2GK
	\end{align*}
	and $K$ is chosen so that 
	\begin{align*}
	\alpha_{0}B+2\sqrt{K}\cdot G & \leq\mu K
	\end{align*}
	and
	\begin{align*}
	\|w_{0}-w^{\ast}\|^{2} & \leq K
	\end{align*}
	
	Now we can conclude that 
	\begin{align*}
	\mathbb{E}\|\overline{w}_{t+1}-w^{\ast}\|^{2} & \leq(1-\mu\alpha_{t})\mathbb{E}\|\overline{w}_{t}-w^{\ast}\|^{2}+\alpha_{t}^{2}(B'+C)
	\end{align*}
	where 
	\begin{align*}
	C & =\frac{16}{S}E^{2}G^{2}
	\end{align*}
	
	The exact argument then yields 
	\begin{align*}
	\mathbb{E}F(w_{T})-F^{\ast} & \leq\frac{2\kappa}{\gamma+T}(\frac{B'+C}{\mu}+2L(\|w_{0}-w^{\ast}\|^{2})
	\end{align*}
\end{proof}
\subsubsection{Convex Non-smooth}
\label{sec:nasgdcvxnonsmth}
% !TEX ROOT=./main.tex

In this section, we study the convergence of accelerated FedAve 
algorithm for convex, non-smooth function.

\begin{itemize}
	\item MGD vs NASGD
	\begin{align}
	\vy^k_{t+1} &= \gamma \vy^k_t + \grad F_k(\vw^k_{t})	\\
	\vw^k_{t+1} & = \vw^k_{t} - \eta \vy^k_{t+1}
	\end{align}
	\begin{align}
	\vy^k_{t+1} &= \vw^k_{t} - \alpha \grad F_k(\vw^k_{t})	\\
	\vw^k_{t+1} & = \vy^k_{t+1} + \beta(\vy^k_{t+1} - \vy^k_{t})
	\end{align}
	
\end{itemize}

\textbf{NASGD updates}
\begin{align}
	\vy^k_{t+1} &= \vw^k_{t} - \alpha  \vg_{t,k}	\\
	\vw^k_{t+1} & = \vy^k_{t+1} + \beta(\vy^k_{t+1} - \vy^k_{t})
\end{align}
$\vp_t$ satisfies the recursive equation:
\begin{align}
	\ov{p}_{t}=\left\{\begin{array}{l}\frac{\beta}{1-\beta}\left[\ov{w}_{t}-\ov{w}_{t-1}+\alpha \vg_{t-1}\right], t \geqslant 1 \\ 0, t=0\end{array}\right.
\end{align}
and 
\begin{align}
	\vp_{t+1} =  \beta \vp_t - \frac{\beta^2}{1 - \beta} \vg_t
	\label{eq:recursivep}
\end{align}
\begin{align}
	\underbrace{\overline{\vw}_{t+1} + \overline{\vp}_{t+1}}_{\overline{\vz}_{t+1}} &= \underbrace{\overline{\vw}_{t} + \overline{\vp}_{t}}_{\overline{\vz}_{t}} - \frac{\alpha}{1- \beta} \vg_t
	\label{eq:nasgdzt}
\end{align}

With the recursive equation \eq{\ref{eq:nasgdzt}}, we have the following 
inequality.
\begin{align}
	\EE\|\overline{\vz}_{t+1} - \vw^* \|^2  &= \EE\|\overline{\vz}_{t} - \eta_t\vg_t - \vw^* \|^2 \\
& \leq  \| \overline{\vz}_{t} - \vw^*\|^2  - 2\eta_t\left<\EE\vg_t, \overline{\vz}_{t} - \vw^*\right> +  \eta_t^2\EE\|\vg_t\|^2 \\
& \leq  \| \overline{\vz}_{t} - \vw^*\|^2  - 2\eta_t\sum_{k=1}^K p_k\left<\EE g_{t,k}, \overline{\vz}_{t} - \vw^*\right> +  \eta_t^2\EE\|\vg_t\|^2 \label{eq:nagcvx1}
\end{align}

where we set $\eta_t = \frac{\alpha}{1- \beta}$. Follow the same reasoning 
in \eq{\ref{eq:4}}, use convexity of the objective function, we have 
\begin{align}
   & - \left<\EE g_{t,k}, \overline{\vz}_{t} - \vw^*\right> \\
	\leq & \frac{1}{2\eta_t} \|\overline{\vz}_t - \vw_t^k\|^2 + \frac{\eta_t}{2} \|\EE \vg_{t,k}\|^2 - (F_k(\vw_t^k) - F_k(\vw^*)) \label{eq:nagcvx2}
\end{align}
Plug in \eq{\ref{eq:nagcvx2}} to \eq{\ref{eq:nagcvx1}}:
\begin{align}
	\EE\|\overline{\vz}_{t+1} - \vw^* \|^2 & \leq \| \overline{\vz}_{t} - \vw^*\|^2  +  \eta_t^2\EE\|\vg_t\|^2  \\
	&  + \sum_{k=1}^K p_k \underbrace{\|\overline{\vz}_t - \vw_t^k\|^2}_{A_1}  +\eta^2_t \sum_{k=1}^K p_k\|\EE \vg_{t,k}\|^2 \underbrace{- 2\eta_t\sum_{k=1}^K p_k(F_k(\vw_t^k) - F_k(\vw^*))}_{A_2} \label{eq:nagcvx3}
\end{align}
Bound $A_2$, use the convexity of $F_k$, same reasoning as \eq{\ref{eq:a24}}:
\begin{align*}
& -2\eta_t\sum_{k=1}^K p_k (F_k(\vw_t^k) - F_k(\vw^*)) \\
\leq &\sum_{k=1}^K p_k (\eta_t^2 \|\EE \vg_{t,k}(\overline{\vw}_t) \|^2 +\|\vw_t^k - \overline{\vw}_t\|^2) -2\eta_t (F(\overline{w}_t) - F^*)
\end{align*}
Plug in the upper bound of $A_2$ into the \eq{\ref{eq:nagcvx3}}, 
\begin{align}
	\EE\|\overline{\vz}_{t+1} - \vw^* \|^2 & \leq \| \overline{\vz}_{t} - \vw^*\|^2  +  \eta_t^2\EE\|\vg_t\|^2  \nonumber\\
	&  + \sum_{k=1}^K p_k \underbrace{\|\overline{\vz}_t - \vw_t^k\|^2}_{A_1}  +\eta^2_t \sum_{k=1}^K p_k\|\EE \vg_{t,k}\|^2 \nonumber \\
	& + \sum_{k=1}^K p_k (\eta_t^2 \|\EE \vg_{t,k}(\overline{\vw}_t)\|^2 + \|\vw_t^k - \overline{\vw}_t\|^2) - 2\eta_t(F(\overline{w}_t) - F^*) 
	\label{eq:nagcvx4}
\end{align}
Bound $\EE\|\overline{\vz}_t - \vw_t^k\|^2$ and $\EE \|\overline{\vw}_t- \vw_t^k \|^2$, this is similar to the proof in Lemma 3~\cite{li2019convergence}: 

\begin{align}
	\EE \|\ov{\vw}_t -\vw_t^k \|^2 & = \EE \|\ov{w}_t - \ov{w}_{t_0} + \ov{w}_{t_0}- \vw_t^k \|^2\\
% & \leq \EE\|\ov{w}_{t_0}- \vw_t^k \|^2 + \EE \|\ov{w}_t - \ov{w}_{t_0}\|^2\\
& \leq \EE\|\vw_t^k  - \ov{w}_{t_0}\|^2  \\
& \leq \EE \|\vz_t^k - \vp_t^k  - \vz^k_{t_0} + \vp^k_{t_0}\|^2  \\
& \leq \EE \|\vz_t^k - \vz^k_{t_0}\|^2 + \EE\|\vp_t^k - \vp^k_{t_0}\|^2  \\
& = \EE \|\sum_{j=t_0}^{t-1}\eta_j \vg_{j,k}\|^2 + \EE\|\vp_t^k - \vp^k_{t_0}\|^2  \\
& = \EE \|\sum_{j=t_0}^{t-1}\eta_j \vg_{j,k}\|^2 + \frac{(\beta^{t-t_0}-1)^2\beta^4}{(1-\beta)^2}\EE\|\sum_{j=0}^{t_0 - 1}\beta^{t_0-1-j}\vg_j\|^2 + \frac{\beta^4}{(1 - \beta)^2} \EE\|\sum_{j=t_0}^{t-1}\beta^{t-j-1}\vg_{j,k}\|^2  
\end{align}
where the second line is same from Lemma 3~\cite{li2019convergence}. $\ov{w}_{t_0} = \vw^k_{t_0}$ and $t_0$ is the step we communicate. In the third line, we use the \eq{\ref{eq:nasgdzt}}. In fourth line, we use \eq{\ref{eq:recursivep}}. 
The upper bound of $\EE\|\vp_t^k - \vp^k_{t_0}\|^2$ is given as follows:
\begin{align}
	\EE\|\vp_t^k - \vp^k_{t_0}\|^2 & = \EE\|\beta \vp^k_{t-1} - \frac{\beta^2}{1 - \beta} \vg_{t-1,k} - \vp^k_{t_0}\|^2\\
	&= \EE\|\beta^{t-t_0} \vp^k_{t_0} - \frac{\beta^2}{1 - \beta} \sum_{j=t_0}^{t-1}\beta^{t-j-1}\vg_{j,k} - \vp^k_{t_0}\|^2\\
	&= \EE\|(\beta^{t-t_0}-1) \vp^k_{t_0} - \frac{\beta^2}{1 - \beta} \sum_{j=t_0}^{t-1}\beta^{t-j-1}\vg_{j,k}\|^2\\
	&= \EE\|(\beta^{t-t_0}-1) \vp^k_{t_0}\|^2 + \frac{\beta^2}{1 - \beta} \EE\|\sum_{j=t_0}^{t-1}\beta^{t-j-1}\vg_{j,k}\|^2\\
	&= (\beta^{t-t_0}-1)^2\EE\|\ov{p}_{t_0}\|^2 + \frac{\beta^4}{(1 - \beta)^2} \EE\|\sum_{j=t_0}^{t-1}\beta^{t-j-1}\vg_{j,k}\|^2\\
	&= (\beta^{t-t_0}-1)^2\frac{\beta^4}{(1-\beta)^2}\EE\|\sum_{j=0}^{t_0 - 1}\beta^{t_0-1-j}\vg_j\|^2 + \frac{\beta^4}{(1 - \beta)^2} \EE\|\sum_{j=t_0}^{t-1}\beta^{t-j-1}\vg_{j,k}\|^2 \label{eq:nagcvx5}
\end{align}

\begin{align}
	\EE \|\overline{\vz}_t - \vw_t^k\|^2 & =  \EE \|\overline{\vw}_t + \overline{\vp}_{t}  - \vw_t^k\|^2\\ 
 & \leq  \EE \|\overline{\vw}_t -\vw_t^k \|^2 + \EE\| \overline{\vp}_{t}\|^2  \\
 &  =\EE \|\overline{\vw}_t -\vw_t^k \|^2 + \frac{\beta^4}{(1-\beta)^2}\EE\|\sum_{j=0}^{t_0 - 1}\beta^{t_0-1-j}\vg_j\|^2 \label{eq:nagcvx6}
\end{align}
Plug in \eq{\ref{eq:nagcvx5}} and \eq{\ref{eq:nagcvx6}} into
\eq{\ref{eq:nagcvx4}}:
\begin{align}
	\EE\|\overline{\vz}_{t+1} - \vw^* \|^2 & \leq \| \overline{\vz}_{t} - \vw^*\|^2  +  \eta_t^2\EE\|\vg_t\|^2  \nonumber\\
	&  + \sum_{k=1}^K p_k \|\overline{\vz}_t - \vw_t^k\|^2  +\eta^2_t \sum_{k=1}^K p_k\|\EE \vg_{t,k}\|^2 \nonumber \\
	& + \sum_{k=1}^K p_k (\eta_t^2 \|\EE \vg_{t,k}(\overline{\vw}_t)\|^2 + \|\vw_t^k - \overline{\vw}_t\|^2) - 2\eta_t(F(\overline{w}_t) - F^*) \\
	& =  \| \overline{\vz}_{t} - \vw^*\|^2  +  \eta_t^2\EE\|\vg_t\|^2 +\eta^2_t \sum_{k=1}^K p_k\|\EE \vg_{t,k}\|^2  \nonumber\\
	& + \sum_{k=1}^K p_k \eta_t^2 \|\EE \vg_{t,k}(\overline{\vw}_t)\|^2  - 2\eta_t(F(\overline{w}_t) - F^*) \nonumber \\
	&  + \sum_{k=1}^K p_k \|\ov{p}_t\|^2 + 2\sum_{k=1}^K p_k \|\ov{w}_t - \vw_t^k\|^2  \\
	& \leq \| \overline{\vz}_{t} - \vw^*\|^2  +  \eta_t^2\EE\|\vg_t\|^2 +\eta^2_t \sum_{k=1}^K p_k\|\EE \vg_{t,k}\|^2  \nonumber\\
	& + \sum_{k=1}^K p_k \eta_t^2 \|\EE \vg_{t,k}(\overline{\vw}_t)\|^2  - 2\eta_t(F(\overline{w}_t) - F^*) \nonumber \\
	& + \sum_{k=1}^K p_k \left(\frac{\beta^4}{(1-\beta)^2}\EE\|\underline{\sum_{j=0}^{t_0 - 1}\beta^{t_0-1-j}\vg_j}\|^2 \right) \nonumber\\
	& + 2\sum_{k=1}^K p_k \left(\EE \|\sum_{j=t_0}^{t-1}\eta_j \vg_{j,k}\|^2 + \frac{(\beta^{t-t_0}-1)^2\beta^4}{(1-\beta)^2}\EE\|\underline{\sum_{j=0}^{t_0 - 1}\beta^{t_0-1-j}\vg_j}\|^2 + \frac{\beta^4}{(1 - \beta)^2} \EE\|\sum_{j=t_0}^{t-1}\beta^{t-j-1}\vg_{j,k}\|^2 \right) \\
	& \leq \| \overline{\vz}_{t} - \vw^*\|^2  - 2\eta_t(F(\overline{w}_t) - F^*)  \nonumber\\
	& +  \eta_t^2\EE\|\vg_t\|^2 +\eta^2_t \sum_{k=1}^K p_k\|\EE \vg_{t,k}\|^2 + \sum_{k=1}^K p_k \eta_t^2 \|\EE \vg_{t,k}(\overline{\vw}_t)\|^2  \nonumber \\
	& + \sum_{k=1}^K p_k \left(\frac{\beta^4(2(\beta^{t-t_0}-1)^2 + 1)}{(1-\beta)^2} \EE\|\sum_{j=0}^{t_0 - 1}\beta^{t_0-1-j}\vg_j\|^2 \right) \nonumber\\
	& + 2\sum_{k=1}^K p_k \left(\EE \|\sum_{j=t_0}^{t-1}\eta_j \vg_{j,k}\|^2  + \frac{\beta^4}{(1 - \beta)^2} \EE\|\sum_{j=t_0}^{t-1}\beta^{t-j-1}\vg_{j,k}\|^2 \right) \label{eq:nagcvx7}
\end{align}

Denote the last three lines in \eq{\ref{eq:nagcvx7}} as $C$: 
\begin{align}
C &= \eta_t^2\EE\|\vg_t\|^2 +\eta^2_t \sum_{k=1}^K p_k\|\EE \vg_{t,k}\|^2+ \sum_{k=1}^K p_k \eta_t^2 \|\EE \vg_{t,k}(\overline{\vw}_t)\|^2  \nonumber \\
% 2\sum_{k=1}^K p_k \EE \|\sum_{j=t_0}^{t-1}\eta_j \vg_{j,k}\|^2 
	& + \sum_{k=1}^K p_k \left(\underbrace{\frac{\beta^4(2(\beta^{t-t_0}-1)^2 + 1)}{(1-\beta)^2} \EE\|\sum_{j=0}^{t_0 - 1}\beta^{t_0-1-j}\vg_j\|^2}_{C_1} \right) \nonumber\\
	& + 2\sum_{k=1}^K p_k \left(\underbrace{\EE \|\sum_{j=t_0}^{t-1}\eta_j \vg_{j,k}\|^2 }_{C_2} + \underbrace{\frac{\beta^4}{(1 - \beta)^2} \EE\|\sum_{j=t_0}^{t-1}\beta^{t-j-1}\vg_{j,k}\|^2}_{C_3} \right)  \label{eq:nagcvxc}
\end{align}



We now bound $C_1$, we set $\Lambda_{t_0} = \sum_{j=0}^{t_0 - 1}\beta^{t_0-1-j} = \frac{1 - \beta^{t_0}}{1- \beta} \leq \frac{1}{1 - \beta}$, \lkxcom{assume $\beta \in (0, 1)$ and set $\alpha = \beta$}.
\begin{align}
	C_1 &=  \frac{\beta^4(2(\beta^{t-t_0}-1)^2 + 1)}{(1-\beta)^2} \EE\|\sum_{j=0}^{t_0 - 1}\beta^{t_0-1-j}\vg_j\|^2\\
        &= \eta_t^2 \frac{\beta^4(2(\beta^{t-t_0}-1)^2 + 1)}{\alpha^2} \EE\|\sum_{j=0}^{t_0 - 1}\beta^{t_0-1-j}\vg_j\|^2\\
        &= \eta_t^2  \frac{\beta^4(2(\beta^{t-t_0}-1)^2 + 1)}{\alpha^2}\Lambda_{t_0}^2 \EE\|\sum_{j=0}^{t_0 - 1}\beta^{t_0-1-j}/\Lambda_{t_0} \vg_j\|^2\\
        & \leq \eta_t^2  \frac{\beta^4(2(\beta^{t-t_0}-1)^2 + 1)}{\alpha^2}\Lambda_{t_0}^2 \sum_{j=0}^{t_0 - 1}\beta^{t_0-1-j}/\Lambda_{t_0}\EE\| \vg_j\|^2\\ 
        & = \eta_t^2 \Lambda_{t_0} \frac{\beta^4(2(\beta^{t-t_0}-1)^2 + 1)}{\alpha^2} \sum_{j=0}^{t_0 - 1}\beta^{t_0-1-j} \EE\| \vg_j\|^2\\
		& \leq  \eta_t^2 \Lambda_{t_0} \frac{\beta^4(2(\beta^{t-t_0}-1)^2 + 1)}{\alpha^2} \sum_{j=0}^{t_0 - 1}\beta^{t_0-1-j} \sum_{k=1}^N p_k G_k^2   \\
		& =  \eta_t^2 \Lambda_{t_0} \frac{\beta^4(2(\beta^{t-t_0}-1)^2 + 1)}{\beta^2} \Lambda_{t_0}  \sum_{k=1}^N p_k G_k^2  \\
		& \leq  \eta_t^2 \frac{1}{(1-\beta)^2} \frac{\beta^2(2 \times 1 + 1)}{1} \sum_{k=1}^N p_k G_k^2   \\
		& \leq  3 \eta_t^4\sum_{k=1}^N p_k G_k^2   \label{eq:nagcvxc1}
\end{align}
In the fourth line we use the Jensen's inequality. In the fifth line, we use the assumption~\ref{ass:subgrad2}. 


To bound $C_2$, we assume $\eta_t$ is non-increasing and $\eta_t  \leq \eta_{t_0}  \leq 2 \eta_t$
\begin{align}
 C_2 = & 2\sum_{k=1}^K p_k \EE \|\sum_{j=t_0}^{t-1}\eta_j \vg_{j,k}\|^2\\
 \leq& 2\sum_{k=1}^K p_k \eta_{t_0}^2 \EE \|\sum_{j=t_0}^{t-1} \vg_{j,k}\|^2
 \leq 2\sum_{k=1}^K p_k  (E-1)\sum_{j=t_0}^{t-1}\eta_{j}^2 \EE \| \vg_{j,k}\|^2\\
 \leq & 2\sum_{k=1}^K p_k  (E-1)\eta_{t_0}^2 \sum_{j=t_0}^{t-1} \EE \| \vg_{j,k}\|^2\\
 \leq & 2\sum_{k=1}^K p_k  (E-1)4\eta_{t}^2 \sum_{j=t_0}^{t-1} \EE \| \vg_{j,k}\|^2 \leq 8(E-1)\eta_{t}^2 \sum_{k=1}^K p_k \sum_{j=t_0}^{t-1} \EE \| \vg_{j,k}\|^2 \\
 \leq & 8(E-1)\eta_{t}^2 \sum_{k=1}^K p_k (E-1)G_k^2 \\
 =  & 8(E-1)^2\eta_{t}^2 \sum_{k=1}^K p_k G_k^2   \label{eq:nagcvxc2}
\end{align}
To bound $C_3$, similar to $C_1$, we set $\Lambda_{t_0}^{t} = \sum_{j=t_0}^{t - 1}\beta^{t_0-1-j} =  \frac{1 - \beta^{t-t_0}}{1- \beta} \leq \frac{1}{1 - \beta} $
\begin{align}
	C_3 & = 2\sum_{k=1}^K p_k \frac{\beta^4}{(1 - \beta)^2} \EE\|\sum_{j=t_0}^{t-1}\beta^{t-j-1}\vg_{j,k}\|^2\\
        & = 2\eta_t^2 \Lambda_{t_0}^{t}\sum_{k=1}^K p_k \frac{\beta^4}{\alpha^2} \sum_{j=t_0}^{t-1}\beta^{t-j-1} \EE\|\vg_{j,k}\|^2 \\
        & \leq 2 \eta_t^4 \sum_{k=1}^K p_k G_k^2  \label{eq:nagcvxc3}
\end{align}

Based on \eq{\ref{eq:nagcvxc}}, \eq{\ref{eq:nagcvxc1}}, \eq{\ref{eq:nagcvxc2}}, and \eq{\ref{eq:nagcvxc3}}, we can denote $C = \eta_t^2 D$, 
plug in $C$ back to \eq{\ref{eq:nagcvx7}}, we have 
\begin{align}
\EE\|\overline{\vz}_{t+1} - \vw^* \|^2 & \leq \EE \| \overline{\vz}_{t} - \vw^*\|^2  +  \eta_t^2 D   - 2\eta_t(F(\overline{w}_t) - F^*) \nonumber\\	
 2\eta_t(F(\overline{w}_t) - F^*) & \leq  \Delta_{t}  - \Delta_{t+1}  +  \eta_t^2 D    \nonumber
\end{align}

Sum two sides over time step $t$:
\begin{align}
\frac{1}{T}\sum_{t=0}^{T-1} 2 \eta (F(\overline{w}_t) - F^*) & \leq \Delta_0 - \Delta_{T} + \sum_{t=0}^{T-1} \eta^2 D \nonumber\\
\frac{1}{T}\sum_{t=0}^{T-1} 2 \eta (F(\overline{w}_t) - F^*) & \leq \Delta_0/T  + \sum_{t=0}^{T-1} \eta^2 D/T\nonumber\\
 F(\hat{w}_t) - F^* & \leq \frac{\Delta_0}{2T\eta}   + \frac{\eta D}{2}\nonumber\\
 F(\hat{w}_t) - F^* & \leq \sqrt{\frac{\Delta_0D}{T}} \nonumber\\
\end{align}
where $\hat{w}_t = \frac{1}{T}\sum_{t=0}^{T-1} \ov{w}_t$, $\eta = \sqrt{\frac{\Delta_0}{DT}}$.



\begin{itemize}
	\item Update NASG or MSG.
	\item Store the momentum every $E$~\cite{huo2020faster} , every step~\cite{liu2019accelerating}, rely on L-smoothness.
\end{itemize}















\end{document}



\iffalse

\endif